
%MAKRO DATEI


%Aktuelle Makros
%%%%%%%%%%%%%%%%
%%%%%%%%%%%%%%%%
%%%%%%%%%%%%%%%%

\newcommand{\Gd}{G_\delta} % eine G delta menge
\newcommand{\Fs}{F_\sigma} % eine F sigma Menge
\newcommand{\I}{\mathcal{I}}
\newcommand{\mcN}{\mathcal{N}}
\newcommand{\question}[1]{\todo[noline,linecolor=black,backgroundcolor=white]{\textit{Question:} #1}}
\newcommand{\task}[1]{\todo[noline,linecolor=black,backgroundcolor=pink]{\textit{Task:} #1}}
\newcommand{\note}[1]{\todo[noline,linecolor=black,backgroundcolor=white]{\textit{Note:} #1}}
\newcommand{\corrected}[1]{\todo[linecolor=green,backgroundcolor=green]{\textit{Corrected up to this point.} #1}}
\newcommand{\bfn}{\textbf{n}}
\newcommand{\outermu}{\mu^*}
\newcommand{\innermu}{\mu_*}
\newcommand{\Amu}{\mathcal{A}_\mu}
\newcommand{\tAmu}{\tilde{\A}_\mu}
\newcommand{\mumeasurable}{$\mu $-measurable }
\newcommand{\As}{\mathcal{A}_{*}} % universally measurable sets
\renewcommand{\P}{\mathbb{P}}
\newcommand{\X}{\mathbb{X}}
\newcommand{\bbF}{\mathbb{F}}
\newcommand{\Rp}{[0,\infty)} % change it back if you want to
\newcommand{\Xt}{(X_t)_{t\geq 0}}

\newcommand{\Ftp}{\mathcal{F}_{t^+}}
\newcommand{\Fsp}{\mathcal{F}_{s^+}}
\newcommand{\REF}{{\color{orange}\textbf{REF}} } % Change to color to orange later on so that you can fill in the missing references.

% BASS APPROACH
\newcommand{\Ps}{\mathbb{P}^{*}} % outer probability
\newcommand{\Ko}{\mathcal{K}^{0}}
\newcommand{\K}{\mathcal{K}}
\newcommand{\Kd}{\mathcal{K}_{\delta}}
\newcommand{\tapproximable}{$t$-approximable }
\newcommand{\Tt}{\mathcal{T}_t}
\newcommand{\pX}{\rho^{X}}

\newcommand{\Lo}{\mathcal{L}_{0}}
\newcommand{\Li}{\mathcal{L}_{1}}
\newcommand{\Ls}{\mathcal{L}_{\sigma}}
\newcommand{\Lsd}{\mathcal{L}_{\sigma\delta}}

\newcommand{\tn}{\tau_{n}}
\newcommand{\Bnm}{B_{nm}}
\newcommand{\Cnm}{C_{nm}}
\newcommand{\xnk}{x_{n_{k}}}
\newcommand{\xinf}{x_{\infty}}
\newcommand{\nk}{n_k}
\newcommand{\pipX}{{\pi\circ\rho}^{X}}
%\newcommand{\tilA}{\tilde{A}}

\newcommand{\Ytd}{Y_{t}^{\delta}}
\newcommand{\UAd}{U_{A}^{\delta}}
\newcommand{\Ftd}{\mathcal{F}_{t}^{\delta}}

%KLASSISCHE ABBILDUNGEN%
\newcommand{\skp}[1]{\left\langle #1 \right\rangle}
\newcommand{\pdv}[2]{\frac{\partial #1}{\partial #2}}
\newcommand{\ppdv}[3]{\frac{\partial^{2} #1}{\partial #2 \partial #3}}
\newcommand{\norm}[1]{{\left\lVert \ifthenelse{\equal{#1}{}}{\, \cdot \,}{#1}\right\rVert}} %durch ifthenelse wird ein default argument eingerichtet was greift, sobald man \norm{} schreibt, dabei ist dann "#1" leer
\newcommand{\betrag}[1]{\left| \ifthenelse{\equal{#1}{}}{\, \cdot \,}{#1} \right|}
\newcommand{\inhalt}[1]{\mu{\left( #1 \right)}} 
\newcommand{\mass}[1]{\mu{\left( #1 \right)}}
\newcommand{\lmass}[1]{\lambda{\left( #1 \right)}}
\newcommand{\smass}[1]{\mu^{\ast}{\left( #1 \right)}}
\newcommand{\wmass}[1]{\mathbf{P}{\left( #1 \right)}}
\newcommand{\indikator}[1]{\mathds{1}_{#1}}
\newcommand{\ugauss}[1]{{\lfloor \ifthenelse{\equal{#1}{}}{\, \cdot \,}{#1} \rfloor}}
\newcommand{\ogauss}[1]{{\lceil \ifthenelse{\equal{#1}{}}{\, \cdot \,}{#1} \rceil}}

















