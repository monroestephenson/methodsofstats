% Für Hausaufgaben article, für Skripte report
% \documentclass[open=any, 11pt,paper=A4,headings=big,titlepage = firstiscover]{scrreprt}
\documentclass[open=any, 11pt,paper=A4]{scrreprt}


\KOMAoptions{}


  
%\usepackage[colorinlistoftodos]{todonotes}
\usepackage[
backend=biber,
style=alphabetic, % Label wird verschönert
sorting=ynt,
backrefstyle=two % sortiert nach year, name und dann title
]{biblatex}
\usepackage[textsize=tiny,linecolor=black,backgroundcolor=white]{todonotes}
\usepackage{0_einstellungen}
\addbibresource{0_bibliography.bib} %Imports bibliography file
\usepackage[font=footnotesize]{caption}
\usepackage{wrapfig}
\usepackage{lmodern}
\usepackage{xcolor}
\usepackage{pdfpages,datetime}


% \author{Said Kassner}
% \title{Lecture Notes}

% MAKROS
%%%%%%%%%%%%%%%%%

%MAKRO DATEI


%Aktuelle Makros
%%%%%%%%%%%%%%%%
%%%%%%%%%%%%%%%%
%%%%%%%%%%%%%%%%

\newcommand{\Gd}{G_\delta} % eine G delta menge
\newcommand{\Fs}{F_\sigma} % eine F sigma Menge
\newcommand{\I}{\mathcal{I}}
\newcommand{\mcN}{\mathcal{N}}
\newcommand{\question}[1]{\todo[noline,linecolor=black,backgroundcolor=white]{\textit{Question:} #1}}
\newcommand{\task}[1]{\todo[noline,linecolor=black,backgroundcolor=pink]{\textit{Task:} #1}}
\newcommand{\note}[1]{\todo[noline,linecolor=black,backgroundcolor=white]{\textit{Note:} #1}}
\newcommand{\corrected}[1]{\todo[linecolor=green,backgroundcolor=green]{\textit{Corrected up to this point.} #1}}
\newcommand{\bfn}{\textbf{n}}
\newcommand{\outermu}{\mu^*}
\newcommand{\innermu}{\mu_*}
\newcommand{\Amu}{\mathcal{A}_\mu}
\newcommand{\tAmu}{\tilde{\A}_\mu}
\newcommand{\mumeasurable}{$\mu $-measurable }
\newcommand{\As}{\mathcal{A}_{*}} % universally measurable sets
\renewcommand{\P}{\mathbb{P}}
\newcommand{\X}{\mathbb{X}}
\newcommand{\bbF}{\mathbb{F}}
\newcommand{\Rp}{[0,\infty)} % change it back if you want to
\newcommand{\Xt}{(X_t)_{t\geq 0}}

\newcommand{\Ftp}{\mathcal{F}_{t^+}}
\newcommand{\Fsp}{\mathcal{F}_{s^+}}
\newcommand{\REF}{{\color{orange}\textbf{REF}} } % Change to color to orange later on so that you can fill in the missing references.

% BASS APPROACH
\newcommand{\Ps}{\mathbb{P}^{*}} % outer probability
\newcommand{\Ko}{\mathcal{K}^{0}}
\newcommand{\K}{\mathcal{K}}
\newcommand{\Kd}{\mathcal{K}_{\delta}}
\newcommand{\tapproximable}{$t$-approximable }
\newcommand{\Tt}{\mathcal{T}_t}
\newcommand{\pX}{\rho^{X}}

\newcommand{\Lo}{\mathcal{L}_{0}}
\newcommand{\Li}{\mathcal{L}_{1}}
\newcommand{\Ls}{\mathcal{L}_{\sigma}}
\newcommand{\Lsd}{\mathcal{L}_{\sigma\delta}}

\newcommand{\tn}{\tau_{n}}
\newcommand{\Bnm}{B_{nm}}
\newcommand{\Cnm}{C_{nm}}
\newcommand{\xnk}{x_{n_{k}}}
\newcommand{\xinf}{x_{\infty}}
\newcommand{\nk}{n_k}
\newcommand{\pipX}{{\pi\circ\rho}^{X}}
%\newcommand{\tilA}{\tilde{A}}

\newcommand{\Ytd}{Y_{t}^{\delta}}
\newcommand{\UAd}{U_{A}^{\delta}}
\newcommand{\Ftd}{\mathcal{F}_{t}^{\delta}}

%KLASSISCHE ABBILDUNGEN%
\newcommand{\skp}[1]{\left\langle #1 \right\rangle}
\newcommand{\pdv}[2]{\frac{\partial #1}{\partial #2}}
\newcommand{\ppdv}[3]{\frac{\partial^{2} #1}{\partial #2 \partial #3}}
\newcommand{\norm}[1]{{\left\lVert \ifthenelse{\equal{#1}{}}{\, \cdot \,}{#1}\right\rVert}} %durch ifthenelse wird ein default argument eingerichtet was greift, sobald man \norm{} schreibt, dabei ist dann "#1" leer
\newcommand{\betrag}[1]{\left| \ifthenelse{\equal{#1}{}}{\, \cdot \,}{#1} \right|}
\newcommand{\inhalt}[1]{\mu{\left( #1 \right)}} 
\newcommand{\mass}[1]{\mu{\left( #1 \right)}}
\newcommand{\lmass}[1]{\lambda{\left( #1 \right)}}
\newcommand{\smass}[1]{\mu^{\ast}{\left( #1 \right)}}
\newcommand{\wmass}[1]{\mathbf{P}{\left( #1 \right)}}
\newcommand{\indikator}[1]{\mathds{1}_{#1}}
\newcommand{\ugauss}[1]{{\lfloor \ifthenelse{\equal{#1}{}}{\, \cdot \,}{#1} \rfloor}}
\newcommand{\ogauss}[1]{{\lceil \ifthenelse{\equal{#1}{}}{\, \cdot \,}{#1} \rceil}}

















 
%%%%%%%%%%%%%%%%%

% In den folgenden drei Zeilen sorgst du dafür dass es kein Problem mehr damit gibt dass du auf der Titelseite \sc benutzt
\makeatletter
\DeclareOldFontCommand{\sc}{\normalfont\scshape}{\@nomath\sc}
\makeatother
\makeatletter
\DeclareOldFontCommand{\it}{\normalfont\itshape}{\@nomath\it}
\makeatother

%%%%%%%%%%%%%%%%%
%%%%%%%%%%%%%%%%%


\begin{document}
% \includepdf{0_title.pdf}

%\KOMAoptions{DIV=15}
%\KOMAoptions{DIV=22}
% In diesem Dokument wird der TitleHead über eine Tabelle eingebunden
\titlehead
{\centering
\includegraphics[scale=1]{bilder/hu-logo}
}


% Title content
\subject{\sc Humboldt-Universität zu Berlin}
\title{Methods of Statistics (M25)}
\author{Lecturer: {Prof. Dr. Sven Wang}\\
WS 24/25}
\date{Last Update: \today}



\maketitle	% du kannst am ende von diesem Ding auch eine zahl durch zB [3] einbinden, dass ist dann die zahl mit der die Titelseite gezählt wird also \maketitle[3]

% Du kannst hier auch alles in die titlepage umbegung packen, weiß aber nicht genau was sich dann wirklich ändert, scheinbar macht man das aber so meistens


 % Kann man auch mit include machen, so war es ursprünglich
%\newpage\null\thispagestyle{empty}\newpage % leere Seite nach dem Titelblatt



\tableofcontents % Inhaltsverzeichnis direkt nach der Title Page
\thispagestyle{empty}


%%%%%%%%%%%%%%%%%%%%%%%%%%%%%%%%%%%%%%%%%%%%%%%%%%%%%%%%%%%%%%%%%%%%%%%%%%%%%%%%%%%%%%%%%%%%%%%%%%%%%%%%%%%%%%%%%%%%%%%%%%%%%%%%%%%%%%%%%%%%%%%%%%%%%%%%%%%%%%%%%%%%%%%%%%%%%%%%%%%%%%%%%%%%%%%%%%%%%%%%%%%%%%%%%%%%%%%%%%%%%%%%%%%%%%%%%%%%%%%%%%%%%%%%%%%%%%%%%%%%%%%%%%%%%%%%%%%%%%%%%%%%%%%%%%%%%%%%%%%%%%%%%%%%%%%%%%%%%%%%%%%%%%%%%%%%%%%%%%%%%%%%%%%%%%%%%%%%%%%%%%%%%%%%%%%%%%%%%%%%%%%%%%%%%%%%%%%%%%%%%%%%%%%%%%%%%%%%%%%%%%%%%%%%%%%%%%%%%%%%%%%%%%%%%%%%%%%%%%%%%%%%%%%%%%%%%%%%%%%%%%%%%%%%%%%%%%%%%%%%%%%%%%%%%%%%%%%%%%%%%%%%%%%%%%%%%%%%%%%%%%%%%%%%%%%%%%%%%%%%%%%%%%%%%%%%%%%%%%%%%%%%%%%%%%%%%%%%%%%%%%%%%%%%%%%%%%%%%%%%%%%%%%%%%%%%%%%%%%%%%%%%%%%%%%%%%%%%%%%%%%%%%%%%%%%%%%%%%%%%%%%%%%%%%%%%%%%%%%%%%%%%%%%%%%%%%%%%%%%%%%%%%%%%%%%%%%%%%%%%%%%%%%%%%%%%%%%%%%%%%%%%%%%%%%%%%%%%%%%%%%%%%%%%%%%%%%%%%%%%%%%%%%%%%%%%%%%%%%%%%%%%%%%%%%%%%%%%%%%%%%%%%%%%%%%%%%%%%%%%%%%%%%%%%%%%%%%%%%%%%%%%%%%%%%%%%%%%%%%%%%%%%%%%%%%%%%%%%%%%%%%%%%%%%%%%%%%%%%%%%%%%%%%%%%%%%%%%%%%%%%%%%%%%%%%%%%%%%%%%%%%%%%%%%%%%%%%%%%%%%%%%%%%%%%%%%%%%%%%%%%%%%%%%%%%%%%%%%%%%%%%%%%%%%%%%%%%%%%%%%%%%%%%%%%%%%%%%%%%%%%%%%%%%%%%%%%%%%%%%%%%%%%%%%%%%%%%%%%%%%%%%%%%%%%%%%%%%%%%%%%%%%%%%%%%%%%%%%%%%%%%%%%%%%%%%%%%%%%%%%%%%%%%%%%%%%%%%%%%%%%%%%%%%%%%%%%%%%%%%%%%%%%%%%%%%%%%%%%%%%%%%%%%%%%%%%%%%%%%%%%%%%%%%%%%%%%%%%%%%%%%%%%%%%%%%%%%%%%%%%%%%%%%%%%%%%%%%%%%%%%%%%%%%%%%%%%%%%%%%%%%%%%%%%%%%%%%%%%%%%%%%%%%%%%%%%%%%%%%%%%%%%%%%%%%%%%%%%%%%%%%%%%%%%%%%%%%%%%%%%%%%%%%%%%%%%%%%%%%%%%%%%%%%%%%%%%%%%%%%%%%%%%%%%%%%%%%%%%%%%%%%%%%%%%%%%%%%%%%%%%%%%%%%%%%%%%%%%%%%%%%%%%%%%%%%%%%%%%%%%%%%%%%%%%%%%%%%%%%%%%%%%%%%%%%%%%%%%%%%%%%%%%%%%%%%%%%%%%%%%%%%%%%%%%%%%%%%%%%%%%%%%%%%%%%%%%%%%%%%%%%%%%%%%%%%%%%%%%%%%%%%%%%%%%%%%%%%%%%%%%%%%%%%%%%%%%%%%%%%%%%%%%%%%%%%%%%%%%%%%%%%%%%%%%%%%%%%%%%%%%%%%%%%%%%%%%%%%%%%%%%%%%%%%%%%%%%%%%%%%%%%%%%%%%%%%%%%%%%%%%%%%%%%%%%%%%%%%%%%%%%%%%%%%%%%%%%%%%%%%%%%%%%%%%%%%%%%%%%%%%%%%%%%%%%%%%%%%%%%%%%%%%%%%%%%%%%%%%%%%%%%%%%%%%%%%%%%%%%%%%%%%%%%%%%%%%%%%%%%%%%%%%%%%%%%%%%%%%%%%%%%%%%%%%%%%%%%%%%%%%%%%%%%%%%%%%%%%%%%%%%%%%%%%%%%%%%%%%%%%%%%%%%%%%%%%%%%%%%%%%%%%%%%%%%%%%%%%%%%%%%%%%%%%%%%%%%%%%%%%%%%%%%%%%%%%%%%%%%%%%%%%%%%%%%%%%%%%%%%%%%%%%%%%%%%%%%%%%%%%%%%%%%%%%%%%%%%%%%%%%%%%%%%%%%%%%%%%%%%%%%%%%%%%%%%%%%%%%%%%%%%%%%%%%%%%%%%%%%%%%%%%%%%%%%%%%%%%%%%%%%%%%%%%%%%%%%%%%%%%%%%%%%%%%%%%%%%%%%%%%%%%%%%%%%%%%%%%%%%%%%%%%%%%%%%%%%%%%%%%%%%%%%%%%%%%%%%%%%%%%%%%%%%%%%%%%%%%%%%%%%%%%%


% REWORKED CODE

%%%%%%%%%%%%%%%%%%%%%%%%%%%%%%%%%%%%%%%%%%%%%%%%%%%%%%%%%%%%%%%%%%%%%%%%%%%%%%%
%%%%%%%%%%%%%%%%%%%%%%%%%%%%%%%%%%%%%%%%%%%%%%%%%%%%%%%%%%%%%%%%%%%%%%%%%%%%%%%
\chapter{Intro and Disclaimer}\label{ch:intro}
%%%%%%%%%%%%%%%%%%%%%%%%%%%%%%%%%%%%%%%%%%%%%%%%%%%%%%%%%%%%%%%%%%%%%%%%%%%%%%%
%%%%%%%%%%%%%%%%%%%%%%%%%%%%%%%%%%%%%%%%%%%%%%%%%%%%%%%%%%%%%%%%%%%%%%%%%%%%%%%

These lecture notes are based on the material presented by Professor Wang during class and are written by students. They may contain errors or omissions. Please refer to the in-person lectures and the literature on Moodle for accurate and authoritative information. 
If you find an error or want to help, please send an email to:
\begin{center}
    \href{mailto:said.kassner@student.hu-berlin.de}{said.kassner@student.hu-berlin.de}
\end{center}

\begin{center}
    \href{mailto:stephensonmonroe@gmail.com}{stephensonmonroe@gmail.com}
\end{center}

\begin{center}
    \href{mailto:salihiad@hu-berlin.de}{salihiad@hu-berlin.de}
\end{center}




%%%%%%%%%%%%%%%%%%%%%%%%%%%%%%%%%%%%%%%%%%%%%%%%%%%%%%%%%%%%%%%%%%%%%%%%%%%%%%%
%%%%%%%%%%%%%%%%%%%%%%%%%%%%%%%%%%%%%%%%%%%%%%%%%%%%%%%%%%%%%%%%%%%%%%%%%%%%%%%
\chapter{Basic Statistical Concepts}\label{ch:intro}
%%%%%%%%%%%%%%%%%%%%%%%%%%%%%%%%%%%%%%%%%%%%%%%%%%%%%%%%%%%%%%%%%%%%%%%%%%%%%%%
%%%%%%%%%%%%%%%%%%%%%%%%%%%%%%%%%%%%%%%%%%%%%%%%%%%%%%%%%%%%%%%%%%%%%%%%%%%%%%%

\section*{Lecture 1}
Here is the literature that the class is based on.

\textbf{Literature:}
\begin{itemize}
    \item WS 19/20 R. Altmeyer \textit{"Gliederung Methoden der Statistik"}
    \item L. Wasserman, \textit{All of Statistics}
    \item M. Trabs, K. Krenz, M. Jirak and M. Reiss. \textit{Methoden der Statistik.}
    \item Hastie, Tibshirani, et al., \textit{Elements of Statistical Learning}
\end{itemize}

Let’s start with a (simplest possible) example:

\begin{example}[Polling]\label{ex:polling}
Consider a poll with two answers A and B (representing political parties).
\begin{itemize}
    \item \( N \) = total number of votes
    \item \( M \) = total number of votes supporting party A
\end{itemize}
\end{example}

\textbf{Poll Definitions:}
\begin{itemize}
    \item \( n \) = size of the poll
    \item \( x = (x_1, ..., x_n) \) = responses, where:
    \[
    x_i =
    \begin{cases} 
    0 & \text{if the i\text{-}th person supports B} \\
    1 & \text{if the i\text{-}th person supports A}
    \end{cases}
    \]
\end{itemize}

\textbf{Additional Assumptions:}
\begin{itemize}
    \item \( n \)-times, we select a person randomly from the set \( \{1, ..., N\} \), and record their (truthful) response.
    \item Every asked person responds (i.e., no selection bias).
    \item People can be asked repeatedly.
\end{itemize}

\textbf{Aim of the Poll:} The aim of the poll is to estimate the fraction of party A supporters. This can be written as:

\[
\theta = \frac{M}{N} \in [0, 1]
\]

An intuitive estimate of \( \theta \) is:

\[
\hat{\theta} = \frac{1}{n} \sum_{i=1}^{n} x_i
\]

\textbf{Question:} Is this a good (or best possible) estimator? What properties does it have?

To answer this, we formalize some statistical notions.

\begin{definition}[Sample space]\label{def:samplespace}
A \textit{sample space} is a measurable space \((\mathcal{X}, \mathcal{F})\), i.e., a set \(\mathcal{X}\) with a \(\sigma\)-algebra \(\mathcal{F}\), in which our statistical observations take values.
\end{definition}

\begin{definition}[Statistical model]\label{def:statmodel}
Let \((\mathcal{X}, \mathcal{F})\) be some sample space and let \(\Theta\) be a set, which we call the \textit{parameter space}. A \textit{statistical model} is a family of probability measures \(\{P_\theta : \theta \in \Theta\}\) on \((\mathcal{X}, \mathcal{F})\).
\end{definition}

\begin{remark}
Often, \((\mathcal{X}, \mathcal{F})\) is a "product space." For example, in Example \ref{ex:polling}, \(\mathcal{X} = \{0, 1\}^n\), and each \(P_\theta\) is a product distribution, i.e., \(x_1, \dots, x_n\) are independent, identically distributed. Then we say \(\{P_\theta : \theta \in \Theta\}\) is an \textit{iid statistical model}.
\end{remark}

\begin{remark}[Back to Example \ref{ex:polling}]
Here:
\begin{itemize}
    \item \(\mathcal{X} = \{0, 1\}^n\)
    \item \(\Theta = [0, 1]\)
    \item \(\mathcal{F} = \mathcal{P}(\{0, 1\}^n)\)
    \item \(P_\theta = (\text{Bernoulli}(\theta))^{\otimes n}\)
\end{itemize}
\end{remark}

\begin{remark}
If every person could only be asked once, we would have \( P_\theta = \text{Hypergeometric}(N, M, n) \), which "converges" to the Bernoulli model as \( N, M \to \infty \). 
We might have to discretize \(\Theta\) and take \( \theta = \frac{M}{N} \). 
(Exercise: Think about it!)
\end{remark}

\chapter{Parameter Estimation}
Assume that \( \Theta \subseteq \mathbb{R}^p \), for \( p \geq 1 \). This is the setting of parametric statistics. 
[Assume \(\Theta\) is measurable.]

\begin{definition}[Estimator]\label{def:estimator}
An estimator for \( \theta \in \Theta \) is any measurable function:
\[
\hat{\theta}: (\mathcal{X}, \mathcal{F}) \to \Theta.
\]
Any function that, based on some data \( x \in \mathcal{X} \), outputs a guess / estimate \( \hat{\theta}(x) \in \Theta \).
\end{definition}


\section*{Lecture 2}

\textbf{Last time:} Statistical model = family of probability measures on $(\mathcal{X}, \mathcal{F})$ indexed by $\theta \in \Theta$. 

\textbf{Sample space:} $(\mathcal{X}, \mathcal{F})$

\textbf{Estimator:} = measurable function $(\mathcal{X}, \mathcal{F}) \to \Theta$

Now, what are some desirable properties we would like to have?

\begin{definition}[Unbiased estimator]\label{def:unbiased}
Let $\Theta = \mathbb{R}^p$ (measurable), $p \geq 1$. An estimator $\hat{\theta}$ is unbiased if 
\[
    \mathbb{E}_{\theta}[\hat{\theta}] = \mathbb{E}_{\mathbb{P}_\theta}[\hat{\theta}] = \theta, \text{ for all } \theta \in \Theta.
\]
Where $\mathbb{E}_{\theta}[\cdot] = \mathbb{E}_{\mathbb{P}_\theta}[\cdot]$ denotes expectation under the law $\mathbb{P}_\theta$.
\end{definition}


In more explicit terms:
\[
    \mathbb{E}_{x \sim \mathbb{P}_\theta}[\hat{\theta}(x)] = \theta \quad \forall \theta
\]

\begin{remark}[Unbiasedness]\label{rem:unbiasedness}
Unbiasedness means "no systematic errors." However, we'd also like a "good" $\hat{\theta}$ to be concentrated around the data-generating parameter.
\end{remark}

\begin{definition}[Consistent estimator]\label{def:consistent}
Let $(\mathbb{P}_{\theta}^n)_{\theta \in \Theta}$ be a sequence of statistical models ($n \geq 1$), on the same parameter space $\Theta$ not depending on $n \geq 1$.

Let $\hat{\theta}_n$ be a sequence of estimators. Then $\hat{\theta}_n$ is called consistent if for every $\theta \in \Theta$,
\[
    \hat{\theta}_n \xrightarrow{\mathbb{P}_\theta^n} \theta
\]
or explicitly, for every $\epsilon > 0$,
\[
    \lim_{n \to \infty} \mathbb{P}^n_\theta (|\hat{\theta}_n - \theta| > \epsilon) = 0.
\]
\end{definition}

\textbf{Back to Example \ref{ex:polling}:}
\begin{itemize}
    \item $X_i = \{0, 1\}^n$
    \item $\Theta = [0, 1]$
    \item $\mathbb{P}_\theta^n = \text{Bernoulli}(\theta)^{\otimes n}$
    \item $\hat{\theta}_n = \frac{1}{n} \sum_{i=1}^{n} X_i$
\end{itemize}

\textbf{Unbiasedness:} \\
Let $\theta \in \Theta$, then
\[
    \mathbb{E}_\theta \left[ \frac{1}{n} \sum_{i=1}^{n} X_i \right] = \frac{1}{n} \sum_{i=1}^{n} \mathbb{E}_\theta [X_i] = \frac{1}{n} \sum_{i=1}^{n} \theta = \theta.
\]
Thus, $\mathbb{E}_\theta [\hat{\theta}_n] = \theta$.

\textbf{Consistency:}
\begin{itemize}
    \item We could use the Weak Law of Large Numbers (WLLN).
    \item Alternatively,
\[
    \text{Var}_\theta(\hat{\theta}_n) = \frac{1}{n^2} \sum_{i=1}^{n} \text{Var}_\theta (X_i) = \frac{1}{n^2} \sum_{i=1}^{n} \theta (1-\theta) = \frac{\theta(1-\theta)}{n}
\]
which tends to zero as $n \to \infty$.
\end{itemize}

It follows: For every $\epsilon > 0$,
\[
    \mathbb{P}^n_\theta(|\hat{\theta}_n - \theta| > \epsilon) = \mathbb{P}^n_\theta(|\hat{\theta}_n - \mathbb{E}[\hat{\theta}_n]| > \epsilon)
\]
By Markov's inequality:
\[
    \mathbb{P}^n_\theta(|\hat{\theta}_n - \mathbb{E}[\hat{\theta}_n]| > \epsilon) \leq \frac{\mathbb{E}_\theta \left[ (\hat{\theta}_n - \mathbb{E}[\hat{\theta}_n])^2 \right]}{\epsilon^2} = \frac{\text{Var}_\theta (\hat{\theta}_n)}{\epsilon^2} = \frac{\theta(1-\theta)}{n \epsilon^2}
\]
which tends to zero as $n \to \infty$. Thus,
\[
    (\hat{\theta}_n : n \geq 1) \text{ is consistent. } \quad \square
\]

\section{Maximum Likelihood Principle}

Is there another way to motivate $\hat{\theta}_n = \frac{1}{n} \sum_{i=1}^{n} X_i$? Yes, it turns out it is the maximum likelihood estimator, i.e.,
\[
    \text{MLE} = \text{"parameter which assigns the highest probability to the observed data."}
\]

\textbf{In our example}, each $\mathbb{P}_\theta^n$ has a probability density (likelihood)
\[
    \mathbb{P}_\theta^n(x) = \prod_{i=1}^{n} \mathbb{P}_\theta(x_i) = \prod_{i=1}^{n} \theta^{x_i} (1-\theta)^{1-x_i}
\]
\[
    = \theta^{\sum_{i=1}^{n} x_i} (1-\theta)^{n-\sum_{i=1}^{n} x_i}.
\]

Fixing $x \in \{0,1\}^{n}$ and maximizing in $\theta \in [0,1]$ gives the following:

\begin{itemize}
    \item If $\sum_{i=1}^{n} x_i = 0$, then $\hat{\theta}_n = 0$ is the maximizer.
    \item If $\sum_{i=1}^{n} x_i = n$, then $\hat{\theta}_n = 1$ is the maximizer.
    \item If $\hat{\theta}_n \in \{1, \ldots, n-1\}$, then writing $S_n = \sum_{i=1}^{n} x_i$ gives:
\end{itemize}

\[
    \frac{\partial}{\partial \theta} \mathbb{P}_\theta^n(x) = S_n \theta^{S_n-1} (1-\theta)^{n-S_n-1} - (n-S_n)\theta^{S_n} (1-\theta)^{n-S_n-1} = 0
\]
\[
    \Leftrightarrow S_n (1 - \theta) - \theta (n - S_n)
\]
\[
    \Leftrightarrow \theta = \frac{S_n}{n}.
    \quad \square
\]

\begin{definition}[Dominated statistical model \& MLE]\label{def:dominatedmodel}
A model $(\mathbb{P}_\theta^n)_{\theta \in \Theta}$ is called dominated if there exists a measure $\mu$ on $(\mathcal{X}, \mathcal{F})$ such that for every $\theta \in \Theta$, $\mathbb{P}_\theta \ll \mu$ or equivalently (by Radon-Nikodym), for all $\theta \in \Theta$, there is a probability density $\frac{\mathrm{d} \mathbb{P}_\theta}{\mathrm{d}\mu}$ of $\mathbb{P}_\theta$ with respect to $\mu$.

The MLE is defined as any $\hat{\theta} \in \Theta$ that maximizes the function
\[
    \theta \mapsto \frac{\mathrm{d} \mathbb{P}_\theta}{\mathrm{d}\mu}(x) = \mathbb{P}_\theta(x).
\]
\end{definition}

\begin{remark}[Caveats]\label{rem:caveats}
\begin{itemize}
    \item MLE might not be unique.
    \item MLE might not exist.
    \item It's not always clear that some selection $\hat{\theta}(x) \in \arg \max_{\theta} \mathbb{P}_\theta(x)$ is a measurable function of $x \in \mathcal{X}$. However, there are measurable selection theorems that permit a measurable choice of $\hat{\theta}$ under very general conditions.
\end{itemize}
\end{remark}

\begin{remark}\label{rem:mu}
In all the models we study, we will work with the Lebesgue measure (for continuous data) or the counting measure (for discrete data).
\end{remark}

\begin{example}[Normal model]\label{ex:normalmodel}
Consider random samples $X_1, \dots, X_n \sim N(\mu, \sigma^2)$ for some unknown $\mu \in \mathbb{R}$, $\sigma^2 > 0$, and let $\lambda$ denote the Lebesgue measure on $\mathbb{R}^n$.

\[
    \theta = (\mu, \sigma^2) \in \mathbb{R} \times (0, \infty).
\]
Then the likelihood is:
\[
    L(\mu, \sigma^2 \mid x) = \prod_{i=1}^{n} \frac{1}{\sqrt{2 \pi \sigma^2}} \exp \left( - \frac{1}{2} \left( \frac{X_i - \mu}{\sigma} \right)^2 \right)
\]
\[
    = (2 \pi \sigma^2)^{-n/2} \exp \left( - \frac{1}{2 \sigma^2} \norm{X-\mu\cdot 1_n}^2 \right),
\]
where by $1_n$ we denote the vector of ones of dimension $n$.

Here, the MLE is given as:
\[
    \hat{\mu} = \frac{1}{n} \sum_{i=1}^{n} X_i \quad \text{[Sample mean]}, \quad \hat{\sigma}^2 = \frac{1}{n} \sum_{i=1}^{n} (X_i - \hat{\mu})^2 \quad  \text{[Sample variance]}.
\]
\[
    \mathbb{E}_\theta[\hat{\mu}] = \mu, \quad \mathbb{E}_\theta[\hat{\sigma}^2] = \frac{n-1}{n} \sigma^2, \quad \text{so } \hat{\sigma}^2 \text{ is biased.}
\]
\[
    \Rightarrow \text{MLE is not always unbiased, and not always a "good" method.}
\]

\end{example}

\section{Bayesian method}

\paragraph{Motivation}
In Bayesian statistics, a key element is the prior distribution, which we denote by $\pi$, reflecting our "beliefs" about the parameter $\theta \in \Theta$ before observing data ($\pi$ is a probability measure on $\Theta$).

A prior $\pi$, together with a model $(P_\theta : \theta \in \Theta)$, gives rise to a joint probability distribution for the pair $(\theta, x) \in \Theta \times \mathcal{X}$.

The Bayesian approach bases statistical inference on the posterior distribution of $\theta$ conditioned on $x$.

\textbf{Joint probability:}
\[
    (\theta, x) \mapsto \pi(\theta) P_\theta(x)
\]
\hspace{4cm} \text{conditional distribution of } $x \mid \theta$

\textbf{Posterior:}
\[
    \pi(\theta \mid x) = \frac{\pi(\theta) P_\theta(x)}{\int_{\Theta} \pi(\theta) P_\theta(x) \, d\theta}
\]

\begin{remark}
Bayesian methods automatically generate "error bars" because the posterior is not an estimator but a whole probability distribution.
\end{remark}



\section*{Lecture 3}

\begin{definition}[Prior, Posterior, Bayes' Rule]\label{def:prior, posterior, bayes rule}
Let $\mathcal{F}_\Theta$ be a $\sigma$-algebra on $\Theta$, and suppose 
\[
\left\{ P_\theta : \theta \in \Theta \right\}
\]
is a dominated statistical model with densities $p_\theta(x)$, and assume that 
\[
(\theta, x) \mapsto p_\theta(x)
\]
is "jointly measurable" (i.e., w.r.t. $\sigma(\mathcal{F}_\Theta \times \mathcal{F})$).

Let $\Pi$ be a prior distribution on $\Theta$, with density $\pi(\theta)$ w.r.t. measure $\nu(\cdot)$. Then, define the posterior density
\[
\pi(\theta \mid x) := \frac{p_\theta(x) \pi(\theta)}{\int_\Theta p_{\tilde{\theta}}(x) \, d\Pi(\tilde{\theta})}.
\]
\end{definition}

The corresponding probability measure $\Pi(\cdot \mid x)$ is called the posterior distribution:
\[
\Pi(B \mid x) = \int_B \pi(\theta \mid x) \, d\nu(\theta), \quad B \in \mathcal{F}_\Theta.
\]
\[
= \frac{\int_B p_\theta(x) \, \pi(\theta) \, d\nu(\theta)}{\int_\Theta p_{\tilde{\theta}}(x) \, d\Pi(\tilde{\theta})},
\]
\[
= \frac{\int_B p_\theta(x) \, d\Pi(\theta)}{\int_\Theta p_{\tilde{\theta}}(x) \, d\Pi(\tilde{\theta})},
\]

\begin{remark}
Think of $\Theta \subseteq \mathbb{R}^p$, $\nu(\cdot)$ as a Lebesgue measure, $\pi(\cdot)$ as a Lebesgue density.

\textbf{Exception:} $\Theta = \{0,1\}$ in hypothesis testing. Then, we'd take $\nu(\cdot)$ to be the counting measure.
\end{remark}

From the posterior, we can derive several estimators:
\begin{itemize}
    \item \textbf{Maximum-a-posterior (MAP) estimator:}
    \[
    \hat{\theta}_{\text{MAP}}(x) = \underset{\theta \in \Theta}{\text{argmax}} \, \pi(\theta \mid x).
    \]
    \item \textbf{Posterior mean:} Say \( \Theta \subseteq \mathbb{R}^p \) convex
    \[
    \hat{\theta}(x) = \int_\Theta \theta \, \pi(\theta \mid x) \, d\nu(\theta) \in \mathbb{R}^p.
    \]
\end{itemize}

\textbf{Back to Example 1.1:}  
\textit{Binomial model:} $\mathcal{X} = \{0, 1, \dots, n\}$,  
$p_\theta = \text{Bin}(n, \theta)$,  
$\theta \in \Theta = [0,1]$.  

\textit{Prior (uniform):} $\Pi = \text{Unif}(0,1)$.

We know: 
\[
\hat{\theta}_{\text{MAP}} = \hat{\theta}_{\text{MLE}} \quad \text{(for the uniform prior)},
\]
\[
\hat{\theta}_{\text{MAP}} = \frac{X}{n}.
\]

\begin{itemize}
    \item \textbf{Posterior mean}:
    \[
    \pi(\theta | x) = \frac{p_\theta(x)}
    {\int p_{\tilde{\theta}}(x) d\tilde{\theta}} \propto  \binom{n}{k} \theta^{x}(1-\theta)^{n-x}.
    \]
    
    \item \textbf{Binomial distribution}:
    \[
    \text{Bin}(n, p)(k) = \binom{n}{k} p^k (1-p)^{n-k},
    \]
    where \( k \in \{0, \ldots, n\} \) is the number of successes, and \( p \) is the probability of success, and $n$ would be interpreted as the "number of coin flips".
\end{itemize}

\[
\pi(\theta | x) \propto \theta^x (1-\theta)^{n-x}.
\]
and 

\[ \int_0^1 \pi(\theta \mid x) d\nu  =1\]

We conclude that \( \pi(\theta | x) \) is a \textbf{Beta-distribution} on \( [0,1] \),
\[
\text{Beta}(x+1, n-x+1).
\]

The mean is given by:
\[
\hat{\theta} = \frac{x+1}{n+2}.
\]

\begin{remark}[Beta distribution]
The Beta distribution is defined as:
\[
\text{Beta}(a, b), \quad a,b \geq 0.
\]
The probability density function of the Beta distribution is given by:
\[
P_{\text{Beta}(a,b)}(x) = x^{a}(1-x)^{b}.
\]
\end{remark}

\begin{definition}[Conjugate Bayesian models]
Let \((P_\theta : \theta \in \Theta)\) be a statistical model. Then, some family \(\mathcal{D}\) of p.m.s on \(\Theta\) is called \textit{conjugate} if
\[
\Pi \in \mathcal{D} \implies \Pi(\cdot | x) \in \mathcal{D} \quad \text{for all } x \in \mathcal{X}.
\]
\end{definition}

\noindent \textbf{Examples:}
\begin{itemize}
    \item \((\text{Bin}(n, \theta)) : \theta \in [0,1], \quad \mathcal{D} = \text{Beta}(a,b), \ a,b \geq 0.\)
    \item \((\mathcal{N}(\mu, \sigma^2)) : \mu \in \mathbb{R}, \quad \mathcal{D} = \left\{\mathcal{N}\left(\mu, n^2\right), \mu \in \mathbb{R}, n^2 > 0 \right\}, \ \sigma^2 \)\text{ known}.
\end{itemize}


\chapter{Decision Theory}

Here suppose that \(\Theta \subseteq \mathbb{R}^p\).
\begin{definition}[Loss function]
A function \(\ell : \Theta \times \mathbb{R}^p \to [0, \infty)\) is a \textit{loss function} if for every \(\theta \in \Theta\), \(\ell(\theta, \cdot)\) is measurable. Given some estimator \(\hat{\theta}\), the expected loss is
\[
\mathcal{R}(\theta, \hat{\theta}) = \mathbb{E}_\theta \left[\ell(\theta, \hat{\theta})\right].
\]
\end{definition}

\noindent \textbf{Example:} Take \(\ell(\theta, \hat{\theta}) = \| \theta - \hat{\theta} \|^2_{\mathbb{R}^p}.\) Then,
\[
\mathcal{R}(\theta, \hat{\theta}) = \mathbb{E}_\theta \left[\|\theta - \hat{\theta}\|^2_{\mathbb{R}^p} \right]
\]
is the mean squared error (MSE).

\begin{proposition}[Bias-Variance Decomposition]
    Let $\hat{\theta} \in L^2(\mathbb{P}_\theta)$. Then it holds that:
    \[
    R(\hat{\theta}, \theta) = (\mathbb{E}_\theta[\hat{\theta}] - \theta)^2 + \operatorname{Var}_\theta(\hat{\theta}).
    \]
\end{proposition}

\begin{proof}
    We have
    \[
    R(\hat{\theta}, \theta) = \mathbb{E}_\theta[(\hat{\theta} - \mathbb{E}_\theta[\hat{\theta}] + \mathbb{E}_\theta[\hat{\theta}] - \theta)^2]
    \]
    Expanding the squared term:
    \[
    = \mathbb{E}_\theta[(\hat{\theta} - \mathbb{E}_\theta[\hat{\theta}])^2] + (\mathbb{E}_\theta[\hat{\theta}] - \theta)^2 + 2\mathbb{E}_\theta[(\hat{\theta} - \mathbb{E}_\theta[\hat{\theta}])(\mathbb{E}_\theta[\hat{\theta}] - \theta)].
    \]
    Since $\mathbb{E}_\theta[\hat{\theta} - \mathbb{E}_\theta[\hat{\theta}]] = 0$, the last term vanishes, leaving us with:
    \[
    R(\hat{\theta}, \theta) = \operatorname{Var}_\theta(\hat{\theta}) + (\mathbb{E}_\theta[\hat{\theta}] - \theta)^2.
    \]
\end{proof}

\begin{definition}[Minimax Risk]
    Given an estimator $\hat{\theta}$ in a model $(\mathbb{P}_\theta : \theta \in \Theta)$, the maximal risk of $\hat{\theta}$ is
    \[
    \sup_{\theta \in \Theta} R(\theta, \hat{\theta}).
    \]
    The minimax risk of a model $(\mathbb{P}_\theta : \theta \in \Theta)$ is given as
    \[
    \inf_{\hat{\theta}} \sup_{\theta \in \Theta} R(\theta, \hat{\theta}),
    \]
    where the infimum is taken over all estimators. An estimator is called minimax if
    \[
    \sup_{\theta \in \Theta} R(\theta, \hat{\theta}) = \inf_{\hat{\theta}} \sup_{\theta \in \Theta} R(\theta, \hat{\theta}).
    \]
\end{definition}

\begin{definition}[Bayes Risk]
    Given a prior $\pi$ on $\Theta$, the $\pi$-Bayes risk of a decision rule $\delta$ for the loss function $L$ is defined as
    \[
    R_\Pi(\delta) = \mathbb{E}_\Pi [R(\delta, \theta)] = \int_{\Theta} R(\delta, \theta) \pi(\theta) d\theta = \int_{\Theta} \int_{\mathcal{X}} L(\delta(x), \theta) \pi(\theta) p_\theta(x) dx d\theta.
    \]
    A $\Pi$-Bayes decision rule $\hat\theta_\Pi$ is any decision rule that minimizes $R_\Pi(\hat\theta)$.
\end{definition}

\textcolor{red}{SW: $\ell$ instead of $L$ below, $p_{\theta}(x)$ instaed of  $f(x, \theta)$ }
\textcolor{green}{Note: Has been corrected. }

\begin{definition}[Posterior Risk]
    For a Bayesian model, the posterior risk $R_{\Pi}$ is defined as the average loss under the posterior distribution for some observation $x \in \mathcal{X}$:
    \[
    R_{\Pi(\cdot|x)}(\delta) = \mathbb{E}_\Pi[\ell(\delta(x), \theta) | x].
    \]
    Here, the notation $\mathbb E_\Pi[\cdot|x]$ stands for the expectation under the posterior distribution.
\end{definition}

\begin{proposition}[Bayes Risk and Posterior Risk]
    An estimator $\delta$ that minimizes the $\Pi$-posterior risk $R_{\Pi}$ also minimizes the $\pi$-Bayes risk $R_\pi$.
\end{proposition}

\begin{proof}
    The $\pi$-Bayes risk can be rewritten as
\begin{align*}
    R_\pi(\delta) &= \int_{\Theta} \mathbb{E}_\theta[\ell(\delta(X), \theta)] \pi(\theta) d\theta\\
    &= \int_{\Theta} \int_{\mathcal{X}} \ell(\delta(x), \theta) \pi(\theta) p_{\theta}(x) dx d\theta\\
    &= \int_{\mathcal{X}} \int_{\Theta} \ell(\delta(x), \theta) \frac{p_{\theta}(x) \pi(\theta)}{\int_{\Theta} p_{\theta'}(x) \pi(\theta') d\theta'} \times \underbrace{\int_{\Theta} p_{\theta'}(x) \pi(\theta') d\theta'}_{=: n(x) \geq 0} dx d\theta\\
    &= \int_{\mathcal{X}} \mathbb{E}_{\Pi}[\ell(\delta(x), \theta) | x] n(x) dx.
\end{align*}
    [Notation $n(x)$ motivated by the word `normalising constant'].

    
    Let $\delta_\Pi$ be a decision rule that minimizes the posterior risk, i.e., such that for all $x \in \mathcal{X}$,
    \[
    \mathbb{E}_{\Pi}[\ell(\delta_\Pi(x), \theta) | x] \leq \mathbb{E}_{\Pi}[\ell(\delta(x), \theta) | x].
    \]
    Multiplying by $m(x) \geq 0$ and integrating on both sides over $\mathcal{X}$ yields the desired result.
\end{proof}

\begin{example}
    For the quadratic risk with the squared-loss, the posterior risk is minimized by taking $\delta(X) = \mathbb{E}_{\Pi}[\theta | X]$, by minimizing the quadratic function in $\delta$. Other losses will give other ways to minimize the posterior risk, and other Bayes decision rules.
\end{example}

\begin{proposition}
Let \((\mathcal{X}, \mathcal{F}, (\mathbb{P}_{\theta})_{\theta \in \Theta})\) be a statistical model and let \(\hat{\theta}\) be an estimator. Then we have
\[
\sup_{\theta \in \Theta} R(\hat{\theta}, \theta) = \sup_{\Pi} \int_{\Theta} R(\hat{\theta}, \theta) \, \Pi(d\theta),
\]
where the supremum is taken over all prior distributions \(\Pi\).
\end{proposition}

\begin{proof}
Obviously, we have 
\[
\int_{\Theta} R(\hat{\theta}, \theta) \, \Pi(d\theta) \leq \sup_{\theta \in \Theta} R(\hat{\theta}, \theta).
\]
On the other hand, by using the prior distributions \(\delta_{\theta}\) (Dirac measure on \(\theta \in \Theta\)), we obtain
\[
\sup_{\Pi} \int_{\Theta} R(\hat{\theta}, \theta) \, \Pi(d\theta) \geq \int_{\Theta} R(\hat{\theta}, \theta) \, \delta_{\theta}(d\theta) = R(\hat{\theta}, \theta).
\]
\end{proof}

[Note: In the following we use the notation $\delta$ for decision rules while on the blackboard we used $\hat{\theta}$ or $\tilde{\theta}$. If you want to adjust this please contact me so that I can give you access.]

\begin{proposition}
Let \(\pi\) be a prior on \(\Theta\) such that
\[
R_{\pi}(\delta_{\pi}) = \sup_{\theta \in \Theta} R(\delta_{\pi}, \theta),
\]
where \(\delta_{\pi}\) is a \(\pi\)-Bayes rule. Then it holds that
\begin{enumerate}
    \item The rule \(\delta_{\pi}\) is minimax.
    \item If \(\delta_{\pi}\) is unique Bayes, then it is unique minimax.
\end{enumerate}
\end{proposition}

\begin{proof} Let \(\delta\) be any decision rule. Then
\[
\sup_{\theta \in \Theta} R(\delta, \theta) \geq \mathbb{E}_{\pi}[R(\delta, \theta)],
\]
\[
\int_{\Theta} R(\delta, \theta) \pi(\theta) \, d\theta \geq \mathbb{E}_{\pi}[R(\delta, \theta)],
\]
\[
\int_{\Theta} R(\delta, \theta) \pi(\theta) \, d\theta = R_{\pi}(\delta_{\pi}) = \sup_{\theta \in \Theta} R(\delta_{\pi}, \theta).
\]
Taking the infimum over \(\delta\) gives the result.

2. If \(\delta_{\pi}\) is unique Bayes, the second inequality is strict for any \(\delta' \neq \delta_{\pi}\).
\end{proof}

\begin{corollary}
If a (unique) Bayes rule \(\delta_{\pi}\) has constant risk in \(\theta\), then it is (unique) minimax.
\end{corollary}

\begin{proof}
If a Bayes rule \(\delta_{\pi}\) has constant risk, then
\[
R_{\pi}(\delta_{\pi}) = \mathbb{E}_{\pi}[R(\delta_{\pi}, \theta)] = \sup_{\theta \in \Theta} R(\delta_{\pi}, \theta),
\]
where \(R(\delta_{\pi}, \theta)\) is constant in \(\theta\). Uniqueness of the Bayes rule implies uniqueness of the minimax rule, as in part 2 of the former proposition.
\end{proof}

\begin{example}
Hence, if the maximal risk of a Bayes rule \(\delta_{\pi}\) equals the Bayes risk, then \(\pi\) is least favorable, and the corresponding Bayes rule is minimax.
\begin{itemize}
    \item In a \(\text{Bin}(n, \theta)\) model, let \(\pi_{a,b}\) be a \(\text{Beta}(a, b)\) prior on \(\theta \in [0,1]\). Then the unique Bayes rule for \(\pi_{a,b}\) over the quadratic risk is the posterior mean \(\delta_{a,b} = \bar{\theta}_{a,b}\). Trying to solve the equation
    \[
    R(\delta_{a,b}, \theta) = \text{const.} \quad \forall \theta \in [0,1]
    \]
    we can find a prior \(\pi_{a^*, b^*}\) and a corresponding Bayes rule \(\delta_{\pi_{a^*, b^*}}\) of constant risk. It is therefore unique minimax, and different from the MLE (see Examples sheet).
    
    \item In a \(\mathcal{N}(\theta, 1)\) model, \(\bar{X}_n\) is minimax, as proved later.
\end{itemize}
\end{example}

\section{Another optimality concept: Admissibility}

\begin{definition}
A decision rule \(\delta\) is \emph{inadmissible} if there exists \(\delta'\) such that
\[
R(\delta', \theta) \leq R(\delta, \theta) \quad \forall \theta \in \Theta \quad \text{and} \quad R(\delta', \theta) < R(\delta, \theta) \quad \text{for some } \theta \in \Theta.
\]
\end{definition}

\begin{remark}
\begin{itemize}
    \item The intuition is that there is no reason to choose an inadmissible estimator or decision rule: it would always be better to choose another estimator that dominates it.
    \item Admissibility is not the only criterion to evaluate an estimator: In most cases, a constant estimator will be admissible for the quadratic risk, but it is often not reasonable.
\end{itemize}
\end{remark}

\begin{proposition}
\begin{enumerate}
    \item A unique Bayes rule is admissible.
    \item If \(\delta\) is admissible and has constant risk, then it is minimax.
\end{enumerate}
\end{proposition}

Proof may be done in the Examples sheet.

\begin{definition}
For a vector \(X \in \mathbb{R}^p\), the \emph{James–Stein estimator} is defined as
\[
\delta^{JS}(X) = \left(1 - \frac{p - 2}{\|X\|^2}\right) X.
\]
\end{definition}

In a Gaussian model \(X \sim \mathcal{N}(\theta, I_p)\) for \(\theta \in \mathbb{R}^p\) (with a single observation, to simplify notation), the risk of the MLE is given by
\[
R(\hat{\theta}_{\text{MLE}}, \theta) = \mathbb{E}_{\theta}[\|X - \theta\|^2] = \sum_{j=1}^p \mathbb{E}_{\theta}[(X_j - \theta_j)^2] = p.
\]

For \( X \sim \mathcal{N}(\theta, I_p) \) with \( p \geq 3 \), the risk of the James–Stein estimator satisfies for all \( \theta \in \mathbb{R}^p \)
\[
R(\delta^{JS}, \theta) < p.
\]

\chapter{Confidence Sets}

\begin{definition}[Confidence Set]
Let \((\mathcal{X}, \mathcal{F}, (\mathbb{P}_{\theta})_{\theta \in \Theta})\) be a statistical model. For a given \(\alpha \in [0,1]\), consider sets \( C_{1-\alpha}(x) \subseteq \Theta \) for each \( x \in \mathcal{X} \). Then \( C_{1-\alpha}(x) \) is called a random confidence set at level \( 1 - \alpha \) (or with coverage probability \( 1 - \alpha \)) if
\[
\forall \theta \in \Theta : \mathbb{P}_{\theta}(\theta \in C_{1-\alpha}) = \mathbb{P}_{\theta}( \{ x \in \mathcal{X} : \theta \in C_{1-\alpha}(x) \}) \geq 1 - \alpha.
\]
\end{definition}

\textbf{Note:} The following example was only started in Lecture 4 and may be fully covered in Lecture 5.

\begin{example}
Consider the statistical model \((\{0,1\}^n, \mathcal{P}(\{0,1\}^n), (\mathbb{P}_p)_{p \in [0,1]})\) with \(\mathbb{P}_p = \text{Ber}(p)^{\otimes n}\) and independent observations \(X_k \sim \text{Ber}(p)\), \( k \in \{0, \ldots, n\}\). We are looking for a confidence interval \( C_{1-\alpha} \) around \( \hat{p} = \overline{X}_n = \frac{1}{n} \sum_{k=1}^n X_k \), i.e.,
\[
C_{1-\alpha} = [\overline{X}_n - a, \overline{X}_n + b] \quad \left(C_{1-\alpha}(x) = [\overline{X}_n(x) - a(x), \overline{X}_n(x) + b(x)] \right)
\]
should satisfy (where \(a\) and \(b\) might be random) that
\[
1 - \alpha \leq \mathbb{P}_p \left( p \in C_{1-\alpha} \right) = \mathbb{P}_p \left( \overline{X}_n - a \leq p \leq \overline{X}_n + b \right) = \mathbb{P}_p \left(-b \leq \overline{X}_n - p \leq a \right).
\]

Let \(t \mapsto F_p^n(t) := \mathbb{P}_p \left( \overline{X}_n - p \leq t \right)\) be the distribution function. Then,
\[
\mathbb{P}_p \left(-b \leq \overline{X}_n - p \leq a \right) = \mathbb{P}_p \left( \overline{X}_n - p \leq a \right) - \mathbb{P}_p \left( \overline{X}_n - p < -b \right)
\]
\[
= F_p^n(a) - F_p^n(-b) + R_n,
\]
where \( R_n = \mathbb{P}_p \left( \overline{X}_n - p = -b \right) \). Choose \(a, b\) as quantiles of \(\mathbb{P}_p^n\), i.e., \(a = \left(F_p^n\right)^{-1}(1 - \alpha / 2)\) and \( -b = \left(F_p^n\right)^{-1}(\alpha / 2)\) (with quantile function \(t \mapsto \left(F_p^n\right)^{-1}(t) := \inf \{t \in \mathbb{R} : F_p^n(t) \geq t \}\)).

However, \(F_p^n\) and thus \(a, b\) are unknown. Consider two possibilities:

\paragraph{Normal Approximation.} It holds that \( \mathbb{E}_p^n[X_k] = p \), \( \sigma := \text{Var}_p^n(X_k) = p(1 - p) \). By the central limit theorem, we have
\[
\frac{\sqrt{n}}{\sigma} \left( \overline{X}_n - p \right) = \frac{1}{\sqrt{n}} \sum_{k=1}^n \frac{X_k - p}{\sigma} \xrightarrow{d} N(0,1), \quad n \to \infty.
\]

For \( Z \sim N(0,1) \), it holds that
\[
F_p^n(a) = \mathbb{P}_p^n \left( \overline{X}_n - p \leq a \right) = \mathbb{P}_p^n \left( \frac{\sqrt{n}}{\sigma} \left( \overline{X}_n - p \right) \leq \frac{\sqrt{n}}{\sigma} a \right) \approx \mathbb{P}(|Z| \leq \frac{\sqrt{n}}{\sigma} a)
\]
\[
= \Phi \left( \frac{\sqrt{n}}{\sigma} a \right) = \Phi(z_\beta)
\]
for \( a := \frac{\sigma}{\sqrt{n}} z_\beta \) (where \( z_\beta \) is the \( \beta \)-quantile of the \( N(0,1) \)-distribution, i.e., \( \Phi(z_\beta) = \beta \)). In particular, \( R_n = o(1) \) (i.e., \( R_n \to 0 \) as \( n \to \infty \)). For \( a = b \) (since the \( N(0,1) \)-distribution is symmetric) and because \( \Phi(-x) = 1 - \Phi(x) \), it follows that
\[
\mathbb{P}_p^n(p \in C_{1-\alpha}) = F_p^n(a) - F_p^n(-a) + R_n \approx \Phi(z_\beta) - (1 - \Phi(z_\beta)) + o(1) = 2 \Phi(z_\beta) - 1 + o(1).
\]

For \( \beta = 1 - \alpha/2 \), \( C_{1-\alpha} \) is an asymptotically correct confidence interval. However, \( p \) and therefore \( \sigma \) and \( a \) are unknown. Solutions:

\begin{itemize}
    \item Estimate \( \sigma = p(1 - p) \leq 1/4 \) to widen the confidence interval.
    
    \item For the empirical variance \( \hat{\sigma}^2 = \frac{1}{n} \sum_{k=1}^n (X_k - \overline{X}_n)^2 \), we have \( \hat{\sigma}^2 \to \sigma^2 \) almost surely (by the law of large numbers). Using Slutsky’s lemma (Lemma: For random variables \( (X_n, Y_n)_{n \geq 1} \) with \( X_n \xrightarrow{d} X \), \( Y_n \xrightarrow{p} c \in \mathbb{R} \) (where \( c \) is deterministic), it holds that \( X_n + Y_n \xrightarrow{d} X + c \) and \( X_n \cdot Y_n \xrightarrow{d} c \cdot X \)), it follows that
    \[
    \frac{\sqrt{n}}{\hat{\sigma}} \left( \overline{X}_n - p \right) = \frac{\sigma}{\hat{\sigma}} \cdot \frac{\sqrt{n}}{\sigma} \left( \overline{X}_n - p \right) \xrightarrow{d} N(0,1), \quad n \to \infty.
    \]
\end{itemize}

From this, we derive \( a = \frac{\hat{\sigma}}{\sqrt{n}} z_{1 - \alpha / 2} \) (randomly chosen).

\[
\mathbb{P}_p^n(p \in C_{1 - \alpha}) = \mathbb{P}_p^n \left( |\overline{X}_n - p| \leq a \right) = \mathbb{P}_p^n \left( \left| \frac{\sqrt{n}}{\hat{\sigma}} (\overline{X}_n - p) \right| \leq z_{1 - \alpha / 2} \right)
\]
\[
\approx \mathbb{P}(|Z| \leq z_{1 - \alpha / 2}) = 2 \Phi(z_{1 - \alpha / 2}) - 1 = 1 - \alpha.
\]
\end{example}

\section{Hypothesis Testing}

\subsection{Basic Definitions}
Let \( (P_\theta : \theta \in \Theta) \) be a statistical model, and let \(\Theta = \Theta_0 \cup \Theta_1\) be a partition:
\begin{itemize}
    \item A \textbf{statistical test} is a measurable function of the data \( \varphi : (\mathcal{X}, \mathcal{F}) \to [0,1] \).
    \item If \( \varphi(x) \in \{0,1\} \) for all \( x \in \mathcal{X} \), then \( \varphi \) is a \textbf{non-randomized test}; otherwise, it is \textbf{randomized}.
    \item \( H_0 : \theta \in \Theta_0 \) is the \textbf{null hypothesis}.
    \item \( H_1 : \theta \in \Theta_1 \) is the \textbf{alternative hypothesis}.
    \item The map \( \theta \to \beta_\varphi (\theta) = P_\theta (\varphi = 1) \) is called the \textbf{power function} of a test \( \varphi \).
\end{itemize}

\subsection{Type I and Type II Errors}
\begin{itemize}
    \item For \(\theta \in \Theta_0\), \( \beta_\varphi(\theta) \) represents the \textbf{Type I error} (wrongly rejecting the null).
    \item For \(\theta \in \Theta_1\), \(1 - \beta_\varphi(\theta)\) represents the \textbf{Type II error} (failing to reject the alternative when it is true).
\end{itemize}

\[
1 \quad \quad \beta_\varphi(\theta) \quad 0 \quad \quad \Theta_0 \quad \quad \Theta_1 \quad \quad \Theta
\]

\textbf{Note:}
\[
1 - P_\theta (\varphi = 1) = P_\theta (\varphi = 0) = P_\theta \text{ (wrongly accepting the null)}
\]

\subsection{Level and Uniformly Most Powerful Tests}
\begin{definition}[Level]
A test \( \varphi : \mathcal{X} \to [0,1] \) has \textbf{level} \( \alpha \in [0,1] \) if
\[
\sup_{\theta \in \Theta_0} \beta_\varphi (\theta) \leq \alpha.
\]
\end{definition}

\begin{definition}[Uniformly Most Powerful Test]
Given a level \( \alpha \in (0,1) \), \( \varphi : \mathcal{X} \to [0,1] \) is called \textbf{uniformly most powerful (UMP)} if, for every other test \( \varphi' \) of level \( \alpha \) and all \( \theta \in \Theta_1 \),
\[
\beta_\varphi (\theta) \geq \beta_{\varphi'}(\theta).
\]
\end{definition}

\subsection{The Neyman-Pearson Lemma}
The Neyman-Pearson Lemma provides a basis for constructing the most powerful tests for simple hypotheses:
\begin{theorem}[Neyman-Pearson Lemma]
Let \( \Theta_0 = \{\theta_0\}\) and \( \Theta_1 = \{\theta_1\}\) be simple hypotheses:
\begin{enumerate}
    \item \textbf{Existence:} There exists a test \( \varphi \) and a constant \( k \in [0, \infty) \) such that \( P_{\theta_0} (\varphi = 1) = \alpha \), with
    \[
    \varphi(x) = 
    \begin{cases}
      1, & \text{if } \frac{p_{\theta_1}(x)}{p_{\theta_0}(x)} > k \\
      0, & \text{if } \frac{p_{\theta_1}(x)}{p_{\theta_0}(x)} < k
    \end{cases}
    \]
    Here, \( p_{\theta_1} \) and \( p_{\theta_0} \) are densities with respect to some dominated measure \( \mu \).
    
    \item \textbf{Sufficiency:} If \( \varphi \) satisfies \( P_{\theta_0} (\varphi = 1) = \alpha \) and the above form, then \( \varphi \) is a UMP level \( \alpha \) test.
    
    \item \textbf{Necessity:} If \( \varphi_k \) is UMP for level \( \alpha \), then it must be of the form shown above.
\end{enumerate}
\end{theorem}

\subsection{Proof of the Neyman-Pearson Lemma}
\begin{enumerate}
    \item Define the likelihood ratio \( r(x) = \frac{p_{\theta_1}(x)}{p_{\theta_0}(x)} \in [0, \infty) \). Let \( F_0 \) be the CDF of \( r(x) \) under \( P_{\theta_0} \).
    \[
    F_0(t) = P_{\theta_0} (r(x) \leq t).
    \]
    Define \( \alpha(t) = 1 - F_0(t) = P_{\theta_0} (r(x) > t) \) and note:
    \begin{itemize}
        \item \( \alpha \) is right-continuous:
        \[
        \lim_{\epsilon \to 0} \alpha(t + \epsilon) = P_{\theta_0} (r(x) > t).
        \]
        \item \( \alpha \) is non-increasing.
        \item \( \alpha \) has left limits.
    \end{itemize}

    \textbf{\( \alpha \) is cadlag:} It is continuous from the right and has a left limit.

    \[
    \text{There exists } k \in [0, \infty) \text{ such that }
    \alpha \leq \alpha(k^-) \quad \text{and} \quad \alpha \geq \alpha(k).
    \]

    We define the test
    \[
    \varphi(x) = 
    \begin{cases} 
      1 & \text{if } r(x) > k, \\
      \gamma & \text{if } r(x) = k, \\
      0 & \text{if } r(x) < k.
    \end{cases}
    \]
    
    Set \(\gamma = \frac{\alpha - \alpha(k)}{\alpha(k^-) - \alpha(k)}\).

    The level of \( \varphi \) is
    \[
    E_{\theta_0} [\varphi(x)] = P_{\theta_0} (\varphi(x) = 1).
    \]
    \[
    = P_{\theta_0} (r(x) > k) + P_{\theta_0} (r(x) = k) \cdot \gamma = \alpha.
    \]
\end{enumerate}
\chapter{Lecture 6: Neyman-Pearson Lemma and Likelihood Ratio Tests}
\section*{Neyman-Pearson Lemma}

\subsection*{Power of a Test}
The \textbf{power} of a test is defined as:
\[
E_{\theta_1}[\varphi] = P_{\theta_1}(\varphi = 1)
\]

\subsection*{Likelihood Ratio Test}
The \textbf{likelihood ratio} is given by:
\[
\Lambda(x) = \frac{p_{\theta_1}(x)}{p_{\theta_0}(x)} = r(x)
\]

\subsection*{Likelihood Ratio (LR) Test}
The LR test is defined as:
\[
\varphi(x) = 
\begin{cases} 
1 & \text{if } r(x) > k, \\
\gamma & \text{if } r(x) = k, \\
0 & \text{if } r(x) < k,
\end{cases}
\]
where \( k \in [0, \infty) \) and \( \gamma \in [0, 1] \).

\textbf{Note:} LR tests are Uniformly Most Powerful (UMP) for simple hypothesis testing:
\begin{itemize}
    \item Given a significance level \( \alpha \), if the LR test satisfies \( E_{\theta_0}[\varphi] = \alpha \), it controls the Type I error.
    \item The LR test minimizes the Type II error:
    \[
    E_{\theta_1}[\varphi] \geq E_{\theta_1}[\varphi'] \quad \forall \varphi'
    \]
\end{itemize}

\subsection*{Continuation of Proof (Part of UMP)}
Let \( \varphi' \) be another level \( \alpha \) test such that \( E_{\theta_0}[\varphi'] \leq \alpha \).

\textbf{Goal:} Show that \( E_{\theta_1}[\varphi] \geq E_{\theta_1}[\varphi'] \).

Let \( \mu \) be the dominating measure. Consider:
\[
\int (\varphi(x) - \varphi'(x)) \left( p_{\theta_1}(x) - k p_{\theta_0}(x) \right) d\mu(x) = 0
\]
\textbf{Claim:} \( p \geq 0 \).

\textbf{Observation:}
\begin{itemize}
    \item If \( p_{\theta_1}(x) - k p_{\theta_0}(x) > 0 \), then \( \frac{p_{\theta_1}(x)}{p_{\theta_0}(x)} > k \), implying \( \varphi(x) = 1 \).
    \item If \( p_{\theta_1}(x) - k p_{\theta_0}(x) < 0 \), then \( \varphi(x) = 0 \).
    \item If \( p_{\theta_1}(x) - k p_{\theta_0}(x) = 0 \), then the integrand is \( 0 \).
\end{itemize}
Thus, \( p = 0 \), leading to:
\[
\int (\varphi - \varphi') p_{\theta_1} \, d\mu = \int (\varphi - \varphi') p_{\theta_0} \, d\mu = k \left[ E_{\theta_0}[\varphi] - E_{\theta_0}[\varphi'] \right] \geq 0
\]
Therefore,
\[
E_{\theta_1}[\varphi] \geq E_{\theta_1}[\varphi']
\]

\subsection*{Part (3) UMP \(\Rightarrow\) LR}
Assume \( \varphi^* \) is a UMP test with \( E_{\theta_0}[\varphi^*] = \alpha \). Let \( \varphi \) be the LR test satisfying \( E_{\theta_0}[\varphi] = \alpha \).

\textbf{Goal:} Show that \( \varphi = \varphi^* \) almost everywhere except on \( \{ r(x) = k \} \).

Define the sets:
\[
x^+ = \{ x : \varphi(x) > \varphi^*(x) \}, \quad 
x^- = \{ x : \varphi(x) < \varphi^*(x) \}, \quad
x^0 = \{ x : \varphi(x) = \varphi^*(x) \}
\]
\[
\tilde{x} = (x^+ \cup x^-) \cap \{ x : p_{\theta_1}(x) \neq k p_{\theta_0}(x) \}
\]
It suffices to show \( \mu(\tilde{x}) = 0 \).

On \( \tilde{x} \):
\[
(\varphi - \varphi^*) (p_{\theta_1} - k p_{\theta_0}) > 0
\]
If \( \mu(\tilde{x}) > 0 \), then:
\[
\int_{\mathcal{X}} (\varphi - \varphi^*)(p_{\theta_1} - k p_{\theta_0}) \, d\mu \geq 0
\]
\[
\int_{\tilde{x}} (\varphi - \varphi^*)(p_{\theta_1} - k p_{\theta_0}) \, d\mu \geq 0
\]
However,
\[
E_{\theta_1}[\varphi] - E_{\theta_1}[\varphi^*] > k \left[ E_{\theta_0}[\varphi] - E_{\theta_0}[\varphi^*] \right] \geq 0
\]
This leads to a contradiction, implying \( \mu(\tilde{x}) = 0 \) and thus \( \varphi = \varphi^* \) almost everywhere.

\subsection*{Example: Gaussian Location Model}
Consider:
\[
X_1, \dots, X_n \overset{\text{iid}}{\sim} \mathcal{N}(\mu, \sigma^2)
\]
Testing:
\[
H_0 : \mu = \mu_0 \quad \text{vs.} \quad H_1 : \mu = \mu_1, \quad \mu_0 < \mu_1
\]
The likelihood ratio is:
\[
\frac{p_1(X_1, \dots, X_n)}{p_0(X_1, \dots, X_n)} = \exp \left( -\frac{1}{2\sigma^2} \sum_{i=1}^n (X_i - \mu_1)^2 + \frac{1}{2\sigma^2} \sum_{i=1}^n (X_i - \mu_0)^2 \right)
\]
Simplifying:
\[
= \exp \left( -\frac{n}{2\sigma^2} (\mu_1^2 - \mu_0^2) - \frac{2(\mu_1 - \mu_0)}{\sigma^2} \sum_{i=1}^n X_i \right) \geq K_\alpha
\]
This implies:
\[
\frac{1}{n} \sum_{i=1}^n X_i \geq K_\alpha, \quad \text{for some } K_\alpha \in \mathbb{R}
\]
To determine \( K_\alpha \):
\[
\bar{X}_n := \frac{1}{n} \sum X_i \overset{H_0}{\sim} \mathcal{N}\left(\mu_0, \frac{\sigma^2}{n}\right)
\]
Thus:
\[
P_{H_0} \left( \bar{X}_n \geq K_\alpha \right) = 1 - \Phi \left( \frac{\sqrt{n}}{\sigma} (K_\alpha - \mu_0) \right)
\]
Solving for \( K_\alpha \):
\[
K_\alpha = \mu_0 + \frac{\sigma}{\sqrt{n}} \Phi^{-1}(1 - \alpha)
\]
Therefore, the LR test is:
\[
\varphi(X_1, \dots, X_n) = 
\begin{cases} 
1 & \text{if } \bar{X}_n \geq \mu_0 + \frac{\sigma}{\sqrt{n}} \Phi^{-1}(1 - \alpha), \\
0 & \text{otherwise}.
\end{cases}
\]

\section*{Corollary}

Consider simple hypothesis testing. Let \( \varphi \) be a UMP test at level \( \alpha \). Then:
\[
\alpha = E_{H_0}[\varphi] \leq E_{\theta_1}[\varphi]
\]
Suppose \( E_{\theta_1}[\varphi] = E_{\theta_1}[\varphi_0] \). Then \( \varphi_0 \) is also UMP, implying \( \varphi_0 \) is an LR test:
\[
\varphi_0 = 
\begin{cases} 
1 & \text{if } \frac{p_{\theta_1}}{p_{\theta_0}} \geq K \quad \text{a.s., for some } K, \\
0 & \text{otherwise}.
\end{cases}
\]
Since \( \varphi_0 \in \{\varphi, \beta\} \), it follows that \( p_{\theta_1} = K p_{\theta_0} \) almost surely.

Moreover:
\[
\int p_{\theta_0} \, d\mu = K \int p_{\theta_0} \, d\mu = 1 \quad \Rightarrow \quad K = 1
\]

\subsection*{Correspondence Theorem}

\textbf{Statement:} There is a correspondence between tests and confidence regions.

\[
\text{Tests} \quad \longleftrightarrow \quad \text{Confidence regions } C(x)
\]
with
\[
\Pr_{\theta}(\theta \in C(x)) \geq 1 - \alpha
\]
and
\[
\Pr_{\theta}(\phi_{\theta}(x) = 1) = \alpha
\]

\textbf{Theorem:} Let \( \{P_\theta : \theta \in \Theta\} \) be a statistical model and \( \alpha \in (0, 1) \).

\begin{enumerate}
    \item[(i)] If \( C = C(X) \) is a level-\(\alpha\) confidence set, then
    \[
    \phi_{\theta_0}(x) = \mathbb{I}\left\{ \theta_0 \notin C(x) \right\}
    \]
    is a level-\(\alpha\) test for \( H_0: \theta = \theta_0 \) vs. \( H_1: \theta \neq \theta_0 \).
    
    \item[(ii)] If \( \{\phi_{\theta} : \theta \in \Theta \} \) is a family of level-\(\alpha\) tests, then
    \[
    C(X) = \left\{ \theta \in \Theta : \phi_{\theta}(X) = 0 \right\}
    \]
    is a \( 1 - \alpha \) confidence set.
\end{enumerate}

\textbf{Proof:}
\begin{enumerate}
    \item[(i)] 
    \[
    \Pr_{\theta_0}(\phi_{\theta_0} = 1) = \Pr_{\theta_0}(\theta_0 \notin C(X)) \leq \alpha
    \]
    
    \item[(ii)] 
    \[
    \Pr_{\theta}(\theta \notin C(X)) = \Pr_{\theta}(\phi_{\theta}(X) = 1) \leq \alpha
    \]
\end{enumerate}

\section*{UMPT Tests in Models with Monotone Likelihoods}

\textbf{Proposition:} Let \( \Theta \subseteq \mathbb{R} \). Consider testing:
\[
H_0 : \theta \leq \theta_0 \quad \text{vs.} \quad H_1 : \theta > \theta_0,
\]
for some \( \theta_0 \in \mathbb{R} \).

Assume there exists a test statistic \( T : X \to \mathbb{R} \) and a function \( h : \mathbb{R} \times \Theta \times \Theta \to \mathbb{R} \) such that:
\[
\frac{P_{\theta}(X)}{P_{\tilde{\theta}}(X)} = h(T(X), \theta, \tilde{\theta})
\]
and for all \( \theta \geq \tilde{\theta} \), the function \( t \mapsto h(t, \theta, \tilde{\theta}) \) is monotone increasing.

\textbf{Conclusion:} LR tests are also UMP for level \( \alpha \). Specifically, the LR test of \( H_0: \theta = \theta_0 \) against \( H_1: \theta = \theta_1 \) for any \( \theta_1 > \theta_0 \) will be UMP.























\iffalse


\appendix

\numberwithin{definition}{section}
\numberwithin{theorem}{section}
\numberwithin{proposition}{section}
\numberwithin{lemma}{section}
\numberwithin{corollary}{section}
\numberwithin{remark}{section}
\numberwithin{example}{section}
\numberwithin{supplement}{section}

%%%%%%%%%%%%%%%%%%%%%%%%%%%%%%%%%%%%%%%%%%%%%%%%%%%%%%%%%%%%%%%%%%%%%%%%%%%%%%%
%%%%%%%%%%%%%%%%%%%%%%%%%%%%%%%%%%%%%%%%%%%%%%%%%%%%%%%%%%%%%%%%%%%%%%%%%%%%%%%
%%%%%%%%%%%%%%%%%%%%%%%%%%%%%%%%%%%%%%%%%%%%%%%%%%%%%%%%%%%%%%%%%%%%%%%%%%%%%%%
\chapter{Appendix to the chapters} \label{ch:appendix}
%%%%%%%%%%%%%%%%%%%%%%%%%%%%%%%%%%%%%%%%%%%%%%%%%%%%%%%%%%%%%%%%%%%%%%%%%%%%%%%
%%%%%%%%%%%%%%%%%%%%%%%%%%%%%%%%%%%%%%%%%%%%%%%%%%%%%%%%%%%%%%%%%%%%%%%%%%%%%%%
%%%%%%%%%%%%%%%%%%%%%%%%%%%%%%%%%%%%%%%%%%%%%%%%%%%%%%%%%%%%%%%%%%%%%%%%%%%%%%%
%%%%%%%%%%%%%%%%%%%%%%%%%%%%%%%%%%%%%%%%%%%%%%%%%%%%%%%%%%%%%%%%%%%%%%%%%%%%%%%
\section{Introducing the problem}\label{sec: appendix to chap 1}
%%%%%%%%%%%%%%%%%%%%%%%%%%%%%%%%%%%%%%%%%%%%%%%%%%%%%%%%%%%%%%%%%%%%%%%%%%%%%%%

.....

\fi 

\medskip
\printbibliography[heading=bibintoc,title={References}] % you can change the title however you want
\end{document}