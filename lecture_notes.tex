% Für Hausaufgaben article, für Skripte report
% \documentclass[open=any, 11pt,paper=A4,headings=big,titlepage = firstiscover]{scrreprt}
\documentclass[open=any, 11pt,paper=A4]{scrreprt}


\KOMAoptions{}


  
%\usepackage[colorinlistoftodos]{todonotes}
\usepackage[
backend=biber,
style=alphabetic, % Label wird verschönert
sorting=ynt,
backrefstyle=two % sortiert nach year, name und dann title
]{biblatex}
\usepackage[textsize=tiny,linecolor=black,backgroundcolor=white]{todonotes}
\usepackage{0_einstellungen}
\addbibresource{0_bibliography.bib} %Imports bibliography file
\usepackage[font=footnotesize]{caption}
\usepackage{wrapfig}
\usepackage{lmodern}
\usepackage{xcolor}
\usepackage{bm}

\usepackage{pdfpages,datetime}


% \author{Said Kassner}
% \title{Lecture Notes}

% MAKROS
%%%%%%%%%%%%%%%%%

%MAKRO DATEI


%Aktuelle Makros
%%%%%%%%%%%%%%%%
%%%%%%%%%%%%%%%%
%%%%%%%%%%%%%%%%

\newcommand{\Gd}{G_\delta} % eine G delta menge
\newcommand{\Fs}{F_\sigma} % eine F sigma Menge
\newcommand{\I}{\mathcal{I}}
\newcommand{\mcN}{\mathcal{N}}
\newcommand{\question}[1]{\todo[noline,linecolor=black,backgroundcolor=white]{\textit{Question:} #1}}
\newcommand{\task}[1]{\todo[noline,linecolor=black,backgroundcolor=pink]{\textit{Task:} #1}}
\newcommand{\note}[1]{\todo[noline,linecolor=black,backgroundcolor=white]{\textit{Note:} #1}}
\newcommand{\corrected}[1]{\todo[linecolor=green,backgroundcolor=green]{\textit{Corrected up to this point.} #1}}
\newcommand{\bfn}{\textbf{n}}
\newcommand{\outermu}{\mu^*}
\newcommand{\innermu}{\mu_*}
\newcommand{\Amu}{\mathcal{A}_\mu}
\newcommand{\tAmu}{\tilde{\A}_\mu}
\newcommand{\mumeasurable}{$\mu $-measurable }
\newcommand{\As}{\mathcal{A}_{*}} % universally measurable sets
\renewcommand{\P}{\mathbb{P}}
\newcommand{\X}{\mathbb{X}}
\newcommand{\bbF}{\mathbb{F}}
\newcommand{\Rp}{[0,\infty)} % change it back if you want to
\newcommand{\Xt}{(X_t)_{t\geq 0}}

\newcommand{\Ftp}{\mathcal{F}_{t^+}}
\newcommand{\Fsp}{\mathcal{F}_{s^+}}
\newcommand{\REF}{{\color{orange}\textbf{REF}} } % Change to color to orange later on so that you can fill in the missing references.

% BASS APPROACH
\newcommand{\Ps}{\mathbb{P}^{*}} % outer probability
\newcommand{\Ko}{\mathcal{K}^{0}}
\newcommand{\K}{\mathcal{K}}
\newcommand{\Kd}{\mathcal{K}_{\delta}}
\newcommand{\tapproximable}{$t$-approximable }
\newcommand{\Tt}{\mathcal{T}_t}
\newcommand{\pX}{\rho^{X}}

\newcommand{\Lo}{\mathcal{L}_{0}}
\newcommand{\Li}{\mathcal{L}_{1}}
\newcommand{\Ls}{\mathcal{L}_{\sigma}}
\newcommand{\Lsd}{\mathcal{L}_{\sigma\delta}}

\newcommand{\tn}{\tau_{n}}
\newcommand{\Bnm}{B_{nm}}
\newcommand{\Cnm}{C_{nm}}
\newcommand{\xnk}{x_{n_{k}}}
\newcommand{\xinf}{x_{\infty}}
\newcommand{\nk}{n_k}
\newcommand{\pipX}{{\pi\circ\rho}^{X}}
%\newcommand{\tilA}{\tilde{A}}

\newcommand{\Ytd}{Y_{t}^{\delta}}
\newcommand{\UAd}{U_{A}^{\delta}}
\newcommand{\Ftd}{\mathcal{F}_{t}^{\delta}}

%KLASSISCHE ABBILDUNGEN%
\newcommand{\skp}[1]{\left\langle #1 \right\rangle}
\newcommand{\pdv}[2]{\frac{\partial #1}{\partial #2}}
\newcommand{\ppdv}[3]{\frac{\partial^{2} #1}{\partial #2 \partial #3}}
\newcommand{\norm}[1]{{\left\lVert \ifthenelse{\equal{#1}{}}{\, \cdot \,}{#1}\right\rVert}} %durch ifthenelse wird ein default argument eingerichtet was greift, sobald man \norm{} schreibt, dabei ist dann "#1" leer
\newcommand{\betrag}[1]{\left| \ifthenelse{\equal{#1}{}}{\, \cdot \,}{#1} \right|}
\newcommand{\inhalt}[1]{\mu{\left( #1 \right)}} 
\newcommand{\mass}[1]{\mu{\left( #1 \right)}}
\newcommand{\lmass}[1]{\lambda{\left( #1 \right)}}
\newcommand{\smass}[1]{\mu^{\ast}{\left( #1 \right)}}
\newcommand{\wmass}[1]{\mathbf{P}{\left( #1 \right)}}
\newcommand{\indikator}[1]{\mathds{1}_{#1}}
\newcommand{\ugauss}[1]{{\lfloor \ifthenelse{\equal{#1}{}}{\, \cdot \,}{#1} \rfloor}}
\newcommand{\ogauss}[1]{{\lceil \ifthenelse{\equal{#1}{}}{\, \cdot \,}{#1} \rceil}}

















 
%%%%%%%%%%%%%%%%%

% In den folgenden drei Zeilen sorgst du dafür dass es kein Problem mehr damit gibt dass du auf der Titelseite \sc benutzt
\makeatletter
\DeclareOldFontCommand{\sc}{\normalfont\scshape}{\@nomath\sc}
\makeatother
\makeatletter
\DeclareOldFontCommand{\it}{\normalfont\itshape}{\@nomath\it}
\makeatother

%%%%%%%%%%%%%%%%%
%%%%%%%%%%%%%%%%%


\begin{document}
% \includepdf{0_title.pdf}

%\KOMAoptions{DIV=15}
%\KOMAoptions{DIV=22}
% In diesem Dokument wird der TitleHead über eine Tabelle eingebunden
\titlehead
{\centering
\includegraphics[scale=1]{bilder/hu-logo}
}


% Title content
\subject{\sc Humboldt-Universität zu Berlin}
\title{Methods of Statistics (M25)}
\author{Lecturer: {Prof. Dr. Sven Wang}\\
WS 24/25}
\date{Last Update: \today}



\maketitle	% du kannst am ende von diesem Ding auch eine zahl durch zB [3] einbinden, dass ist dann die zahl mit der die Titelseite gezählt wird also \maketitle[3]

% Du kannst hier auch alles in die titlepage umbegung packen, weiß aber nicht genau was sich dann wirklich ändert, scheinbar macht man das aber so meistens


 % Kann man auch mit include machen, so war es ursprünglich
%\newpage\null\thispagestyle{empty}\newpage % leere Seite nach dem Titelblatt



\tableofcontents % Inhaltsverzeichnis direkt nach der Title Page
\thispagestyle{empty}


%%%%%%%%%%%%%%%%%%%%%%%%%%%%%%%%%%%%%%%%%%%%%%%%%%%%%%%%%%%%%%%%%%%%%%%%%%%%%%%%%%%%%%%%%%%%%%%%%%%%%%%%%%%%%%%%%%%%%%%%%%%%%%%%%%%%%%%%%%%%%%%%%%%%%%%%%%%%%%%%%%%%%%%%%%%%%%%%%%%%%%%%%%%%%%%%%%%%%%%%%%%%%%%%%%%%%%%%%%%%%%%%%%%%%%%%%%%%%%%%%%%%%%%%%%%%%%%%%%%%%%%%%%%%%%%%%%%%%%%%%%%%%%%%%%%%%%%%%%%%%%%%%%%%%%%%%%%%%%%%%%%%%%%%%%%%%%%%%%%%%%%%%%%%%%%%%%%%%%%%%%%%%%%%%%%%%%%%%%%%%%%%%%%%%%%%%%%%%%%%%%%%%%%%%%%%%%%%%%%%%%%%%%%%%%%%%%%%%%%%%%%%%%%%%%%%%%%%%%%%%%%%%%%%%%%%%%%%%%%%%%%%%%%%%%%%%%%%%%%%%%%%%%%%%%%%%%%%%%%%%%%%%%%%%%%%%%%%%%%%%%%%%%%%%%%%%%%%%%%%%%%%%%%%%%%%%%%%%%%%%%%%%%%%%%%%%%%%%%%%%%%%%%%%%%%%%%%%%%%%%%%%%%%%%%%%%%%%%%%%%%%%%%%%%%%%%%%%%%%%%%%%%%%%%%%%%%%%%%%%%%%%%%%%%%%%%%%%%%%%%%%%%%%%%%%%%%%%%%%%%%%%%%%%%%%%%%%%%%%%%%%%%%%%%%%%%%%%%%%%%%%%%%%%%%%%%%%%%%%%%%%%%%%%%%%%%%%%%%%%%%%%%%%%%%%%%%%%%%%%%%%%%%%%%%%%%%%%%%%%%%%%%%%%%%%%%%%%%%%%%%%%%%%%%%%%%%%%%%%%%%%%%%%%%%%%%%%%%%%%%%%%%%%%%%%%%%%%%%%%%%%%%%%%%%%%%%%%%%%%%%%%%%%%%%%%%%%%%%%%%%%%%%%%%%%%%%%%%%%%%%%%%%%%%%%%%%%%%%%%%%%%%%%%%%%%%%%%%%%%%%%%%%%%%%%%%%%%%%%%%%%%%%%%%%%%%%%%%%%%%%%%%%%%%%%%%%%%%%%%%%%%%%%%%%%%%%%%%%%%%%%%%%%%%%%%%%%%%%%%%%%%%%%%%%%%%%%%%%%%%%%%%%%%%%%%%%%%%%%%%%%%%%%%%%%%%%%%%%%%%%%%%%%%%%%%%%%%%%%%%%%%%%%%%%%%%%%%%%%%%%%%%%%%%%%%%%%%%%%%%%%%%%%%%%%%%%%%%%%%%%%%%%%%%%%%%%%%%%%%%%%%%%%%%%%%%%%%%%%%%%%%%%%%%%%%%%%%%%%%%%%%%%%%%%%%%%%%%%%%%%%%%%%%%%%%%%%%%%%%%%%%%%%%%%%%%%%%%%%%%%%%%%%%%%%%%%%%%%%%%%%%%%%%%%%%%%%%%%%%%%%%%%%%%%%%%%%%%%%%%%%%%%%%%%%%%%%%%%%%%%%%%%%%%%%%%%%%%%%%%%%%%%%%%%%%%%%%%%%%%%%%%%%%%%%%%%%%%%%%%%%%%%%%%%%%%%%%%%%%%%%%%%%%%%%%%%%%%%%%%%%%%%%%%%%%%%%%%%%%%%%%%%%%%%%%%%%%%%%%%%%%%%%%%%%%%%%%%%%%%%%%%%%%%%%%%%%%%%%%%%%%%%%%%%%%%%%%%%%%%%%%%%%%%%%%%%%%%%%%%%%%%%%%%%%%%%%%%%%%%%%%%%%%%%%%%%%%%%%%%%%%%%%%%%%%%%%%%%%%%%%%%%%%%%%%%%%%%%%%%%%%%%%%%%%%%%%%%%%%%%%%%%%%%%%%%%%%%%%%%%%%%%%%%%%%%%%%%%%%%%%%%%%%%%%%%%%%%%%%%%%%%%%%%%%%%%%%%%%%%%%%%%%%%%%%%%%%%%%%%%%%%%%%%%%%%%%%%%%%%%%%%%%%%%%%%%%%%%%%%%%%%%%%%%%%%%%%%%%%%%%%%%%%%%%%%%%%%%%%%%%%%%%%%%%%%%%%%%%%%%%%%%%%%%%%%%%%%%%%%%%%%%%%%%%%%%%%%%%%%%%%%%%%%%%%%%%%%%%%%%%%%%%%%%%%%%%%%%%%%%%%%%%%%%%%%%%%%%%%%%%%%%%%%%%%%%%%%%%%%%%%%%%%%%%%%%%%%%%%%%%%%%%%%%%%%%%%%%%%%%%%%%%%%%%%%%%%%%%%%%%%%%%%%%%%%%%%%%%%%%%%%%%%%%%%%%%%%%%%%%%%%%%%%%%%%%%%%%%%%%%%%%%%%%%%%%%%%%%%%%%%%%%%%%%%%%%%%%%%%%%%%%%%%%%%%%%%%%%%%%%%%%%%%%%%%%%%%%%%%%%%%%%%%%%%%%%%%%%%%%%%%%%%%%%%%%%%%%%%%%%%%%%%%%%%%%%%%%%%%%%%%%%


% REWORKED CODE

%%%%%%%%%%%%%%%%%%%%%%%%%%%%%%%%%%%%%%%%%%%%%%%%%%%%%%%%%%%%%%%%%%%%%%%%%%%%%%%
%%%%%%%%%%%%%%%%%%%%%%%%%%%%%%%%%%%%%%%%%%%%%%%%%%%%%%%%%%%%%%%%%%%%%%%%%%%%%%%
\chapter{Intro and Disclaimer}\label{ch:intro}
%%%%%%%%%%%%%%%%%%%%%%%%%%%%%%%%%%%%%%%%%%%%%%%%%%%%%%%%%%%%%%%%%%%%%%%%%%%%%%%
%%%%%%%%%%%%%%%%%%%%%%%%%%%%%%%%%%%%%%%%%%%%%%%%%%%%%%%%%%%%%%%%%%%%%%%%%%%%%%%

These lecture notes are based on the material presented by Professor Wang during class and are written by students. They may contain errors or omissions. Please refer to the in-person lectures and the literature on Moodle for accurate and authoritative information. 
If you find an error or want to help, please send an email to:
\begin{center}
    \href{mailto:said.kassner@student.hu-berlin.de}{said.kassner@student.hu-berlin.de}
\end{center}

\begin{center}
    \href{mailto:stephensonmonroe@gmail.com}{stephensonmonroe@gmail.com}
\end{center}

\begin{center}
    \href{mailto:salihiad@hu-berlin.de}{salihiad@hu-berlin.de}
\end{center}




%%%%%%%%%%%%%%%%%%%%%%%%%%%%%%%%%%%%%%%%%%%%%%%%%%%%%%%%%%%%%%%%%%%%%%%%%%%%%%%
%%%%%%%%%%%%%%%%%%%%%%%%%%%%%%%%%%%%%%%%%%%%%%%%%%%%%%%%%%%%%%%%%%%%%%%%%%%%%%%
\chapter{Basic Statistical Concepts}\label{ch:intro}
%%%%%%%%%%%%%%%%%%%%%%%%%%%%%%%%%%%%%%%%%%%%%%%%%%%%%%%%%%%%%%%%%%%%%%%%%%%%%%%
%%%%%%%%%%%%%%%%%%%%%%%%%%%%%%%%%%%%%%%%%%%%%%%%%%%%%%%%%%%%%%%%%%%%%%%%%%%%%%%

\section*{Lecture 1}
Here is the literature that the class is based on.

\textbf{Literature:}
\begin{itemize}
    \item WS 19/20 R. Altmeyer \textit{"Gliederung Methoden der Statistik"}
    \item L. Wasserman, \textit{All of Statistics}
    \item M. Trabs, K. Krenz, M. Jirak and M. Reiss. \textit{Methoden der Statistik.}
    \item Hastie, Tibshirani, et al., \textit{Elements of Statistical Learning}
\end{itemize}

Let’s start with a (simplest possible) example:

\begin{example}[Polling]\label{ex:polling}
Consider a poll with two answers A and B (representing political parties).
\begin{itemize}
    \item \( N \) = total number of votes
    \item \( M \) = total number of votes supporting party A
\end{itemize}
\end{example}

\textbf{Poll Definitions:}
\begin{itemize}
    \item \( n \) = size of the poll
    \item \( x = (x_1, ..., x_n) \) = responses, where:
    \[
    x_i =
    \begin{cases} 
    0 & \text{if the i\text{-}th person supports B} \\
    1 & \text{if the i\text{-}th person supports A}
    \end{cases}
    \]
\end{itemize}

\textbf{Additional Assumptions:}
\begin{itemize}
    \item \( n \)-times, we select a person randomly from the set \( \{1, ..., N\} \), and record their (truthful) response.
    \item Every asked person responds (i.e., no selection bias).
    \item People can be asked repeatedly.
\end{itemize}

\textbf{Aim of the Poll:} The aim of the poll is to estimate the fraction of party A supporters. This can be written as:

\[
\theta = \frac{M}{N} \in [0, 1]
\]

An intuitive estimate of \( \theta \) is:

\[
\hat{\theta} = \frac{1}{n} \sum_{i=1}^{n} x_i
\]

\textbf{Question:} Is this a good (or best possible) estimator? What properties does it have?

To answer this, we formalize some statistical notions.

\begin{definition}[Sample space]\label{def:samplespace}
A \textit{sample space} is a measurable space \((\mathcal{X}, \mathcal{F})\), i.e., a set \(\mathcal{X}\) with a \(\sigma\)-algebra \(\mathcal{F}\), in which our statistical observations take values.
\end{definition}

\begin{definition}[Statistical model]\label{def:statmodel}
Let \((\mathcal{X}, \mathcal{F})\) be some sample space and let \(\Theta\) be a set, which we call the \textit{parameter space}. A \textit{statistical model} is a family of probability measures \(\{P_\theta : \theta \in \Theta\}\) on \((\mathcal{X}, \mathcal{F})\).
\end{definition}

\begin{remark}
Often, \((\mathcal{X}, \mathcal{F})\) is a "product space." For example, in Example \ref{ex:polling}, \(\mathcal{X} = \{0, 1\}^n\), and each \(P_\theta\) is a product distribution, i.e., \(x_1, \dots, x_n\) are independent, identically distributed. Then we say \(\{P_\theta : \theta \in \Theta\}\) is an \textit{iid statistical model}.
\end{remark}

\begin{remark}[Back to Example \ref{ex:polling}]
Here:
\begin{itemize}
    \item \(\mathcal{X} = \{0, 1\}^n\)
    \item \(\Theta = [0, 1]\)
    \item \(\mathcal{F} = \mathcal{P}(\{0, 1\}^n)\)
    \item \(P_\theta = (\text{Bernoulli}(\theta))^{\otimes n}\)
\end{itemize}
\end{remark}

\begin{remark}
If every person could only be asked once, we would have \( P_\theta = \text{Hypergeometric}(N, M, n) \), which "converges" to the Bernoulli model as \( N, M \to \infty \). 
We might have to discretize \(\Theta\) and take \( \theta = \frac{M}{N} \). 
(Exercise: Think about it!)
\end{remark}

\chapter{Parameter Estimation}
Assume that \( \Theta \subseteq \mathbb{R}^p \), for \( p \geq 1 \). This is the setting of parametric statistics. 
[Assume \(\Theta\) is measurable.]

\begin{definition}[Estimator]\label{def:estimator}
An estimator for \( \theta \in \Theta \) is any measurable function:
\[
\hat{\theta}: (\mathcal{X}, \mathcal{F}) \to \Theta.
\]
Any function that, based on some data \( x \in \mathcal{X} \), outputs a guess / estimate \( \hat{\theta}(x) \in \Theta \).
\end{definition}


\section*{Lecture 2}

\textbf{Last time:} Statistical model = family of probability measures on $(\mathcal{X}, \mathcal{F})$ indexed by $\theta \in \Theta$. 

\textbf{Sample space:} $(\mathcal{X}, \mathcal{F})$

\textbf{Estimator:} = measurable function $(\mathcal{X}, \mathcal{F}) \to \Theta$

Now, what are some desirable properties we would like to have?

\begin{definition}[Unbiased estimator]\label{def:unbiased}
Let $\Theta = \mathbb{R}^p$ (measurable), $p \geq 1$. An estimator $\hat{\theta}$ is unbiased if 
\[
    \mathbb{E}_{\theta}[\hat{\theta}] = \mathbb{E}_{\mathbb{P}_\theta}[\hat{\theta}] = \theta, \text{ for all } \theta \in \Theta.
\]
Where $\mathbb{E}_{\theta}[\cdot] = \mathbb{E}_{\mathbb{P}_\theta}[\cdot]$ denotes expectation under the law $\mathbb{P}_\theta$.
\end{definition}


In more explicit terms:
\[
    \mathbb{E}_{x \sim \mathbb{P}_\theta}[\hat{\theta}(x)] = \theta \quad \forall \theta
\]

\begin{remark}[Unbiasedness]\label{rem:unbiasedness}
Unbiasedness means "no systematic errors." However, we'd also like a "good" $\hat{\theta}$ to be concentrated around the data-generating parameter.
\end{remark}

\begin{definition}[Consistent estimator]\label{def:consistent}
Let $(\mathbb{P}_{\theta}^n)_{\theta \in \Theta}$ be a sequence of statistical models ($n \geq 1$), on the same parameter space $\Theta$ not depending on $n \geq 1$.

Let $\hat{\theta}_n$ be a sequence of estimators. Then $\hat{\theta}_n$ is called consistent if for every $\theta \in \Theta$,
\[
    \hat{\theta}_n \xrightarrow{\mathbb{P}_\theta^n} \theta
\]
or explicitly, for every $\epsilon > 0$,
\[
    \lim_{n \to \infty} \mathbb{P}^n_\theta (|\hat{\theta}_n - \theta| > \epsilon) = 0.
\]
\end{definition}

\textbf{Back to Example \ref{ex:polling}:}
\begin{itemize}
    \item $X_i = \{0, 1\}^n$
    \item $\Theta = [0, 1]$
    \item $\mathbb{P}_\theta^n = \text{Bernoulli}(\theta)^{\otimes n}$
    \item $\hat{\theta}_n = \frac{1}{n} \sum_{i=1}^{n} X_i$
\end{itemize}

\textbf{Unbiasedness:} \\
Let $\theta \in \Theta$, then
\[
    \mathbb{E}_\theta \left[ \frac{1}{n} \sum_{i=1}^{n} X_i \right] = \frac{1}{n} \sum_{i=1}^{n} \mathbb{E}_\theta [X_i] = \frac{1}{n} \sum_{i=1}^{n} \theta = \theta.
\]
Thus, $\mathbb{E}_\theta [\hat{\theta}_n] = \theta$.

\textbf{Consistency:}
\begin{itemize}
    \item We could use the Weak Law of Large Numbers (WLLN).
    \item Alternatively,
\[
    \text{Var}_\theta(\hat{\theta}_n) = \frac{1}{n^2} \sum_{i=1}^{n} \text{Var}_\theta (X_i) = \frac{1}{n^2} \sum_{i=1}^{n} \theta (1-\theta) = \frac{\theta(1-\theta)}{n}
\]
which tends to zero as $n \to \infty$.
\end{itemize}

It follows: For every $\epsilon > 0$,
\[
    \mathbb{P}^n_\theta(|\hat{\theta}_n - \theta| > \epsilon) = \mathbb{P}^n_\theta(|\hat{\theta}_n - \mathbb{E}[\hat{\theta}_n]| > \epsilon)
\]
By Markov's inequality:
\[
    \mathbb{P}^n_\theta(|\hat{\theta}_n - \mathbb{E}[\hat{\theta}_n]| > \epsilon) \leq \frac{\mathbb{E}_\theta \left[ (\hat{\theta}_n - \mathbb{E}[\hat{\theta}_n])^2 \right]}{\epsilon^2} = \frac{\text{Var}_\theta (\hat{\theta}_n)}{\epsilon^2} = \frac{\theta(1-\theta)}{n \epsilon^2}
\]
which tends to zero as $n \to \infty$. Thus,
\[
    (\hat{\theta}_n : n \geq 1) \text{ is consistent. } \quad \square
\]

\section{Maximum Likelihood Principle}

Is there another way to motivate $\hat{\theta}_n = \frac{1}{n} \sum_{i=1}^{n} X_i$? Yes, it turns out it is the maximum likelihood estimator, i.e.,
\[
    \text{MLE} = \text{"parameter which assigns the highest probability to the observed data."}
\]

\textbf{In our example}, each $\mathbb{P}_\theta^n$ has a probability density (likelihood)
\[
    \mathbb{P}_\theta^n(x) = \prod_{i=1}^{n} \mathbb{P}_\theta(x_i) = \prod_{i=1}^{n} \theta^{x_i} (1-\theta)^{1-x_i}
\]
\[
    = \theta^{\sum_{i=1}^{n} x_i} (1-\theta)^{n-\sum_{i=1}^{n} x_i}.
\]

Fixing $x \in \{0,1\}^{n}$ and maximizing in $\theta \in [0,1]$ gives the following:

\begin{itemize}
    \item If $\sum_{i=1}^{n} x_i = 0$, then $\hat{\theta}_n = 0$ is the maximizer.
    \item If $\sum_{i=1}^{n} x_i = n$, then $\hat{\theta}_n = 1$ is the maximizer.
    \item If $\hat{\theta}_n \in \{1, \ldots, n-1\}$, then writing $S_n = \sum_{i=1}^{n} x_i$ gives:
\end{itemize}

\[
    \frac{\partial}{\partial \theta} \mathbb{P}_\theta^n(x) = S_n \theta^{S_n-1} (1-\theta)^{n-S_n-1} - (n-S_n)\theta^{S_n} (1-\theta)^{n-S_n-1} = 0
\]
\[
    \Leftrightarrow S_n (1 - \theta) - \theta (n - S_n)
\]
\[
    \Leftrightarrow \theta = \frac{S_n}{n}.
    \quad \square
\]

\begin{definition}[Dominated statistical model \& MLE]\label{def:dominatedmodel}
A model $(\mathbb{P}_\theta^n)_{\theta \in \Theta}$ is called dominated if there exists a measure $\mu$ on $(\mathcal{X}, \mathcal{F})$ such that for every $\theta \in \Theta$, $\mathbb{P}_\theta \ll \mu$ or equivalently (by Radon-Nikodym), for all $\theta \in \Theta$, there is a probability density $\frac{\mathrm{d} \mathbb{P}_\theta}{\mathrm{d}\mu}$ of $\mathbb{P}_\theta$ with respect to $\mu$.

The MLE is defined as any $\hat{\theta} \in \Theta$ that maximizes the function
\[
    \theta \mapsto \frac{\mathrm{d} \mathbb{P}_\theta}{\mathrm{d}\mu}(x) = \mathbb{P}_\theta(x).
\]
\end{definition}

\begin{remark}[Caveats]\label{rem:caveats}
\begin{itemize}
    \item MLE might not be unique.
    \item MLE might not exist.
    \item It's not always clear that some selection $\hat{\theta}(x) \in \arg \max_{\theta} \mathbb{P}_\theta(x)$ is a measurable function of $x \in \mathcal{X}$. However, there are measurable selection theorems that permit a measurable choice of $\hat{\theta}$ under very general conditions.
\end{itemize}
\end{remark}

\begin{remark}\label{rem:mu}
In all the models we study, we will work with the Lebesgue measure (for continuous data) or the counting measure (for discrete data).
\end{remark}

\begin{example}[Normal model]\label{ex:normalmodel}
Consider random samples $X_1, \dots, X_n \sim N(\mu, \sigma^2)$ for some unknown $\mu \in \mathbb{R}$, $\sigma^2 > 0$, and let $\lambda$ denote the Lebesgue measure on $\mathbb{R}^n$.

\[
    \theta = (\mu, \sigma^2) \in \mathbb{R} \times (0, \infty).
\]
Then the likelihood is:
\[
    L(\mu, \sigma^2 \mid x) = \prod_{i=1}^{n} \frac{1}{\sqrt{2 \pi \sigma^2}} \exp \left( - \frac{1}{2} \left( \frac{X_i - \mu}{\sigma} \right)^2 \right)
\]
\[
    = (2 \pi \sigma^2)^{-n/2} \exp \left( - \frac{1}{2 \sigma^2} \norm{X-\mu\cdot 1_n}^2 \right),
\]
where by $1_n$ we denote the vector of ones of dimension $n$.

Here, the MLE is given as:
\[
    \hat{\mu} = \frac{1}{n} \sum_{i=1}^{n} X_i \quad \text{[Sample mean]}, \quad \hat{\sigma}^2 = \frac{1}{n} \sum_{i=1}^{n} (X_i - \hat{\mu})^2 \quad  \text{[Sample variance]}.
\]
\[
    \mathbb{E}_\theta[\hat{\mu}] = \mu, \quad \mathbb{E}_\theta[\hat{\sigma}^2] = \frac{n-1}{n} \sigma^2, \quad \text{so } \hat{\sigma}^2 \text{ is biased.}
\]
\[
    \Rightarrow \text{MLE is not always unbiased, and not always a "good" method.}
\]

\end{example}

\section{Bayesian method}

\paragraph{Motivation}
In Bayesian statistics, a key element is the prior distribution, which we denote by $\pi$, reflecting our "beliefs" about the parameter $\theta \in \Theta$ before observing data ($\pi$ is a probability measure on $\Theta$).

A prior $\pi$, together with a model $(P_\theta : \theta \in \Theta)$, gives rise to a joint probability distribution for the pair $(\theta, x) \in \Theta \times \mathcal{X}$.

The Bayesian approach bases statistical inference on the posterior distribution of $\theta$ conditioned on $x$.

\textbf{Joint probability:}
\[
    (\theta, x) \mapsto \pi(\theta) P_\theta(x)
\]
\hspace{4cm} \text{conditional distribution of } $x \mid \theta$

\textbf{Posterior:}
\[
    \pi(\theta \mid x) = \frac{\pi(\theta) P_\theta(x)}{\int_{\Theta} \pi(\theta) P_\theta(x) \, d\theta}
\]

\begin{remark}
Bayesian methods automatically generate "error bars" because the posterior is not an estimator but a whole probability distribution.
\end{remark}



\section*{Lecture 3}

\begin{definition}[Prior, Posterior, Bayes' Rule]\label{def:prior, posterior, bayes rule}
Let $\mathcal{F}_\Theta$ be a $\sigma$-algebra on $\Theta$, and suppose 
\[
\left\{ P_\theta : \theta \in \Theta \right\}
\]
is a dominated statistical model with densities $p_\theta(x)$, and assume that 
\[
(\theta, x) \mapsto p_\theta(x)
\]
is "jointly measurable" (i.e., w.r.t. $\sigma(\mathcal{F}_\Theta \times \mathcal{F})$).

Let $\Pi$ be a prior distribution on $\Theta$, with density $\pi(\theta)$ w.r.t. measure $\nu(\cdot)$. Then, define the posterior density
\[
\pi(\theta \mid x) := \frac{p_\theta(x) \pi(\theta)}{\int_\Theta p_{\tilde{\theta}}(x) \, d\Pi(\tilde{\theta})}.
\]
\end{definition}

The corresponding probability measure $\Pi(\cdot \mid x)$ is called the posterior distribution:
\[
\Pi(B \mid x) = \int_B \pi(\theta \mid x) \, d\nu(\theta), \quad B \in \mathcal{F}_\Theta.
\]
\[
= \frac{\int_B p_\theta(x) \, \pi(\theta) \, d\nu(\theta)}{\int_\Theta p_{\tilde{\theta}}(x) \, d\Pi(\tilde{\theta})},
\]
\[
= \frac{\int_B p_\theta(x) \, d\Pi(\theta)}{\int_\Theta p_{\tilde{\theta}}(x) \, d\Pi(\tilde{\theta})},
\]

\begin{remark}
Think of $\Theta \subseteq \mathbb{R}^p$, $\nu(\cdot)$ as a Lebesgue measure, $\pi(\cdot)$ as a Lebesgue density.

\textbf{Exception:} $\Theta = \{0,1\}$ in hypothesis testing. Then, we'd take $\nu(\cdot)$ to be the counting measure.
\end{remark}

From the posterior, we can derive several estimators:
\begin{itemize}
    \item \textbf{Maximum-a-posterior (MAP) estimator:}
    \[
    \hat{\theta}_{\text{MAP}}(x) = \underset{\theta \in \Theta}{\text{argmax}} \, \pi(\theta \mid x).
    \]
    \item \textbf{Posterior mean:} Say \( \Theta \subseteq \mathbb{R}^p \) convex
    \[
    \hat{\theta}(x) = \int_\Theta \theta \, \pi(\theta \mid x) \, d\nu(\theta) \in \mathbb{R}^p.
    \]
\end{itemize}

\textbf{Back to Example 1.1:}  
\textit{Binomial model:} $\mathcal{X} = \{0, 1, \dots, n\}$,  
$p_\theta = \text{Bin}(n, \theta)$,  
$\theta \in \Theta = [0,1]$.  

\textit{Prior (uniform):} $\Pi = \text{Unif}(0,1)$.

We know: 
\[
\hat{\theta}_{\text{MAP}} = \hat{\theta}_{\text{MLE}} \quad \text{(for the uniform prior)},
\]
\[
\hat{\theta}_{\text{MAP}} = \frac{X}{n}.
\]

\begin{itemize}
    \item \textbf{Posterior mean}:
    \[
    \pi(\theta | x) = \frac{p_\theta(x)}
    {\int p_{\tilde{\theta}}(x) d\tilde{\theta}} \propto  \binom{n}{k} \theta^{x}(1-\theta)^{n-x}.
    \]
    
    \item \textbf{Binomial distribution}:
    \[
    \text{Bin}(n, p)(k) = \binom{n}{k} p^k (1-p)^{n-k},
    \]
    where \( k \in \{0, \ldots, n\} \) is the number of successes, and \( p \) is the probability of success, and $n$ would be interpreted as the "number of coin flips".
\end{itemize}

\[
\pi(\theta | x) \propto \theta^x (1-\theta)^{n-x}.
\]
and 

\[ \int_0^1 \pi(\theta \mid x) d\nu  =1\]

We conclude that \( \pi(\theta | x) \) is a \textbf{Beta-distribution} on \( [0,1] \),
\[
\text{Beta}(x+1, n-x+1).
\]

The mean is given by:
\[
\hat{\theta} = \frac{x+1}{n+2}.
\]

\begin{remark}[Beta distribution]
The Beta distribution is defined as:
\[
\text{Beta}(a, b), \quad a,b \geq 0.
\]
The probability density function of the Beta distribution is given by:
\[
P_{\text{Beta}(a,b)}(x) = x^{a}(1-x)^{b}.
\]
\end{remark}

\begin{definition}[Conjugate Bayesian models]
Let \((P_\theta : \theta \in \Theta)\) be a statistical model. Then, some family \(\mathcal{D}\) of p.m.s on \(\Theta\) is called \textit{conjugate} if
\[
\Pi \in \mathcal{D} \implies \Pi(\cdot | x) \in \mathcal{D} \quad \text{for all } x \in \mathcal{X}.
\]
\end{definition}

\noindent \textbf{Examples:}
\begin{itemize}
    \item \((\text{Bin}(n, \theta)) : \theta \in [0,1], \quad \mathcal{D} = \text{Beta}(a,b), \ a,b \geq 0.\)
    \item \((\mathcal{N}(\mu, \sigma^2)) : \mu \in \mathbb{R}, \quad \mathcal{D} = \left\{\mathcal{N}\left(\mu, n^2\right), \mu \in \mathbb{R}, n^2 > 0 \right\}, \ \sigma^2 \)\text{ known}.
\end{itemize}


\chapter{Decision Theory}

Here suppose that \(\Theta \subseteq \mathbb{R}^p\).
\begin{definition}[Loss function]
A function \(\ell : \Theta \times \mathbb{R}^p \to [0, \infty)\) is a \textit{loss function} if for every \(\theta \in \Theta\), \(\ell(\theta, \cdot)\) is measurable. Given some estimator \(\hat{\theta}\), the expected loss is
\[
\mathcal{R}(\theta, \hat{\theta}) = \mathbb{E}_\theta \left[\ell(\theta, \hat{\theta})\right].
\]
\end{definition}

\noindent \textbf{Example:} Take \(\ell(\theta, \hat{\theta}) = \| \theta - \hat{\theta} \|^2_{\mathbb{R}^p}.\) Then,
\[
\mathcal{R}(\theta, \hat{\theta}) = \mathbb{E}_\theta \left[\|\theta - \hat{\theta}\|^2_{\mathbb{R}^p} \right]
\]
is the mean squared error (MSE).

\begin{proposition}[Bias-Variance Decomposition]
    Let $\hat{\theta} \in L^2(\mathbb{P}_\theta)$. Then it holds that:
    \[
    R(\hat{\theta}, \theta) = (\mathbb{E}_\theta[\hat{\theta}] - \theta)^2 + \operatorname{Var}_\theta(\hat{\theta}).
    \]
\end{proposition}

\begin{proof}
    We have
    \[
    R(\hat{\theta}, \theta) = \mathbb{E}_\theta[(\hat{\theta} - \mathbb{E}_\theta[\hat{\theta}] + \mathbb{E}_\theta[\hat{\theta}] - \theta)^2]
    \]
    Expanding the squared term:
    \[
    = \mathbb{E}_\theta[(\hat{\theta} - \mathbb{E}_\theta[\hat{\theta}])^2] + (\mathbb{E}_\theta[\hat{\theta}] - \theta)^2 + 2\mathbb{E}_\theta[(\hat{\theta} - \mathbb{E}_\theta[\hat{\theta}])(\mathbb{E}_\theta[\hat{\theta}] - \theta)].
    \]
    Since $\mathbb{E}_\theta[\hat{\theta} - \mathbb{E}_\theta[\hat{\theta}]] = 0$, the last term vanishes, leaving us with:
    \[
    R(\hat{\theta}, \theta) = \operatorname{Var}_\theta(\hat{\theta}) + (\mathbb{E}_\theta[\hat{\theta}] - \theta)^2.
    \]
\end{proof}

\begin{definition}[Minimax Risk]
    Given an estimator $\hat{\theta}$ in a model $(\mathbb{P}_\theta : \theta \in \Theta)$, the maximal risk of $\hat{\theta}$ is
    \[
    \sup_{\theta \in \Theta} R(\theta, \hat{\theta}).
    \]
    The minimax risk of a model $(\mathbb{P}_\theta : \theta \in \Theta)$ is given as
    \[
    \inf_{\hat{\theta}} \sup_{\theta \in \Theta} R(\theta, \hat{\theta}),
    \]
    where the infimum is taken over all estimators. An estimator is called minimax if
    \[
    \sup_{\theta \in \Theta} R(\theta, \hat{\theta}) = \inf_{\hat{\theta}} \sup_{\theta \in \Theta} R(\theta, \hat{\theta}).
    \]
\end{definition}

\begin{definition}[Bayes Risk]
    Given a prior $\pi$ on $\Theta$, the $\pi$-Bayes risk of a decision rule $\delta$ for the loss function $L$ is defined as
    \[
    R_\Pi(\delta) = \mathbb{E}_\Pi [R(\delta, \theta)] = \int_{\Theta} R(\delta, \theta) \pi(\theta) d\theta = \int_{\Theta} \int_{\mathcal{X}} L(\delta(x), \theta) \pi(\theta) p_\theta(x) dx d\theta.
    \]
    A $\Pi$-Bayes decision rule $\hat\theta_\Pi$ is any decision rule that minimizes $R_\Pi(\hat\theta)$.
\end{definition}

\textcolor{red}{SW: $\ell$ instead of $L$ below, $p_{\theta}(x)$ instaed of  $f(x, \theta)$ }
\textcolor{green}{Note: Has been corrected. }

\begin{definition}[Posterior Risk]
    For a Bayesian model, the posterior risk $R_{\Pi}$ is defined as the average loss under the posterior distribution for some observation $x \in \mathcal{X}$:
    \[
    R_{\Pi(\cdot|x)}(\delta) = \mathbb{E}_\Pi[\ell(\delta(x), \theta) | x].
    \]
    Here, the notation $\mathbb E_\Pi[\cdot|x]$ stands for the expectation under the posterior distribution.
\end{definition}

\begin{proposition}[Bayes Risk and Posterior Risk]
    An estimator $\delta$ that minimizes the $\Pi$-posterior risk $R_{\Pi}$ also minimizes the $\pi$-Bayes risk $R_\pi$.
\end{proposition}

\begin{proof}
    The $\pi$-Bayes risk can be rewritten as
\begin{align*}
    R_\pi(\delta) &= \int_{\Theta} \mathbb{E}_\theta[\ell(\delta(X), \theta)] \pi(\theta) d\theta\\
    &= \int_{\Theta} \int_{\mathcal{X}} \ell(\delta(x), \theta) \pi(\theta) p_{\theta}(x) dx d\theta\\
    &= \int_{\mathcal{X}} \int_{\Theta} \ell(\delta(x), \theta) \frac{p_{\theta}(x) \pi(\theta)}{\int_{\Theta} p_{\theta'}(x) \pi(\theta') d\theta'} \times \underbrace{\int_{\Theta} p_{\theta'}(x) \pi(\theta') d\theta'}_{=: n(x) \geq 0} dx d\theta\\
    &= \int_{\mathcal{X}} \mathbb{E}_{\Pi}[\ell(\delta(x), \theta) | x] n(x) dx.
\end{align*}
    [Notation $n(x)$ motivated by the word `normalising constant'].

    
    Let $\delta_\Pi$ be a decision rule that minimizes the posterior risk, i.e., such that for all $x \in \mathcal{X}$,
    \[
    \mathbb{E}_{\Pi}[\ell(\delta_\Pi(x), \theta) | x] \leq \mathbb{E}_{\Pi}[\ell(\delta(x), \theta) | x].
    \]
    Multiplying by $m(x) \geq 0$ and integrating on both sides over $\mathcal{X}$ yields the desired result.
\end{proof}

\begin{example}
    For the quadratic risk with the squared-loss, the posterior risk is minimized by taking $\delta(X) = \mathbb{E}_{\Pi}[\theta | X]$, by minimizing the quadratic function in $\delta$. Other losses will give other ways to minimize the posterior risk, and other Bayes decision rules.
\end{example}

\begin{proposition}
Let \((\mathcal{X}, \mathcal{F}, (\mathbb{P}_{\theta})_{\theta \in \Theta})\) be a statistical model and let \(\hat{\theta}\) be an estimator. Then we have
\[
\sup_{\theta \in \Theta} R(\hat{\theta}, \theta) = \sup_{\Pi} \int_{\Theta} R(\hat{\theta}, \theta) \, \Pi(d\theta),
\]
where the supremum is taken over all prior distributions \(\Pi\).
\end{proposition}

\begin{proof}
Obviously, we have 
\[
\int_{\Theta} R(\hat{\theta}, \theta) \, \Pi(d\theta) \leq \sup_{\theta \in \Theta} R(\hat{\theta}, \theta).
\]
On the other hand, by using the prior distributions \(\delta_{\theta}\) (Dirac measure on \(\theta \in \Theta\)), we obtain
\[
\sup_{\Pi} \int_{\Theta} R(\hat{\theta}, \theta) \, \Pi(d\theta) \geq \int_{\Theta} R(\hat{\theta}, \theta) \, \delta_{\theta}(d\theta) = R(\hat{\theta}, \theta).
\]
\end{proof}

[Note: In the following we use the notation $\delta$ for decision rules while on the blackboard we used $\hat{\theta}$ or $\tilde{\theta}$. If you want to adjust this please contact me so that I can give you access.]

\begin{proposition}
Let \(\pi\) be a prior on \(\Theta\) such that
\[
R_{\pi}(\delta_{\pi}) = \sup_{\theta \in \Theta} R(\delta_{\pi}, \theta),
\]
where \(\delta_{\pi}\) is a \(\pi\)-Bayes rule. Then it holds that
\begin{enumerate}
    \item The rule \(\delta_{\pi}\) is minimax.
    \item If \(\delta_{\pi}\) is unique Bayes, then it is unique minimax.
\end{enumerate}
\end{proposition}

\begin{proof} Let \(\delta\) be any decision rule. Then
\[
\sup_{\theta \in \Theta} R(\delta, \theta) \geq \mathbb{E}_{\pi}[R(\delta, \theta)],
\]
\[
\int_{\Theta} R(\delta, \theta) \pi(\theta) \, d\theta \geq \mathbb{E}_{\pi}[R(\delta, \theta)],
\]
\[
\int_{\Theta} R(\delta, \theta) \pi(\theta) \, d\theta = R_{\pi}(\delta_{\pi}) = \sup_{\theta \in \Theta} R(\delta_{\pi}, \theta).
\]
Taking the infimum over \(\delta\) gives the result.

2. If \(\delta_{\pi}\) is unique Bayes, the second inequality is strict for any \(\delta' \neq \delta_{\pi}\).
\end{proof}

\begin{corollary}
If a (unique) Bayes rule \(\delta_{\pi}\) has constant risk in \(\theta\), then it is (unique) minimax.
\end{corollary}

\begin{proof}
If a Bayes rule \(\delta_{\pi}\) has constant risk, then
\[
R_{\pi}(\delta_{\pi}) = \mathbb{E}_{\pi}[R(\delta_{\pi}, \theta)] = \sup_{\theta \in \Theta} R(\delta_{\pi}, \theta),
\]
where \(R(\delta_{\pi}, \theta)\) is constant in \(\theta\). Uniqueness of the Bayes rule implies uniqueness of the minimax rule, as in part 2 of the former proposition.
\end{proof}

\begin{example}
Hence, if the maximal risk of a Bayes rule \(\delta_{\pi}\) equals the Bayes risk, then \(\pi\) is least favorable, and the corresponding Bayes rule is minimax.
\begin{itemize}
    \item In a \(\text{Bin}(n, \theta)\) model, let \(\pi_{a,b}\) be a \(\text{Beta}(a, b)\) prior on \(\theta \in [0,1]\). Then the unique Bayes rule for \(\pi_{a,b}\) over the quadratic risk is the posterior mean \(\delta_{a,b} = \bar{\theta}_{a,b}\). Trying to solve the equation
    \[
    R(\delta_{a,b}, \theta) = \text{const.} \quad \forall \theta \in [0,1]
    \]
    we can find a prior \(\pi_{a^*, b^*}\) and a corresponding Bayes rule \(\delta_{\pi_{a^*, b^*}}\) of constant risk. It is therefore unique minimax, and different from the MLE (see Examples sheet).
    
    \item In a \(\mathcal{N}(\theta, 1)\) model, \(\bar{X}_n\) is minimax, as proved later.
\end{itemize}
\end{example}

\section{Another optimality concept: Admissibility}

\begin{definition}
A decision rule \(\delta\) is \emph{inadmissible} if there exists \(\delta'\) such that
\[
R(\delta', \theta) \leq R(\delta, \theta) \quad \forall \theta \in \Theta \quad \text{and} \quad R(\delta', \theta) < R(\delta, \theta) \quad \text{for some } \theta \in \Theta.
\]
\end{definition}

\begin{remark}
\begin{itemize}
    \item The intuition is that there is no reason to choose an inadmissible estimator or decision rule: it would always be better to choose another estimator that dominates it.
    \item Admissibility is not the only criterion to evaluate an estimator: In most cases, a constant estimator will be admissible for the quadratic risk, but it is often not reasonable.
\end{itemize}
\end{remark}

\begin{proposition}
\begin{enumerate}
    \item A unique Bayes rule is admissible.
    \item If \(\delta\) is admissible and has constant risk, then it is minimax.
\end{enumerate}
\end{proposition}

Proof may be done in the Examples sheet.

\begin{definition}
For a vector \(X \in \mathbb{R}^p\), the \emph{James–Stein estimator} is defined as
\[
\delta^{JS}(X) = \left(1 - \frac{p - 2}{\|X\|^2}\right) X.
\]
\end{definition}

In a Gaussian model \(X \sim \mathcal{N}(\theta, I_p)\) for \(\theta \in \mathbb{R}^p\) (with a single observation, to simplify notation), the risk of the MLE is given by
\[
R(\hat{\theta}_{\text{MLE}}, \theta) = \mathbb{E}_{\theta}[\|X - \theta\|^2] = \sum_{j=1}^p \mathbb{E}_{\theta}[(X_j - \theta_j)^2] = p.
\]

For \( X \sim \mathcal{N}(\theta, I_p) \) with \( p \geq 3 \), the risk of the James–Stein estimator satisfies for all \( \theta \in \mathbb{R}^p \)
\[
R(\delta^{JS}, \theta) < p.
\]

\chapter{Confidence Sets}

\begin{definition}[Confidence Set]
Let \((\mathcal{X}, \mathcal{F}, (\mathbb{P}_{\theta})_{\theta \in \Theta})\) be a statistical model. For a given \(\alpha \in [0,1]\), consider sets \( C_{1-\alpha}(x) \subseteq \Theta \) for each \( x \in \mathcal{X} \). Then \( C_{1-\alpha}(x) \) is called a random confidence set at level \( 1 - \alpha \) (or with coverage probability \( 1 - \alpha \)) if
\[
\forall \theta \in \Theta : \mathbb{P}_{\theta}(\theta \in C_{1-\alpha}) = \mathbb{P}_{\theta}( \{ x \in \mathcal{X} : \theta \in C_{1-\alpha}(x) \}) \geq 1 - \alpha.
\]
\end{definition}

\textbf{Note:} The following example was only started in Lecture 4 and may be fully covered in Lecture 5.

\begin{example}
Consider the statistical model \((\{0,1\}^n, \mathcal{P}(\{0,1\}^n), (\mathbb{P}_p)_{p \in [0,1]})\) with \(\mathbb{P}_p = \text{Ber}(p)^{\otimes n}\) and independent observations \(X_k \sim \text{Ber}(p)\), \( k \in \{0, \ldots, n\}\). We are looking for a confidence interval \( C_{1-\alpha} \) around \( \hat{p} = \overline{X}_n = \frac{1}{n} \sum_{k=1}^n X_k \), i.e.,
\[
C_{1-\alpha} = [\overline{X}_n - a, \overline{X}_n + b] \quad \left(C_{1-\alpha}(x) = [\overline{X}_n(x) - a(x), \overline{X}_n(x) + b(x)] \right)
\]
should satisfy (where \(a\) and \(b\) might be random) that
\[
1 - \alpha \leq \mathbb{P}_p \left( p \in C_{1-\alpha} \right) = \mathbb{P}_p \left( \overline{X}_n - a \leq p \leq \overline{X}_n + b \right) = \mathbb{P}_p \left(-b \leq \overline{X}_n - p \leq a \right).
\]

Let \(t \mapsto F_p^n(t) := \mathbb{P}_p \left( \overline{X}_n - p \leq t \right)\) be the distribution function. Then,
\[
\mathbb{P}_p \left(-b \leq \overline{X}_n - p \leq a \right) = \mathbb{P}_p \left( \overline{X}_n - p \leq a \right) - \mathbb{P}_p \left( \overline{X}_n - p < -b \right)
\]
\[
= F_p^n(a) - F_p^n(-b) + R_n,
\]
where \( R_n = \mathbb{P}_p \left( \overline{X}_n - p = -b \right) \). Choose \(a, b\) as quantiles of \(\mathbb{P}_p^n\), i.e., \(a = \left(F_p^n\right)^{-1}(1 - \alpha / 2)\) and \( -b = \left(F_p^n\right)^{-1}(\alpha / 2)\) (with quantile function \(t \mapsto \left(F_p^n\right)^{-1}(t) := \inf \{t \in \mathbb{R} : F_p^n(t) \geq t \}\)).

However, \(F_p^n\) and thus \(a, b\) are unknown. Consider two possibilities:

\paragraph{Normal Approximation.} It holds that \( \mathbb{E}_p^n[X_k] = p \), \( \sigma := \text{Var}_p^n(X_k) = p(1 - p) \). By the central limit theorem, we have
\[
\frac{\sqrt{n}}{\sigma} \left( \overline{X}_n - p \right) = \frac{1}{\sqrt{n}} \sum_{k=1}^n \frac{X_k - p}{\sigma} \xrightarrow{d} N(0,1), \quad n \to \infty.
\]

For \( Z \sim N(0,1) \), it holds that
\[
F_p^n(a) = \mathbb{P}_p^n \left( \overline{X}_n - p \leq a \right) = \mathbb{P}_p^n \left( \frac{\sqrt{n}}{\sigma} \left( \overline{X}_n - p \right) \leq \frac{\sqrt{n}}{\sigma} a \right) \approx \mathbb{P}(|Z| \leq \frac{\sqrt{n}}{\sigma} a)
\]
\[
= \Phi \left( \frac{\sqrt{n}}{\sigma} a \right) = \Phi(z_\beta)
\]
for \( a := \frac{\sigma}{\sqrt{n}} z_\beta \) (where \( z_\beta \) is the \( \beta \)-quantile of the \( N(0,1) \)-distribution, i.e., \( \Phi(z_\beta) = \beta \)). In particular, \( R_n = o(1) \) (i.e., \( R_n \to 0 \) as \( n \to \infty \)). For \( a = b \) (since the \( N(0,1) \)-distribution is symmetric) and because \( \Phi(-x) = 1 - \Phi(x) \), it follows that
\[
\mathbb{P}_p^n(p \in C_{1-\alpha}) = F_p^n(a) - F_p^n(-a) + R_n \approx \Phi(z_\beta) - (1 - \Phi(z_\beta)) + o(1) = 2 \Phi(z_\beta) - 1 + o(1).
\]

For \( \beta = 1 - \alpha/2 \), \( C_{1-\alpha} \) is an asymptotically correct confidence interval. However, \( p \) and therefore \( \sigma \) and \( a \) are unknown. Solutions:

\begin{itemize}
    \item Estimate \( \sigma = p(1 - p) \leq 1/4 \) to widen the confidence interval.
    
    \item For the empirical variance \( \hat{\sigma}^2 = \frac{1}{n} \sum_{k=1}^n (X_k - \overline{X}_n)^2 \), we have \( \hat{\sigma}^2 \to \sigma^2 \) almost surely (by the law of large numbers). Using Slutsky’s lemma (Lemma: For random variables \( (X_n, Y_n)_{n \geq 1} \) with \( X_n \xrightarrow{d} X \), \( Y_n \xrightarrow{p} c \in \mathbb{R} \) (where \( c \) is deterministic), it holds that \( X_n + Y_n \xrightarrow{d} X + c \) and \( X_n \cdot Y_n \xrightarrow{d} c \cdot X \)), it follows that
    \[
    \frac{\sqrt{n}}{\hat{\sigma}} \left( \overline{X}_n - p \right) = \frac{\sigma}{\hat{\sigma}} \cdot \frac{\sqrt{n}}{\sigma} \left( \overline{X}_n - p \right) \xrightarrow{d} N(0,1), \quad n \to \infty.
    \]
\end{itemize}

From this, we derive \( a = \frac{\hat{\sigma}}{\sqrt{n}} z_{1 - \alpha / 2} \) (randomly chosen).

\[
\mathbb{P}_p^n(p \in C_{1 - \alpha}) = \mathbb{P}_p^n \left( |\overline{X}_n - p| \leq a \right) = \mathbb{P}_p^n \left( \left| \frac{\sqrt{n}}{\hat{\sigma}} (\overline{X}_n - p) \right| \leq z_{1 - \alpha / 2} \right)
\]
\[
\approx \mathbb{P}(|Z| \leq z_{1 - \alpha / 2}) = 2 \Phi(z_{1 - \alpha / 2}) - 1 = 1 - \alpha.
\]
\end{example}

\section{Hypothesis Testing}

\subsection{Basic Definitions}
Let \( (P_\theta : \theta \in \Theta) \) be a statistical model, and let \(\Theta = \Theta_0 \cup \Theta_1\) be a partition:
\begin{itemize}
    \item A \textbf{statistical test} is a measurable function of the data \( \varphi : (\mathcal{X}, \mathcal{F}) \to [0,1] \).
    \item If \( \varphi(x) \in \{0,1\} \) for all \( x \in \mathcal{X} \), then \( \varphi \) is a \textbf{non-randomized test}; otherwise, it is \textbf{randomized}.
    \item \( H_0 : \theta \in \Theta_0 \) is the \textbf{null hypothesis}.
    \item \( H_1 : \theta \in \Theta_1 \) is the \textbf{alternative hypothesis}.
    \item The map \( \theta \to \beta_\varphi (\theta) = P_\theta (\varphi = 1) \) is called the \textbf{power function} of a test \( \varphi \).
\end{itemize}

\subsection{Type I and Type II Errors}
\begin{itemize}
    \item For \(\theta \in \Theta_0\), \( \beta_\varphi(\theta) \) represents the \textbf{Type I error} (wrongly rejecting the null).
    \item For \(\theta \in \Theta_1\), \(1 - \beta_\varphi(\theta)\) represents the \textbf{Type II error} (failing to reject the alternative when it is true).
\end{itemize}

\[
1 \quad \quad \beta_\varphi(\theta) \quad 0 \quad \quad \Theta_0 \quad \quad \Theta_1 \quad \quad \Theta
\]

\textbf{Note:}
\[
1 - P_\theta (\varphi = 1) = P_\theta (\varphi = 0) = P_\theta \text{ (wrongly accepting the null)}
\]

\subsection{Level and Uniformly Most Powerful Tests}
\begin{definition}[Level]
A test \( \varphi : \mathcal{X} \to [0,1] \) has \textbf{level} \( \alpha \in [0,1] \) if
\[
\sup_{\theta \in \Theta_0} \beta_\varphi (\theta) \leq \alpha.
\]
\end{definition}

\begin{definition}[Uniformly Most Powerful Test]
Given a level \( \alpha \in (0,1) \), \( \varphi : \mathcal{X} \to [0,1] \) is called \textbf{uniformly most powerful (UMP)} if, for every other test \( \varphi' \) of level \( \alpha \) and all \( \theta \in \Theta_1 \),
\[
\beta_\varphi (\theta) \geq \beta_{\varphi'}(\theta).
\]
\end{definition}

\subsection{The Neyman-Pearson Lemma}
The Neyman-Pearson Lemma provides a basis for constructing the most powerful tests for simple hypotheses:
\begin{theorem}[Neyman-Pearson Lemma]
Let \( \Theta_0 = \{\theta_0\}\) and \( \Theta_1 = \{\theta_1\}\) be simple hypotheses:
\begin{enumerate}
    \item \textbf{Existence:} There exists a test \( \varphi \) and a constant \( k \in [0, \infty) \) such that \( P_{\theta_0} (\varphi = 1) = \alpha \), with
    \[
    \varphi(x) = 
    \begin{cases}
      1, & \text{if } \frac{p_{\theta_1}(x)}{p_{\theta_0}(x)} > k \\
      0, & \text{if } \frac{p_{\theta_1}(x)}{p_{\theta_0}(x)} < k
    \end{cases}
    \]
    Here, \( p_{\theta_1} \) and \( p_{\theta_0} \) are densities with respect to some dominated measure \( \mu \).
    
    \item \textbf{Sufficiency:} If \( \varphi \) satisfies \( P_{\theta_0} (\varphi = 1) = \alpha \) and the above form, then \( \varphi \) is a UMP level \( \alpha \) test.
    
    \item \textbf{Necessity:} If \( \varphi_k \) is UMP for level \( \alpha \), then it must be of the form shown above.
\end{enumerate}
\end{theorem}

\subsection{Proof of the Neyman-Pearson Lemma}
\begin{enumerate}
    \item Define the likelihood ratio \( r(x) = \frac{p_{\theta_1}(x)}{p_{\theta_0}(x)} \in [0, \infty) \). Let \( F_0 \) be the CDF of \( r(x) \) under \( P_{\theta_0} \).
    \[
    F_0(t) = P_{\theta_0} (r(x) \leq t).
    \]
    Define \( \alpha(t) = 1 - F_0(t) = P_{\theta_0} (r(x) > t) \) and note:
    \begin{itemize}
        \item \( \alpha \) is right-continuous:
        \[
        \lim_{\epsilon \to 0} \alpha(t + \epsilon) = P_{\theta_0} (r(x) > t).
        \]
        \item \( \alpha \) is non-increasing.
        \item \( \alpha \) has left limits.
    \end{itemize}

    \textbf{\( \alpha \) is cadlag:} It is continuous from the right and has a left limit.

    \[
    \text{There exists } k \in [0, \infty) \text{ such that }
    \alpha \leq \alpha(k^-) \quad \text{and} \quad \alpha \geq \alpha(k).
    \]

    We define the test
    \[
    \varphi(x) = 
    \begin{cases} 
      1 & \text{if } r(x) > k, \\
      \gamma & \text{if } r(x) = k, \\
      0 & \text{if } r(x) < k.
    \end{cases}
    \]
    
    Set \(\gamma = \frac{\alpha - \alpha(k)}{\alpha(k^-) - \alpha(k)}\).

    The level of \( \varphi \) is
    \[
    E_{\theta_0} [\varphi(x)] = P_{\theta_0} (\varphi(x) = 1).
    \]
    \[
    = P_{\theta_0} (r(x) > k) + P_{\theta_0} (r(x) = k) \cdot \gamma = \alpha.
    \]
\end{enumerate}
\chapter{Lecture 6: Neyman-Pearson Lemma and Likelihood Ratio Tests}
\section*{Neyman-Pearson Lemma}

\subsection*{Power of a Test}
The \textbf{power} of a test is defined as:
\[
E_{\theta_1}[\varphi] = P_{\theta_1}(\varphi = 1)
\]

\subsection*{Likelihood Ratio Test}
The \textbf{likelihood ratio} is given by:
\[
\Lambda(x) = \frac{p_{\theta_1}(x)}{p_{\theta_0}(x)} = r(x)
\]

\subsection*{Likelihood Ratio (LR) Test}
The LR test is defined as:
\[
\varphi(x) = 
\begin{cases} 
1 & \text{if } r(x) > k, \\
\gamma & \text{if } r(x) = k, \\
0 & \text{if } r(x) < k,
\end{cases}
\]
where \( k \in [0, \infty) \) and \( \gamma \in [0, 1] \).

\textbf{Note:} LR tests are Uniformly Most Powerful (UMP) for simple hypothesis testing:
\begin{itemize}
    \item Given a significance level \( \alpha \), if the LR test satisfies \( E_{\theta_0}[\varphi] = \alpha \), it controls the Type I error.
    \item The LR test minimizes the Type II error:
    \[
    E_{\theta_1}[\varphi] \geq E_{\theta_1}[\varphi'] \quad \forall \varphi'
    \]
\end{itemize}

\subsection*{Continuation of Proof (Part of UMP)}
Let \( \varphi' \) be another level \( \alpha \) test such that \( E_{\theta_0}[\varphi'] \leq \alpha \).

\textbf{Goal:} Show that \( E_{\theta_1}[\varphi] \geq E_{\theta_1}[\varphi'] \).

Let \( \mu \) be the dominating measure. Consider:
\[
\int (\varphi(x) - \varphi'(x)) \left( p_{\theta_1}(x) - k p_{\theta_0}(x) \right) d\mu(x) = 0
\]
\textbf{Claim:} \( p \geq 0 \).

\textbf{Observation:}
\begin{itemize}
    \item If \( p_{\theta_1}(x) - k p_{\theta_0}(x) > 0 \), then \( \frac{p_{\theta_1}(x)}{p_{\theta_0}(x)} > k \), implying \( \varphi(x) = 1 \).
    \item If \( p_{\theta_1}(x) - k p_{\theta_0}(x) < 0 \), then \( \varphi(x) = 0 \).
    \item If \( p_{\theta_1}(x) - k p_{\theta_0}(x) = 0 \), then the integrand is \( 0 \).
\end{itemize}
Thus, \( p = 0 \), leading to:
\[
\int (\varphi - \varphi') p_{\theta_1} \, d\mu = \int (\varphi - \varphi') p_{\theta_0} \, d\mu = k \left[ E_{\theta_0}[\varphi] - E_{\theta_0}[\varphi'] \right] \geq 0
\]
Therefore,
\[
E_{\theta_1}[\varphi] \geq E_{\theta_1}[\varphi']
\]

\subsection*{Part (3) UMP \(\Rightarrow\) LR}
Assume \( \varphi^* \) is a UMP test with \( E_{\theta_0}[\varphi^*] = \alpha \). Let \( \varphi \) be the LR test satisfying \( E_{\theta_0}[\varphi] = \alpha \).

\textbf{Goal:} Show that \( \varphi = \varphi^* \) almost everywhere except on \( \{ r(x) = k \} \).

Define the sets:
\[
x^+ = \{ x : \varphi(x) > \varphi^*(x) \}, \quad 
x^- = \{ x : \varphi(x) < \varphi^*(x) \}, \quad
x^0 = \{ x : \varphi(x) = \varphi^*(x) \}
\]
\[
\tilde{x} = (x^+ \cup x^-) \cap \{ x : p_{\theta_1}(x) \neq k p_{\theta_0}(x) \}
\]
It suffices to show \( \mu(\tilde{x}) = 0 \).

On \( \tilde{x} \):
\[
(\varphi - \varphi^*) (p_{\theta_1} - k p_{\theta_0}) > 0
\]
If \( \mu(\tilde{x}) > 0 \), then:
\[
\int_{\mathcal{X}} (\varphi - \varphi^*)(p_{\theta_1} - k p_{\theta_0}) \, d\mu \geq 0
\]
\[
\int_{\tilde{x}} (\varphi - \varphi^*)(p_{\theta_1} - k p_{\theta_0}) \, d\mu \geq 0
\]
However,
\[
E_{\theta_1}[\varphi] - E_{\theta_1}[\varphi^*] > k \left[ E_{\theta_0}[\varphi] - E_{\theta_0}[\varphi^*] \right] \geq 0
\]
This leads to a contradiction, implying \( \mu(\tilde{x}) = 0 \) and thus \( \varphi = \varphi^* \) almost everywhere.

\subsection*{Example: Gaussian Location Model}
Consider:
\[
X_1, \dots, X_n \overset{\text{iid}}{\sim} \mathcal{N}(\mu, \sigma^2)
\]
Testing:
\[
H_0 : \mu = \mu_0 \quad \text{vs.} \quad H_1 : \mu = \mu_1, \quad \mu_0 < \mu_1
\]
The likelihood ratio is:
\[
\frac{p_1(X_1, \dots, X_n)}{p_0(X_1, \dots, X_n)} = \exp \left( -\frac{1}{2\sigma^2} \sum_{i=1}^n (X_i - \mu_1)^2 + \frac{1}{2\sigma^2} \sum_{i=1}^n (X_i - \mu_0)^2 \right)
\]
Simplifying:
\[
= \exp \left( -\frac{n}{2\sigma^2} (\mu_1^2 - \mu_0^2) - \frac{2(\mu_1 - \mu_0)}{\sigma^2} \sum_{i=1}^n X_i \right) \geq K_\alpha
\]
This implies:
\[
\frac{1}{n} \sum_{i=1}^n X_i \geq K_\alpha, \quad \text{for some } K_\alpha \in \mathbb{R}
\]
To determine \( K_\alpha \):
\[
\bar{X}_n := \frac{1}{n} \sum X_i \overset{H_0}{\sim} \mathcal{N}\left(\mu_0, \frac{\sigma^2}{n}\right)
\]
Thus:
\[
P_{H_0} \left( \bar{X}_n \geq K_\alpha \right) = 1 - \Phi \left( \frac{\sqrt{n}}{\sigma} (K_\alpha - \mu_0) \right)
\]
Solving for \( K_\alpha \):
\[
K_\alpha = \mu_0 + \frac{\sigma}{\sqrt{n}} \Phi^{-1}(1 - \alpha)
\]
Therefore, the LR test is:
\[
\varphi(X_1, \dots, X_n) = 
\begin{cases} 
1 & \text{if } \bar{X}_n \geq \mu_0 + \frac{\sigma}{\sqrt{n}} \Phi^{-1}(1 - \alpha), \\
0 & \text{otherwise}.
\end{cases}
\]

\section*{Corollary}

Consider simple hypothesis testing. Let \( \varphi \) be a UMP test at level \( \alpha \). Then:
\[
\alpha = E_{H_0}[\varphi] \leq E_{\theta_1}[\varphi]
\]
Suppose \( E_{\theta_1}[\varphi] = E_{\theta_1}[\varphi_0] \). Then \( \varphi_0 \) is also UMP, implying \( \varphi_0 \) is an LR test:
\[
\varphi_0 = 
\begin{cases} 
1 & \text{if } \frac{p_{\theta_1}}{p_{\theta_0}} \geq K \quad \text{a.s., for some } K, \\
0 & \text{otherwise}.
\end{cases}
\]
Since \( \varphi_0 \in \{\varphi, \beta\} \), it follows that \( p_{\theta_1} = K p_{\theta_0} \) almost surely.

Moreover:
\[
\int p_{\theta_0} \, d\mu = K \int p_{\theta_0} \, d\mu = 1 \quad \Rightarrow \quad K = 1
\]

\subsection*{Correspondence Theorem}

\textbf{Statement:} There is a correspondence between tests and confidence regions.

\[
\text{Tests} \quad \longleftrightarrow \quad \text{Confidence regions } C(x)
\]
with
\[
\Pr_{\theta}(\theta \in C(x)) \geq 1 - \alpha
\]
and
\[
\Pr_{\theta}(\phi_{\theta}(x) = 1) = \alpha
\]

\textbf{Theorem:} Let \( \{P_\theta : \theta \in \Theta\} \) be a statistical model and \( \alpha \in (0, 1) \).

\begin{enumerate}
    \item[(i)] If \( C = C(X) \) is a level-\(\alpha\) confidence set, then
    \[
    \phi_{\theta_0}(x) = \mathbb{I}\left\{ \theta_0 \notin C(x) \right\}
    \]
    is a level-\(\alpha\) test for \( H_0: \theta = \theta_0 \) vs. \( H_1: \theta \neq \theta_0 \).
    
    \item[(ii)] If \( \{\phi_{\theta} : \theta \in \Theta \} \) is a family of level-\(\alpha\) tests, then
    \[
    C(X) = \left\{ \theta \in \Theta : \phi_{\theta}(X) = 0 \right\}
    \]
    is a \( 1 - \alpha \) confidence set.
\end{enumerate}

\textbf{Proof:}
\begin{enumerate}
    \item[(i)] 
    \[
    \Pr_{\theta_0}(\phi_{\theta_0} = 1) = \Pr_{\theta_0}(\theta_0 \notin C(X)) \leq \alpha
    \]
    
    \item[(ii)] 
    \[
    \Pr_{\theta}(\theta \notin C(X)) = \Pr_{\theta}(\phi_{\theta}(X) = 1) \leq \alpha
    \]
\end{enumerate}

\section*{UMPT Tests in Models with Monotone Likelihoods}

\textbf{Proposition:} Let \( \Theta \subseteq \mathbb{R} \). Consider testing:
\[
H_0 : \theta \leq \theta_0 \quad \text{vs.} \quad H_1 : \theta > \theta_0,
\]
for some \( \theta_0 \in \mathbb{R} \).

Assume there exists a test statistic \( T : X \to \mathbb{R} \) and a function \( h : \mathbb{R} \times \Theta \times \Theta \to \mathbb{R} \) such that:
\[
\frac{P_{\theta}(X)}{P_{\tilde{\theta}}(X)} = h(T(X), \theta, \tilde{\theta})
\]
and for all \( \theta \geq \tilde{\theta} \), the function \( t \mapsto h(t, \theta, \tilde{\theta}) \) is monotone increasing.

\textbf{Conclusion:} LR tests are also UMP for level \( \alpha \). Specifically, the LR test of \( H_0: \theta = \theta_0 \) against \( H_1: \theta = \theta_1 \) for any \( \theta_1 > \theta_0 \) will be UMP.

\chapter{Linear Models}

\section{Introduction to Linear Regression Models}

Linear regression models are fundamental tools in statistical analysis, enabling us to understand and quantify the relationship between a dependent variable and one or more independent variables. In this lecture, we will explore the simplest form of linear regression, discuss the statistical framework underpinning it, and delve into key estimation techniques and their optimality properties.

\section{Simple Linear Regression Model}

Consider the simplest scenario where we model the relationship between a dependent variable \( Y_i \) and an independent variable \( X_i \) using a linear relationship:

\begin{equation}
    Y_i = aX_i + b + \varepsilon_i
    \label{eq:simple_linear_model}
\end{equation}

for \( i = 1, \ldots, n \), where:
\begin{itemize}
    \item \( a \) and \( b \) are unknown parameters representing the slope and intercept, respectively.
    \item \( \varepsilon_i \) is the error term, assumed to be centered, i.e., \( E(\varepsilon_i) = 0 \), and having constant variance \( \operatorname{Var}(\varepsilon_i) = \sigma^2 \).
\end{itemize}

We further assume that the error terms are normally distributed, \( \varepsilon_i \sim N(0, \sigma^2) \), with \( \sigma \) known.

\subsection{Statistical Model}

The statistical model can be formally defined as:

\[
(\mathbb{R}, \mathcal{B}(\mathbb{R}), \left(\bigotimes_{i=1}^n N(aX_i + b, \sigma^2)\right)_{(a,b) \in \mathbb{R}^2})
\]

Here:
\begin{itemize}
    \item \( \mathbb{R} \) denotes the real line.
    \item \( \mathcal{B}(\mathbb{R}) \) is the Borel sigma-algebra on \( \mathbb{R} \).
    \item \( \bigotimes_{i=1}^n N(aX_i + b, \sigma^2) \) represents the product measure of the normal distributions for each observation.
\end{itemize}

\subsection{Likelihood Function}

Within this statistical model, the likelihood function is given by:

\[
L((a,b) \mid y) = \prod_{i=1}^n \frac{1}{\sqrt{2\pi \sigma^2}} \exp\left( -\frac{(y_i - aX_i - b)^2}{2\sigma^2} \right)
\]

Simplifying, we obtain:

\[
L((a,b) \mid y) = (2\pi \sigma^2)^{-n/2} \exp\left( -\frac{1}{2\sigma^2} \sum_{i=1}^n (y_i - aX_i - b)^2 \right)
\]

\subsection{Maximum Likelihood Estimation (MLE)}

The maximum likelihood estimators (MLE) for the parameters \( a \) and \( b \) are obtained by maximizing the likelihood function \( L((a,b) \mid y) \). Equivalently, since the logarithm is a monotonically increasing function, we can maximize the log-likelihood:

\[
\log L((a,b) \mid y) = -\frac{n}{2} \log(2\pi \sigma^2) - \frac{1}{2\sigma^2} \sum_{i=1}^n (y_i - aX_i - b)^2
\]

Maximizing the log-likelihood is equivalent to minimizing the sum of squared residuals:

\[
(\hat{a}, \hat{b}) = \arg \min_{(a,b) \in \mathbb{R}^2} \sum_{i=1}^n (y_i - aX_i - b)^2
\]

Provided that \( X_i \neq X_j \) for \( i \neq j \), the least squares problem has a unique solution, historically attributed to Gauss (1801), given by:

\[
(\hat{a}, \hat{b}) = \left( \frac{\frac{1}{n-1} \sum_{i=1}^{n} (X_i - \bar{X})(Y_i - \bar{Y})}{\frac{1}{n-1} \sum_{i=1}^{n} (X_i - \bar{X})^2}, \, \bar{Y} - \hat{a} \bar{X} \right)
\]

where \( \bar{X} \) and \( \bar{Y} \) denote the sample means of \( X \) and \( Y \), respectively.

\section{General Linear Models}

To generalize the simple linear regression model, we introduce the framework of linear models in multiple dimensions.

\begin{definition}[Linear Model]
    A random vector \( Y = (Y_1, \ldots, Y_n)^T \in \mathbb{R}^n \) stems from a linear model if there exists a parameter vector \( \beta \in \mathbb{R}^p \), a matrix \( X \in \mathbb{R}^{n \times p} \), and a random vector \( \varepsilon \in \mathbb{R}^n \) such that
    \[
    Y = X \beta + \varepsilon
    \]
    \begin{enumerate}
        \item \textbf{Regular Linear Model}: A linear model is called \emph{regular} if
        \begin{enumerate}
            \item The number of parameters does not exceed the sample size, i.e., \( p \le n \).
            \item The design matrix \( X \) has full rank, \( \operatorname{rank}(X) = p \le n \), ensuring a unique solution.
            \item The error vector satisfies \( E(\varepsilon) = 0 \), meaning the noise is centered.
            \item The covariance matrix of the errors is positive definite, \( \Sigma = \operatorname{Cov}(\varepsilon_i, \varepsilon_j)_{i,j \in [n]} \).
        \end{enumerate}
        \item \textbf{Ordinary Linear Model}: A linear model is called \emph{ordinary} if \( \Sigma = \sigma^2 I_n \), where \( I_n \) is the \( n \times n \) identity matrix. Typically, the noise is assumed to be Gaussian in this case.
    \end{enumerate}
\end{definition}

\begin{remark}
    \begin{enumerate}
        \item \textbf{Terminology:} Various synonyms are used in the literature:
        \begin{itemize}
            \item \( Y \): Dependent variable, response, regressand.
            \item \( X \): Independent variable, predictor, design matrix, regressor.
            \item \( \varepsilon \): Error, perturbation, regression function.
        \end{itemize}
        \item \textbf{Covariance Matrix \( \Sigma \):} The matrix \( \Sigma \) is symmetric and diagonalizable, i.e., \( \Sigma = UDU^T \) for some orthogonal matrix \( U \) and diagonal matrix \( D = \text{diag}(\lambda_1, \ldots, \lambda_n) \in \mathbb{R}^{n \times n} \).
        \item \textbf{Positive Semi-definiteness:} The covariance matrix \( \Sigma \) is positive semi-definite, meaning \( \lambda_i \ge 0 \) for all \( i \), which can be shown as:
        \[
        \langle \Sigma u, u \rangle = \langle E[(\varepsilon - E[\varepsilon])(\varepsilon - E[\varepsilon])^T]u, u \rangle = E[(\varepsilon - E[\varepsilon])^2] \ge 0, \quad \forall u \in \mathbb{R}^n
        \]
        \item \textbf{Positive Definiteness and Inverses:} If \( \Sigma \) is positive definite (i.e., \( \lambda_i > 0 \) for all \( i \)), then its inverse \( \Sigma^{-1} \) and square root \( \Sigma^{-1/2} \) exist and can be expressed as:
        \[
        \Sigma^{-1} = UD^{-1}U^T \quad \text{and} \quad \Sigma^{-1/2} = UD^{-1/2}U^T
        \]
        \item \textbf{Random Design:} If the matrix \( X \) is not deterministic but random, the model is referred to as having a \emph{random design}.
    \end{enumerate}
\end{remark}

\section{Least Squares Estimation}

In the context of a regular linear model, the least squares estimator (LSE) seeks to minimize the weighted sum of squared residuals. Specifically, the LSE \( \hat{\beta} \) satisfies:

\[
\|\sigma^{-1/2}(Y - X\hat{\beta})\|^2 = \inf_{\beta \in \mathbb{R}^p} \|\sigma^{-1/2}(Y - X\beta)\|^2 = \inf_{\beta \in \mathbb{R}^p} \|\Sigma^{-1/2}Y - X_\Sigma \beta\|^2
\]

where \( X_\Sigma = \Sigma^{-1/2} X \).

\subsection{Geometric Interpretation}

The estimator \( X_\Sigma \hat{\beta} \) represents the point within the subspace:

\[
U = \{ X_\Sigma \beta \mid \beta \in \mathbb{R}^p \} \subseteq \mathbb{R}^n
\]

that is closest to the vector \( \Sigma^{-1/2} Y \) in terms of Euclidean distance. Formally, this can be expressed using the orthogonal projection \( \Pi_U \) onto \( U \):

\[
X_\Sigma \hat{\beta} = \Pi_U(\Sigma^{-1/2} Y)
\]

The orthogonal projection satisfies:
\begin{itemize}
    \item \( \Pi_U u = u \) for all \( u \in U \).
    \item \( \langle \Pi_U v - v, u \rangle = 0 \) for all \( u \in U \) and \( v \in \mathbb{R}^n \).
\end{itemize}

Provided that \( (X_\Sigma^T X_\Sigma)^{-1} \) exists, we can confirm by direct computation that the projection operator \( \Pi_U \) is given by:

\[
\Pi_U = X_\Sigma (X_\Sigma^T X_\Sigma)^{-1} X_\Sigma^T
\]

For any \( u = X_\Sigma \beta \), we have:

\[
X_\Sigma (X_\Sigma^T X_\Sigma)^{-1} X_\Sigma^T X_\Sigma \beta = X_\Sigma \beta = u
\]

By symmetry, for any \( u \in U \) and \( v \in \mathbb{R}^n \):

\[
\langle \Pi_U v - v, u \rangle = \langle v, \Pi_U u \rangle - \langle v, u \rangle = \langle v, u \rangle - \langle v, u \rangle = 0
\]

\section{Representation for the Least Squares Estimator}

\begin{lemma}[Representation for the LSE]
    Consider a regular linear model. Then the least squares estimator (LSE) exists uniquely and is given by:
    \[
    \hat{\beta} = (X_\Sigma^T X_\Sigma)^{-1} X_\Sigma^T \Sigma^{-1/2} Y = X_\Sigma^+ \Sigma^{-1/2} Y
    \]
    where \( X_\Sigma^+ \) denotes the Moore-Penrose pseudoinverse of \( X_\Sigma \).
\end{lemma}

\begin{proof}
    Since \( X \) has full rank and \( p \le n \), the matrix \( X_\Sigma^T X_\Sigma \) is invertible. Suppose \( v \in \ker(X_\Sigma^T X_\Sigma) \), then:

    \[
    0 = v^T X_\Sigma^T X_\Sigma v = (X_\Sigma v)^T (X_\Sigma v) = \| X_\Sigma v \|^2 = \| \Sigma^{-1/2} X v \|^2
    \]

    This implies \( \| X v \|^2 = 0 \), hence \( X v = 0 \). Given that \( X \) has full rank, the only solution is \( v = 0 \). Therefore, \( X_\Sigma^T X_\Sigma \) is invertible.

    The projection of \( \Sigma^{-1/2} Y \) onto \( U \) is:

    \[
    X_\Sigma \hat{\beta} = \Pi_U(\Sigma^{-1/2} Y) = X_\Sigma (X_\Sigma^T X_\Sigma)^{-1} X_\Sigma^T \Sigma^{-1/2} Y
    \]

    Multiplying both sides by \( X_\Sigma^T \):

    \[
    X_\Sigma^T X_\Sigma \hat{\beta} = X_\Sigma^T X_\Sigma (X_\Sigma^T X_\Sigma)^{-1} X_\Sigma^T \Sigma^{-1/2} Y = X_\Sigma^T \Sigma^{-1/2} Y
    \]

    Hence, solving for \( \hat{\beta} \):

    \[
    \hat{\beta} = (X_\Sigma^T X_\Sigma)^{-1} X_\Sigma^T \Sigma^{-1/2} Y
    \]
\end{proof}

\begin{remark}
    \begin{enumerate}
        \item If \( p > n \), the matrix \( X_\Sigma^T X_\Sigma \) is not invertible, and the LSE is not unique. In this case, the set of solutions is a \( (p - n) \)-dimensional subspace, and each solution interpolates the data perfectly, i.e.,
        \[
        \{ \beta \in \mathbb{R}^p \mid \| \Sigma^{-1/2} Y - X_\Sigma \beta \|^2 = 0 \}
        \]
    \end{enumerate}
\end{remark}

\section{Optimality of the Least Squares Estimator}

The least squares estimator possesses several optimality properties under the ordinary linear model. This is formalized in the Gauss-Markov Theorem.

\begin{theorem}[Gauss-Markov Theorem]
    Consider an ordinary linear model with \( \sigma > 0 \). Then:
    \begin{enumerate}
        \item The least squares estimator \( \hat{\beta} = (X^T X)^{-1} X^T Y \) is a linear and unbiased estimator for the parameter \( \beta \).
        \item For any desired linear combination of parameters \( \alpha = \langle \beta, v \rangle \) where \( v \in \mathbb{R}^p \), the estimator \( \hat{\alpha} = \langle \hat{\beta}, v \rangle \) is the best linear unbiased estimator (BLUE) of \( \alpha \). This means that \( \hat{\alpha} \) has the smallest variance among all linear unbiased estimators of \( \alpha \).
        \item The estimator \( \hat{\sigma}^2 = \frac{\| Y - X \hat{\beta} \|^2}{n - p} \) is an unbiased estimator of \( \sigma^2 \).
    \end{enumerate}
\end{theorem}

\begin{proof}
    \begin{enumerate}
        \item \textbf{Linearity and Unbiasedness of \( \hat{\beta} \):}
        
        The estimator \( \hat{\beta} \) is linear because it can be expressed as a linear transformation of \( Y \):
        \[
        \hat{\beta} = (X^T X)^{-1} X^T Y
        \]
        For any two vectors \( y \) and \( \tilde{y} \) in \( \mathbb{R}^n \),
        \[
        \hat{\beta}(y + \tilde{y}) = (X^T X)^{-1} X^T (y + \tilde{y}) = \hat{\beta}(y) + \hat{\beta}(\tilde{y})
        \]
        demonstrating linearity.

        To show unbiasedness, compute the expectation of \( \hat{\beta} \):
        \begin{align*}
            E[\hat{\beta}] &= (X^T X)^{-1} X^T E[Y] \\
            &= (X^T X)^{-1} X^T E[X \beta + \varepsilon] \\
            &= (X^T X)^{-1} X^T (X \beta + E[\varepsilon]) \\
            &= (X^T X)^{-1} X^T X \beta \quad \text{(since \( E[\varepsilon] = 0 \))} \\
            &= \beta
        \end{align*}
        Hence, \( \hat{\beta} \) is unbiased.

        \item \textbf{Optimality of \( \hat{\alpha} \) as BLUE:}

        Consider a linear unbiased estimator \( \tilde{\alpha} \) for \( \alpha = \langle \beta, v \rangle \). Since \( \tilde{\alpha} \) is linear, there exists a vector \( w \in \mathbb{R}^n \) such that:
        \[
        \tilde{\alpha} = \langle Y, w \rangle
        \]
        For \( \tilde{\alpha} \) to be unbiased, we require:
        \[
        E[\tilde{\alpha}] = \langle E[Y], w \rangle = \langle X \beta, w \rangle = \langle \beta, X^T w \rangle = \alpha = \langle \beta, v \rangle
        \]
        This implies that:
        \[
        v = X^T w
        \]
        The variance of \( \tilde{\alpha} \) is:
        \[
        \operatorname{Var}(\tilde{\alpha}) = \operatorname{Var}(\langle Y, w \rangle) = \operatorname{Var}(\langle X \beta + \varepsilon, w \rangle) = \operatorname{Var}(\langle \varepsilon, w \rangle) = \sigma^2 \| w \|^2
        \]
        Now, consider the variance of \( \hat{\alpha} = \langle \hat{\beta}, v \rangle \):
        \begin{align*}
            \operatorname{Var}(\hat{\alpha}) &= \operatorname{Var}(\langle \hat{\beta} - \beta, v \rangle) \\
            &= \operatorname{Var}\left( \langle (X^T X)^{-1} X^T \varepsilon, v \rangle \right) \\
            &= \operatorname{Var}\left( \langle \varepsilon, X (X^T X)^{-1} v \rangle \right) \\
            &= \sigma^2 \| X (X^T X)^{-1} v \|^2 \\
            &= \sigma^2 \| X (X^T X)^{-1} X^T w \|^2 \quad \text{(since \( v = X^T w \))} \\
            &= \sigma^2 \| \Pi_U w \|^2 \\
            &\le \sigma^2 \| w \|^2 \quad \text{(since projection does not increase the norm)}
        \end{align*}
        Therefore, \( \operatorname{Var}(\hat{\alpha}) \le \operatorname{Var}(\tilde{\alpha}) \), showing that \( \hat{\alpha} \) has the smallest variance among all linear unbiased estimators of \( \alpha \).

        \item \textbf{Unbiasedness of \( \hat{\sigma}^2 \):}

        The estimator \( \hat{\sigma}^2 \) is given by:
        \[
        \hat{\sigma}^2 = \frac{\| Y - X \hat{\beta} \|^2}{n - p}
        \]
        To show that \( \hat{\sigma}^2 \) is unbiased, compute its expectation:
        \begin{align*}
            E[\hat{\sigma}^2] &= \frac{E[\| Y - X \hat{\beta} \|^2]}{n - p} \\
            &= \frac{E[\| \varepsilon \|^2 - \| X \hat{\beta} \|^2 + 2 \langle \varepsilon, X \hat{\beta} \rangle]}{n - p} \\
            &= \frac{E[\| \varepsilon \|^2]}{n - p} \quad \text{(since \( E[\langle \varepsilon, X \hat{\beta} \rangle] = 0 \))} \\
            &= \frac{n \sigma^2}{n - p} \quad \text{(since \( \| \varepsilon \|^2 \) is chi-squared with \( n \) degrees of freedom)} \\
            &= \sigma^2 \quad \text{(since \( E[\| \varepsilon \|^2] = n \sigma^2 \))}
        \end{align*}
        Hence, \( \hat{\sigma}^2 \) is an unbiased estimator of \( \sigma^2 \).
    \end{enumerate}
\end{proof}

\begin{remark}
    \begin{enumerate}
        \item The Gauss-Markov theorem establishes that under the ordinary linear model assumptions, the least squares estimator is the best (in the sense of having the smallest variance) among all linear unbiased estimators. This optimality holds without requiring the error terms to be normally distributed.
    \end{enumerate}
\end{remark}

\section{Conclusion}

In this lecture, we have introduced the fundamental concepts of linear regression models, both in their simplest form and in a more general framework. We discussed the estimation of model parameters using the method of least squares, explored the geometric interpretation of the estimator, and established its optimality through the Gauss-Markov theorem. Understanding these foundational elements is crucial for further studies in statistical modeling and inference.


\chapter{Lecture 9: Ridge Regression and Its Properties}

In this lecture, we delve into Ridge Regression, a technique used to address multicollinearity in linear models. We will explore its derivation, properties, and how it compares to Ordinary Least Squares (OLS). Additionally, we will discuss the bias-variance trade-off inherent in Ridge Regression and present propositions regarding its Mean Squared Error (MSE).

\subsection*{Proof of Ridge Regression Estimator}

We begin by deriving the Ridge Regression estimator through optimization of the penalized loss function.

\textbf{Objective:} Minimize the cost function 
\[
\mathcal{J}(\bm{\beta}) = \|\bm{Y} - \bm{X} \bm{\beta}\|^2 + \frac{\sigma^2}{\tau^2} \|\bm{\beta}\|^2
\]
with respect to the coefficient vector \(\bm{\beta}\).

\textbf{Steps:}

1. **Compute the Gradient:**

   To find the minimum, we take the gradient of \(\mathcal{J}(\bm{\beta})\) with respect to \(\bm{\beta}\):
   \[
   \nabla_{\bm{\beta}} \mathcal{J}(\bm{\beta}) = 2 \bm{X}^\top (\bm{Y} - \bm{X} \bm{\beta}) + \frac{2\sigma^2}{\tau^2} \bm{\beta}
   \]

2. **Set Gradient to Zero:**

   Setting the gradient equal to zero to find the critical point:
   \[
   2 \bm{X}^\top (\bm{Y} - \bm{X} \bm{\beta}) + \frac{2\sigma^2}{\tau^2} \bm{\beta} = 0
   \]
   Simplifying:
   \[
   (\bm{X}^\top \bm{X} + \frac{\sigma^2}{\tau^2} \bm{I}) \bm{\beta} = \bm{X}^\top \bm{Y}
   \]

3. **Solve for \(\bm{\beta}\):**

   Finally, we solve for \(\bm{\beta}\):
   \[
   \bm{\beta} = (\bm{X}^\top \bm{X} + \frac{\sigma^2}{\tau^2} \bm{I})^{-1} \bm{X}^\top \bm{Y}
   \]
   
   This is the Ridge Regression estimator, which introduces a penalty term \(\lambda = \frac{\sigma^2}{\tau^2}\) to shrink the coefficients.

\subsection*{Posterior Mean}

In a Bayesian framework, the posterior mean of \(\bm{\beta}\) given the data can be expressed as:
\[
\mu_{\text{post}} = \bm{\Sigma}_{\text{post}}^{-1} (\bm{X}^\top \bm{Y} + \bm{M}_0^{-1} \bm{\mu}_0)
\]
Upon substituting \(\bm{M}_0^{-1} = \frac{\sigma^2}{\tau^2} \bm{I}_p\), we obtain:
\[
\mu_{\text{post}} = (\sigma^{-2} \bm{X}^\top \bm{X} + \tau^{-2} \bm{I}_p)^{-1} \sigma^{-2} \bm{X}^\top \bm{Y}
\]
Simplifying further:
\[
\mu_{\text{post}} = (\bm{X}^\top \bm{X} + \frac{\sigma^2}{\tau^2} \bm{I})^{-1} \bm{X}^\top \bm{Y}
\]
This aligns with the Ridge Regression estimator derived earlier.

\subsection*{Remark on Ridge Regression's Existence}

An important property of Ridge Regression is that the estimator \(\bm{\beta}\) is always defined, even when the design matrix \(\bm{X}\) does not have full rank. Specifically, Ridge Regression provides a unique solution even in cases where \(n < p\) (i.e., when there are more predictors than observations), addressing issues of multicollinearity and overfitting inherent in OLS.

\subsection*{Definition of Ridge Regression Estimator}

\[
\hat{\bm{\beta}}_{\text{ridge}} = \operatorname*{argmin}_{\bm{\beta} \in \mathbb{R}^p} \|\bm{Y} - \bm{X} \bm{\beta} \|^2 + \lambda \|\bm{\beta}\|^2,
\]
is termed the \textbf{Ridge Regression} estimator. Here, \(\lambda > 0\) serves as the regularization parameter that controls the strength of the penalty on the size of the coefficients. A key feature of Ridge Regression is that \(\hat{\bm{\beta}}_{\text{ridge}}\) is always uniquely defined, making it a robust alternative to OLS, especially in high-dimensional settings.

When the model is specified as \(\bm{Y} = \bm{X} \bm{\beta} + \bm{\varepsilon}\), where \(\bm{\varepsilon} \sim N(0, \sigma^2 \bm{I}_n)\), the Ridge estimator takes the form:
\[
\hat{\bm{\beta}}_{\text{ridge}} = (\bm{X}^\top \bm{X} + \lambda \bm{I}_p)^{-1} \bm{X}^\top \bm{Y}.
\]
Notably, this estimator is independent of \(\sigma^2\), highlighting its focus on minimizing the penalized residual sum of squares without direct dependence on the noise variance.

\subsection*{Proposition: Mean Squared Error (MSE) of \(\hat{\bm{\beta}}_{\text{ridge}}\)}

We now analyze the MSE of the Ridge estimator under specific conditions.

\textbf{Assumptions:}
\begin{itemize}
    \item The linear model is given by \(\bm{Y} = \bm{X} \bm{\beta} + \bm{\varepsilon}\).
    \item The error term \(\bm{\varepsilon} \sim N(0, \sigma^2 \bm{I}_n)\), with \(\sigma^2 > 0\) known.
    \item The design matrix satisfies \(\bm{X}^\top \bm{X} = n \bm{I}_p\), indicating an orthonormal design.
\end{itemize}

\textbf{Definition:}
Let \(\mathcal{J} := \langle \bm{\beta}, \bm{v} \rangle\) for some \(\bm{v} \in \mathbb{R}^p\), and define:
\[
\delta_{\text{ridge}} = \langle \hat{\bm{\beta}}_{\text{ridge}}, \bm{v} \rangle.
\]

\textbf{Statements:}
\begin{enumerate}
    \item The expected squared error of \(\delta_{\text{ridge}}\) is:
    \[
    \mathbb{E}_{\bm{\beta}}[(\delta_{\text{ridge}} - \mathcal{J})^2] = (1 + \lambda)^{-2} \langle \bm{\beta}_v, \bm{v} \rangle^2 + \frac{\sigma^2}{n} \|\bm{v}\|^2 (1 + \lambda)^{-2}.
    \]
    
    \item The expected squared error of the Ridge estimator vector is:
    \[
    \mathbb{E}_{\bm{\beta}}[\|\hat{\bm{\beta}}_{\text{ridge}} - \bm{\beta}\|^2] = (1 + \lambda)^{-2} \|\bm{\beta}\|^2 + \frac{p \sigma^2}{n} \frac{1}{(1 + \lambda)^2}.
    \]
\end{enumerate}

\textbf{Derivation:}

Starting from the Ridge estimator:
\[
\hat{\bm{\beta}}_{\text{ridge}} = (\bm{X}^\top \bm{X} + \lambda \bm{I}_p)^{-1} \bm{X}^\top \bm{Y}.
\]
Given the orthonormal design \(\bm{X}^\top \bm{X} = n \bm{I}_p\), this simplifies to:
\[
\hat{\bm{\beta}}_{\text{ridge}} = \left(n \bm{I}_p + \lambda \bm{I}_p\right)^{-1} \bm{X}^\top \bm{Y} = \frac{1}{1 + \frac{\lambda}{n}} (\bm{X}^\top \bm{X} \bm{\beta} + \bm{X}^\top \bm{\varepsilon}).
\]
Simplifying further:
\[
\hat{\bm{\beta}}_{\text{ridge}} = \frac{1}{1 + \frac{\lambda}{n}} \bm{\beta} + \frac{1}{1 + \frac{\lambda}{n}} \bm{X}^\top \bm{\varepsilon}.
\]
This expression separates the estimator into a bias term and a variance term, which will be instrumental in understanding the bias-variance decomposition.

\subsection*{Bias-Variance Decomposition}

The Mean Squared Error (MSE) of an estimator can be decomposed into the sum of the squared bias and the variance of the estimator:
\[
\mathbb{E}\left[(\hat{\bm{\beta}}_{\text{ridge}} - \mathcal{J})^2\right] = (\mathbb{E}[\hat{\bm{\beta}}_{\text{ridge}}] - \mathcal{J})^2 + \text{Var}(\hat{\bm{\beta}}_{\text{ridge}}).
\]
Substituting from our Ridge estimator:
\[
= \left((1 + \frac{\lambda}{n})^{-1} \langle \bm{\beta}, \bm{v} \rangle\right)^2 + \frac{\lambda^2}{(1 + \lambda)^2} \text{Var}(\bm{X}^\top \bm{\varepsilon}, \nu).
\]
\textbf{Observations:}
\[
(1 + \frac{\lambda}{n})^{-1} = \frac{1}{(1 + \frac{\lambda}{n})}.
\]
Furthermore, the variance term can be expressed as:
\[
\text{Var}(\bm{X}^\top \bm{\varepsilon}, \bm{\nu}) = \bm{\nu}^\top \bm{X} \text{Cov}(\bm{\varepsilon}) \bm{X}^\top \bm{\nu} = \sigma^2 \|\bm{\nu}\|^2.
\]
Thus, the bias-variance decomposition clearly illustrates the trade-off controlled by the regularization parameter \(\lambda\).

\subsection*{Corollary: MSE Under Orthonormal Design}

Under the same assumptions, specifically the orthonormal design \(\bm{X}^\top \bm{X} = n \bm{I}_p\), the MSE of the Ridge estimator vector is:
\[
\mathbb{E}[\|\hat{\bm{\beta}}_{\text{ridge}} - \bm{\beta}\|^2] = \frac{1}{(1 + \frac{\lambda}{n})^2} \|\bm{\beta}\|^2 + \frac{p \sigma^2}{n (1 + \frac{\lambda}{n})^2}.
\]
This expression highlights how both the bias and variance components of the estimator's error are influenced by \(\lambda\).

\subsection*{Remark on Ridge Regression Behavior}

An insightful observation about Ridge Regression is its behavior in relation to the norm of \(\bm{\beta}\):
\[
\text{For small } \|\bm{\beta}\|, \text{ Ridge Regression converges to OLS.}
\]
This implies that the optimal choice of \(\lambda\) depends on the true underlying parameter \(\|\bm{\beta}\|\). When \(\|\bm{\beta}\|\) is small, a smaller \(\lambda\) is preferred to minimize bias, whereas larger \(\lambda\) values are beneficial when \(\|\bm{\beta}\|\) is large to control variance.

\subsection*{Confidence Sets \& Tests in Linear Model}

While Ridge and OLS estimators provide point estimates for \(\bm{\beta}\), it's crucial to quantify the uncertainty associated with these estimates. This involves constructing confidence sets and performing hypothesis tests, which inherently depend on the variance \(\sigma^2\).

\textbf{Assumption:} Throughout this discussion, we assume \(\bm{\varepsilon} \sim N(0, \sigma^2 \bm{I}_n)\), ensuring the errors are normally distributed with known variance.

\subsubsection*{Easy Case: Known \(\sigma^2\)}

When the variance \(\sigma^2\) is known, the distribution of the OLS estimator is straightforward:
\[
\hat{\bm{\beta}}_{\text{OLS}} \sim N(\bm{\beta}, \sigma^2 (\bm{X}^\top \bm{X})^{-1}).
\]
Given an orthonormal design (\(\bm{X}^\top \bm{X} = n \bm{I}_p\)), this simplifies to:
\[
\hat{\bm{\beta}}_{\text{OLS}} \sim N\left(\bm{\beta}, \frac{\sigma^2}{n} \bm{I}_p\right).
\]

For any linear combination \(\mathcal{J} = \langle \bm{\beta}, \bm{\nu} \rangle\), the estimator is:
\[
\hat{\mathcal{J}} = \langle \hat{\bm{\beta}}_{\text{OLS}}, \bm{\nu} \rangle \sim N\left(\mathcal{J}, \sigma^2 \bm{\nu}^\top (\bm{X}^\top \bm{X})^{-1} \bm{\nu}\right).
\]
This allows us to construct a 95\% confidence interval for \(\mathcal{J}\):
\[
I_{95\%}(\mathcal{J}) = \left[ \hat{\mathcal{J}} \pm 1.96 \sqrt{\bm{\nu}^\top (\bm{X}^\top \bm{X})^{-1} \bm{\nu}} \right].
\]

\subsection*{Notes on \textit{t}- and \textit{F}-distributions}

In practice, the variance \(\sigma^2\) is often unknown and must be estimated from the data. This necessitates the use of the \textit{t}-distribution and the \textit{F}-distribution for constructing confidence intervals and conducting hypothesis tests.

\subsubsection*{Definitions}

\textbf{t-distribution:}  
The \textit{t}-distribution with \(n \geq 1\) degrees of freedom on \(\mathbb{R}\) has the probability density function:
\[
f_n(x) = C_n \left( 1 + \frac{x^2}{n} \right)^{-\frac{n+1}{2}},
\]
where \(C_n\) is the normalizing constant ensuring that the total probability integrates to 1.

\textbf{Special Case:} For \(n = 1\), the \textit{t}-distribution simplifies to:
\[
f_1(x) = C_1 \frac{1}{1 + x^2},
\]
which corresponds to the \textbf{Cauchy distribution}, a distribution with heavy tails.

\textbf{F-distribution:}  
The \(F\)-distribution with \((m, n) \in \mathbb{N}^2\) degrees of freedom has the density:
\[
f_{m,n}(x) = C_{m,n} \frac{x^{\frac{m}{2} - 1}}{(m x + n)^{\frac{m+n}{2}}}, \quad x \in (0, \infty),
\]
where \(C_{m,n}\) is the normalizing constant. The \(F\)-distribution is pivotal in variance ratio tests and regression analysis.

\subsection*{Utility of \textit{t}- and \textit{F}-distributions}

These distributions are instrumental in constructing confidence intervals and performing hypothesis tests when dealing with normally distributed errors, especially when the variance is unknown and must be estimated from the data.

\begin{lemma}
Let \(X_1, \dots, X_n, Y_1, \dots, Y_n\) be independent and identically distributed (i.i.d.) \(N(0, \Delta)\) random variables. Then:

\begin{enumerate}
    \item 
    \[
    T_n := \frac{X_n}{\sqrt{\frac{1}{n} \sum_{i=1}^n Y_i^2}} \sim t_n.
    \]
    
    \item 
    \[
    F_{m,n} := \frac{\frac{1}{m} \sum_{i=1}^m X_i^2}{\frac{1}{n} \sum_{j=1}^n Y_j^2} \sim F_{m,n}.
    \]
\end{enumerate}
\end{lemma}

\subsection*{Remarks on the Lemma}

\begin{enumerate}
    \item 
    The \textit{t}-distribution emerges when evaluating the ratio of a normally distributed variable to the square root of a scaled chi-squared variable, reflecting the relationship between the sample mean and sample variance.
    
    \item 
    As the degrees of freedom \(n\) increase, the \textit{t}-distribution \(T_n\) converges in distribution to the standard normal distribution \(N(0, 1)\), illustrating the asymptotic behavior of the estimator.
\end{enumerate}

\subsection*{Proof of the Lemma}

\begin{proof}
\begin{enumerate}
    \item 
    To show that \(T_n\) follows a \textit{t}-distribution with \(n\) degrees of freedom, observe that:
    \[
    T^2_n = F_{1,n}.
    \]
    Considering the symmetry of the \textit{t}-distribution around 0, we utilize a change of variables:
    \[
    f_{F_{m,n}}(x) = f_{F_{m,n}}(x^2) \cdot 2x, \quad x > 0.
    \]
    Given that \(T_n\) is symmetric, we derive:
    \[
    f_{T_n}(x) = f_{F_{m,n}}(x^2) |x| = C_n \left( 1 + \frac{x^2}{n} \right)^{-\frac{n+1}{2}}.
    \]
    This aligns with the known density of the \textit{t}-distribution.

    \item 
    To establish the distribution of \(F_{m,n}\), define:
    \[
    X = \sum_{i=1}^m X_i^2, \quad Y = \sum_{j=1}^n Y_j^2.
    \]
    Since each \(X_i\) and \(Y_j\) is normally distributed, it follows that:
    \[
    X \sim \chi^2_m, \quad Y \sim \chi^2_n,
    \]
    where \(\chi^2_m\) and \(\chi^2_n\) denote chi-squared distributions with \(m\) and \(n\) degrees of freedom, respectively. The density of \(\chi^2_m\) is:
    \[
    f(x) \propto x^{m/2 - 1} e^{-x/2}, \quad x > 0.
    \]
\end{enumerate}
\end{proof}

\subsection*{Derivation of the \(F\)-distribution}

To derive the distribution of \(F_{m,n}\), consider the ratio:
\[
W = \frac{X}{Y}.
\]
The cumulative distribution function (CDF) of \(W\) is:
\[
\mathbb{P}\left(\frac{X}{Y} < z\right) = \int_0^\infty \int_0^{zy} 1 \cdot f_X(x) f_Y(y) \, dx \, dy.
\]
By substituting \(x = wy\), we obtain:
\[
= \int_0^\infty \int_0^z f_X(wy) f_Y(y) \cdot y \, dw \, dy.
\]
This simplifies to:
\[
= \int_0^\infty f_X(zy) f_Y(y) \cdot y \, dy.
\]
Substituting the chi-squared densities:
\[
\propto \int_0^\infty (zy)^{\frac{m}{2} - 1} y^{\frac{n}{2} - 1} e^{-(z+y)/2} \, dy.
\]
Introducing the change of variable \(a = \frac{z}{z+1}y\), the integral becomes:
\[
\propto z^{\frac{m}{2} - 1} (z+1)^{-\frac{m+n}{2}} \int_0^\infty a^{\frac{m}{2} - 1} e^{-a} \, da.
\]
Recognizing that the integral is the gamma function \(\Gamma\left(\frac{m}{2}\right)\), we conclude:
\[
f_F(z) \propto z^{\frac{m}{2} - 1} (z+1)^{-\frac{m+n}{2}}.
\]
Finally, by appropriate scaling, we arrive at the density function of the \(F\)-distribution:
\[
f_{m,n}(z) = \frac{m^{m/2} n^{n/2}}{B\left(\frac{m}{2}, \frac{n}{2}\right)} \frac{z^{m/2 - 1}}{(m z + n)^{(m+n)/2}},
\]
where \(B(\cdot, \cdot)\) is the beta function.

\chapter{Lecture 10: Confidence Sets and Hypothesis Testing in Linear Models}

Building upon the foundations laid in the previous lecture, we now focus on constructing confidence sets and performing hypothesis tests within the context of linear models. We will explore both \(t\)-tests and \(F\)-tests, essential tools for statistical inference in regression analysis.

\subsection*{Ridge and OLS Estimators}

Recall that in the linear model \(Y = X\beta + \varepsilon\), with \(\varepsilon \sim N(0, \sigma^2 I_n)\), the OLS estimator is given by:
\[
\hat{\beta} = (X^TX)^{-1}X^TY \sim N(\beta, \sigma^2 (X^TX)^{-1}).
\]
When \(\sigma^2\) is unknown, it is typically estimated by:
\[
\hat{\sigma}^2 = \frac{||Y - X\hat{\beta}_{OLS}||^2}{n - p} \sim \sigma^2 \frac{\chi^2(n - p)}{n - p}.
\]
This estimator is unbiased and follows a scaled chi-squared distribution.

\subsection*{Distributions of Estimators}

\[
t_n = \frac{N(0,1)}{\sqrt{\chi^2(n)/n}} \quad \text{and} \quad F_{m,n} = \frac{\chi^2(m)/m}{\chi^2(n)/n}
\]
These ratios form the basis of the \(t\)- and \(F\)-distributions, respectively.

\begin{lemma}
Let \(\xi \sim N(0, I_n)\), a random variable in \(\mathbb{R}^n\), and let \(R \in \mathbb{R}^{n \times n}\) be an orthogonal projection matrix (\(R = R^2\), \(R = R^T\)), with \(\text{rank}(R) = r \le n\). Then:
\begin{enumerate}
    \item \(\xi^T R \xi = ||R\xi||^2 \sim \chi^2(r).\)
    \item If \(B \in \mathbb{R}^{p \times n}\) is such that \(BR = 0\), then \(B\xi\) is independent of \(R\xi\).
    \item If \(S \in \mathbb{R}^{n \times n}\) is another orthogonal projection with \(\text{rank}(S) = s \le n\) and \(RS = 0\), then 
    \[
    \frac{s}{r} \frac{\xi^T R \xi}{\xi^T S \xi} \sim F(r, s).
    \]
\end{enumerate}
\end{lemma}

\begin{proof}
\begin{enumerate}
    \item 
    Since \(R\) is an orthogonal projection, there exists an orthogonal matrix \(T\) such that:
    \[
    R = T \begin{pmatrix} I_r & \mathbf{0} \\ \mathbf{0} & \mathbf{0} \end{pmatrix} T^T = T D_r T^T.
    \]
    Applying the orthogonal transformation \(T^T\) to \(\xi\), we maintain the distribution:
    \[
    T^T \xi \sim N(0, I_n).
    \]
    Therefore,
    \[
    \xi^T R \xi = \xi^T (T D_r T^T) \xi = (T^T \xi)^T D_r (T^T \xi) = \sum_{i=1}^{r} (T^T \xi)_i^2 \sim \chi^2(r).
    \]
    
    \item 
    Let \(A_1 = B\xi\) and \(A_2 = R\xi\). The covariance between \(A_1\) and \(A_2\) is:
    \[
    \text{Cov}(A_1, A_2) = \text{Cov}(B\xi, R\xi) = B \text{Cov}(\xi, \xi) R^T = BR^T = BR = 0,
    \]
    since \(BR = 0\) by assumption. Given that both \(A_1\) and \(A_2\) are linear combinations of jointly normal variables, zero covariance implies independence.
    
    \item 
    Given that \(RS = 0\), by part (2), \(S\xi\) and \(R\xi\) are independent. From part (1), we have \(S\xi \sim \chi^2(s)\) and \(R\xi \sim \chi^2(r)\). Thus, the ratio:
    \[
    \frac{s}{r} \frac{\xi^T R \xi}{\xi^T S \xi} \sim F(r, s)
    \]
    follows directly from the definition of the \(F\)-distribution as the ratio of two scaled chi-squared variables.
\end{enumerate}
\end{proof}

\subsection*{Theorem: Confidence Sets in Linear Models with Unknown \(\sigma^2\)}

\begin{theorem}[Linear Model Confidence Sets - Unknown \(\sigma^2\)]
Assume the regular linear model \(Y = X\beta + \varepsilon\), with \(\text{rank}(X) = p \le n\) and \(\varepsilon \sim N(0, \sigma^2 I_n)\). Let \(\alpha \in (0,1)\). Then:

\begin{enumerate}
    \item 
    Let \(q_{F_{p, n-p},1-\alpha}\) be the \(1-\alpha\) quantile of the \(F_{p, n-p}\) distribution. Then the set
    \[
    C(Y, X) = \left\{\beta \in \mathbb{R}^p \mid \frac{||X (\beta - \hat{\beta}_{OLS})||^2}{p \hat{\sigma}^2} \le q_{F_{p, n-p},1-\alpha} \right\}
    \]
    is a \(1-\alpha\) confidence set for \(\beta\).
    
    \item 
    For a specific linear combination \(\alpha = \langle \beta, v \rangle\) for some \(v \in \mathbb{R}^p\), a \(1-\alpha\) confidence interval is given by:
    \[
    C = C(Y, X) = \left\{ \alpha \in \mathbb{R} \mid \left| \frac{\alpha - \hat{\alpha}}{\hat{\sigma} \sqrt{v^T (X^TX)^{-1} v}} \right| < q \right\},
    \]
    where \(\hat{\alpha} = \langle \hat{\beta}_{OLS}, v \rangle\) and \(q\) is the \(1 - \frac{\alpha}{2}\) quantile of the \(t_{n-p}\) distribution.
\end{enumerate}
\end{theorem}

\begin{proof}
\begin{enumerate}
    \item 
    Consider the residual sum of squares under the null hypothesis \(H_0: \beta = \hat{\beta}_{OLS}\):
    \[
    RSS = ||Y - X\hat{\beta}_{OLS}||^2.
    \]
    The residual sum of squares under the alternative hypothesis is:
    \[
    RSS_{H_0} = ||Y - X\beta||^2.
    \]
    Given the orthonormal design and the properties of the Ridge estimator, we have:
    \[
    RSS_{H_0} - RSS = (K \hat{\beta}_{OLS} - d)^2 (K (X^TX)^{-1} K^T)^{-1},
    \]
    where \(K\) is the contrast matrix. Under \(H_0\), this difference follows a chi-squared distribution with \(r\) degrees of freedom:
    \[
    \frac{RSS_{H_0} - RSS}{\sigma^2} \sim \chi^2(r).
    \]
    Consequently, the ratio:
    \[
    \frac{RSS_{H_0} - RSS}{r \hat{\sigma}^2} \sim F(r, n - p).
    \]
    This establishes that:
    \[
    \frac{||X (\beta - \hat{\beta}_{OLS})||^2}{p \hat{\sigma}^2} \sim F(p, n - p),
    \]
    leading to the construction of the confidence set \(C(Y, X)\) as stated.

    \item 
    For the specific linear combination \(\alpha = \langle \beta, v \rangle\), the OLS estimator satisfies:
    \[
    \hat{\alpha} = \langle \hat{\beta}_{OLS}, v \rangle \sim N\left(\alpha, \sigma^2 v^T (X^TX)^{-1} v \right).
    \]
    The standardized statistic is:
    \[
    \frac{\alpha - \hat{\alpha}}{\sigma \sqrt{v^T (X^TX)^{-1} v}} \sim t_{n - p}.
    \]
    However, since \(\sigma^2\) is unknown and estimated by \(\hat{\sigma}^2\), the statistic becomes:
    \[
    \frac{\alpha - \hat{\alpha}}{\hat{\sigma} \sqrt{v^T (X^TX)^{-1} v}} \sim t_{n - p}.
    \]
    Thus, the confidence interval \(C\) is constructed by:
    \[
    \left| \frac{\alpha - \hat{\alpha}}{\hat{\sigma} \sqrt{v^T (X^TX)^{-1} v}} \right| < q,
    \]
    where \(q\) is the critical value from the \(t_{n - p}\) distribution corresponding to the desired confidence level.
\end{enumerate}
\end{proof}

\subsection{The \(t\)- and \(F\)-Tests}

Confidence sets facilitate hypothesis testing. Here, we outline the procedures for conducting \(t\)-tests and \(F\)-tests in the linear model context.

\subsubsection*{Remark: Method (t-test)}

In a regular linear model with \(\varepsilon \sim N(0, \sigma^2 I_n)\), consider the null hypothesis:
\[
H_0: \gamma = \gamma_0 \quad \text{vs} \quad H_1: \gamma \neq \gamma_0,
\]
where \(\gamma = \langle \beta, v \rangle\) for some \(v \in \mathbb{R}^p\).

The two-sided \(t\)-test statistic is defined as:
\[
T_{\gamma_0, n - p} = \frac{\gamma_0 - \hat{\gamma}}{\hat{\sigma} \sqrt{v^T (X^TX)^{-1} v}},
\]
where \(\hat{\gamma} = \langle \hat{\beta}_{OLS}, v \rangle\).

The corresponding test function is:
\[
\varphi_{\alpha_0}(Y, X) = \mathbf{1} \left( \left| T_{\gamma_0, n - p}(Y, X) \right| > q \right),
\]
where \(q\) is the critical value from the \(t_{n - p}\) distribution corresponding to the significance level \(\alpha_0\).

\subsubsection*{Remark: Method (F-test)}

For the same linear model setting, consider the null hypothesis:
\[
H_0: \beta = \beta_0 \quad \text{vs} \quad H_1: \beta \neq \beta_0,
\]
where \(\beta_0 \in \mathbb{R}^p\).

The \(F\)-test statistic is defined as:
\[
F_{\beta_0, n - p}(Y, X) = \frac{||X (\beta - \hat{\beta}_{OLS})||^2}{p \hat{\sigma}^2},
\]
which follows an \(F_{p, n - p}\) distribution under \(H_0\).

The corresponding test function is:
\[
\varphi_{\beta_0}(Y, X) = \mathbb{1} \left( |F_{\beta_0, n - p}(Y, X)| > q \right),
\]
where \(q\) is the critical value from the \(F_{p, n - p}\) distribution corresponding to the desired significance level.

\subsection{General Linear Hypothesis Testing Problems}

Beyond testing individual parameters, we often encounter more complex hypotheses involving multiple linear constraints on \(\beta\). This section addresses such scenarios.

\begin{definition}
A \textbf{linear hypothesis testing problem} is of the form:
\[
H_0: K\beta = d \quad \text{vs} \quad H_1: K\beta \neq d,
\]
where \(K \in \mathbb{R}^{r \times p}\) is a matrix with \(\text{rank}(K) = r \le p\) and \(d \in \mathbb{R}^r\). Essentially, this represents \(r\) linear constraints on the parameter vector \(\beta\). The matrix \(K\) is referred to as the \textbf{contrast matrix}.
\end{definition}

\begin{theorem}
Assume the regular linear model \(Y = X\beta + \varepsilon\), with \(\text{rank}(X) = p \le n\) and \(\varepsilon \sim N(0, \sigma^2 I_n)\). Consider the hypothesis:
\[
H_0: K\beta = d \quad \text{vs} \quad H_1: K\beta \neq d.
\]
Define the residual sum of squares (RSS) as:
\[
RSS = ||Y - X\hat{\beta}_{OLS}||^2 \quad \text{and} \quad RSS_{H_0} = ||Y - X\hat{\beta}_{H_0}||^2,
\]
where \(\hat{\beta}_{H_0}\) is the estimator of \(\beta\) under the constraint \(K\beta = d\).

Then:
\begin{enumerate}
    \item The constrained estimator is:
    \[
    \hat{\beta}_{H_0} = \hat{\beta}_{OLS} - (X^TX)^{-1} K^T (K (X^TX)^{-1} K^T)^{-1} (K\hat{\beta}_{OLS} - d).
    \]
    
    \item The difference in RSS satisfies:
    \[
    RSS_{H_0} - RSS = (K \hat{\beta}_{OLS} - d)^T (K (X^TX)^{-1} K^T)^{-1} (K \hat{\beta}_{OLS} - d) \sim \sigma^2 \chi^2(r).
    \]
    
    \item The test statistic:
    \[
    F = \frac{(RSS_{H_0} - RSS)/r}{RSS/(n - p)} \sim F_{r, n - p}
    \]
    under the null hypothesis \(H_0\).
\end{enumerate}
\end{theorem}

\begin{proof}
\begin{enumerate}
    \item 
    The constrained estimator \(\hat{\beta}_{H_0}\) minimizes the RSS subject to \(K\beta = d\). By applying the method of Lagrange multipliers or directly solving the constrained optimization problem, we obtain:
    \[
    \hat{\beta}_{H_0} = \hat{\beta}_{OLS} - (X^TX)^{-1} K^T (K (X^TX)^{-1} K^T)^{-1} (K\hat{\beta}_{OLS} - d).
    \]
    
    \item 
    The difference in RSS between the unconstrained and constrained models is:
    \[
    RSS_{H_0} - RSS = (K \hat{\beta}_{OLS} - d)^T (K (X^TX)^{-1} K^T)^{-1} (K \hat{\beta}_{OLS} - d).
    \]
    Under \(H_0\), \(K\hat{\beta}_{OLS} \sim N(d, \sigma^2 K (X^TX)^{-1} K^T)\). Thus, the quadratic form above follows a chi-squared distribution with \(r\) degrees of freedom:
    \[
    \frac{RSS_{H_0} - RSS}{\sigma^2} \sim \chi^2(r).
    \]
    
    \item 
    Defining the \(F\)-statistic as:
    \[
    F = \frac{(RSS_{H_0} - RSS)/r}{RSS/(n - p)} = \frac{RSS_{H_0} - RSS}{r \hat{\sigma}^2} \sim F_{r, n - p},
    \]
    where \(\hat{\sigma}^2 = \frac{RSS}{n - p}\) is the unbiased estimator of \(\sigma^2\). Under \(H_0\), this statistic follows the \(F\)-distribution with \(r\) and \(n - p\) degrees of freedom.
\end{enumerate}
\end{proof}


\subsection{The \(t\)- and \(F\)-Tests}

\subsubsection*{Remark: \(t\)-Test Method}

Consider a regular linear model where \(\varepsilon \sim N(0, \sigma^2 I_n)\). To test the null hypothesis \(H_0: \gamma = \gamma_0\) against the alternative \(H_1: \gamma \neq \gamma_0\), where \(\gamma = \langle \beta, v \rangle\) for some \(v \in \mathbb{R}^p\), we employ the two-sided \(t\)-test.

The \(t\)-test statistic is defined as:
\[
T_{\gamma_0, n - p} = \frac{\gamma_0 - \hat{\gamma}}{\hat{\sigma} \sqrt{v^T (X^TX)^{-1} v}},
\]
where \(\hat{\gamma} = \langle \hat{\beta}_{OLS}, v \rangle\).

The corresponding test function is:
\[
\varphi_{\alpha_0}(Y, X) = \mathbf{1} \left( \left| T_{\gamma_0, n - p}(Y, X) \right| > q \right),
\]
where \(q\) is the critical value from the \(t_{n - p}\) distribution corresponding to the significance level \(\alpha_0\).

\subsubsection*{Remark: \(F\)-Test Method}

In the same linear model setting, to test the null hypothesis \(H_0: \beta = \beta_0\) against \(H_1: \beta \neq \beta_0\), we use the \(F\)-test.

The \(F\)-test statistic is defined as:
\[
F_{\beta_0, n - p}(Y, X) = \frac{||X (\beta - \hat{\beta}_{OLS})||^2}{p \hat{\sigma}^2},
\]
which follows an \(F_{p, n - p}\) distribution under \(H_0\).

The corresponding test function is:
\[
\varphi_{\beta_0}(Y, X) = \mathbb{1} \left( |F_{\beta_0, n - p}(Y, X)| > q \right),
\]
where \(q\) is the critical value from the \(F_{p, n - p}\) distribution at the desired significance level.

\subsection{General Linear Hypothesis Testing Problems}

Beyond testing individual coefficients or simple contrasts, we often need to test more general linear hypotheses involving multiple parameters. This section formalizes such testing problems and provides the necessary statistical framework.

\begin{definition}
A \textbf{linear hypothesis testing problem} involves testing \(H_0: K\beta = d\) against \(H_1: K\beta \neq d\), where:
\begin{itemize}
    \item \(K \in \mathbb{R}^{r \times p}\) is the contrast matrix with \(\text{rank}(K) = r \le p\).
    \item \(d \in \mathbb{R}^r\) is the vector of constants.
\end{itemize}
\end{definition}

\begin{theorem}
Assume the regular linear model \(Y = X\beta + \varepsilon\), with \(\text{rank}(X) = p \le n\) and \(\varepsilon \sim N(0, \sigma^2 I_n)\). Consider the linear hypothesis:
\[
H_0: K\beta = d \quad \text{vs} \quad H_1: K\beta \neq d.
\]
Define the residual sum of squares (RSS) under \(H_0\) as:
\[
RSS_{H_0} = ||Y - X\hat{\beta}_{H_0}||^2,
\]
where \(\hat{\beta}_{H_0}\) is the constrained estimator minimizing RSS subject to \(K\beta = d\).

Then:
\begin{enumerate}
    \item 
    The constrained estimator is:
    \[
    \hat{\beta}_{H_0} = \hat{\beta}_{OLS} - (X^TX)^{-1} K^T (K (X^TX)^{-1} K^T)^{-1} (K \hat{\beta}_{OLS} - d).
    \]
    
    \item 
    The difference in RSS is given by:
    \[
    RSS_{H_0} - RSS = (K \hat{\beta}_{OLS} - d)^T (K (X^TX)^{-1} K^T)^{-1} (K \hat{\beta}_{OLS} - d) \sim \sigma^2 \chi^2(r).
    \]
    
    \item 
    The \(F\)-statistic is defined as:
    \[
    F = \frac{RSS_{H_0} - RSS}{r \hat{\sigma}^2} \sim F_{r, n - p},
    \]
    under the null hypothesis \(H_0\).
\end{enumerate}
\end{theorem}

\begin{proof}
\begin{enumerate}
    \item 
    To derive the constrained estimator \(\hat{\beta}_{H_0}\), we minimize the RSS subject to \(K\beta = d\). Utilizing the method of Lagrange multipliers or projecting the OLS estimator onto the constraint set, we obtain:
    \[
    \hat{\beta}_{H_0} = \hat{\beta}_{OLS} - (X^TX)^{-1} K^T (K (X^TX)^{-1} K^T)^{-1} (K \hat{\beta}_{OLS} - d).
    \]
    
    \item 
    The difference in RSS between the constrained and unconstrained models is a quadratic form:
    \[
    RSS_{H_0} - RSS = (K \hat{\beta}_{OLS} - d)^T (K (X^TX)^{-1} K^T)^{-1} (K \hat{\beta}_{OLS} - d).
    \]
    Under \(H_0\), \(K\hat{\beta}_{OLS}\) follows a normal distribution centered at \(d\), and the quadratic form follows a scaled chi-squared distribution:
    \[
    \frac{RSS_{H_0} - RSS}{\sigma^2} \sim \chi^2(r).
    \]
    
    \item 
    Defining the \(F\)-statistic as:
    \[
    F = \frac{RSS_{H_0} - RSS}{r \hat{\sigma}^2},
    \]
    and noting that \(\hat{\sigma}^2 = \frac{RSS}{n - p}\) is an unbiased estimator of \(\sigma^2\), we conclude:
    \[
    F \sim F_{r, n - p}.
    \]
    This completes the proof.
\end{enumerate}
\end{proof}
\chapter{Lecture 11}
\section{Introduction}

In this lecture, we delve into hypothesis testing within the framework of regular linear models (LM) and explore the Analysis of Variance (ANOVA). We will cover the derivation and proof of key theorems related to the F-test in linear models and the decomposition of the Residual Sum of Squares (RSS) in ANOVA.

\section{Hypothesis Testing in Linear Models}

Consider the standard linear model:
\[
Y = X\beta + \varepsilon,
\]
where:
\begin{itemize}
    \item $Y \in \mathbb{R}^n$ is the response vector,
    \item $X \in \mathbb{R}^{n \times p}$ is the design matrix,
    \item $\beta \in \mathbb{R}^p$ is the parameter vector,
    \item $\varepsilon \sim N(0, \sigma^2 I_n)$ is the error term.
\end{itemize}

We are interested in testing linear hypotheses of the form:
\[
H_0: K \beta = d \quad \text{vs.} \quad H_1: K \beta \neq d,
\]
where $K \in \mathbb{R}^{m \times p}$ is a known matrix and $d \in \mathbb{R}^m$ is a known vector.

\subsection{Residual Sum of Squares (RSS)}

Define the Residual Sum of Squares as:
\[
RSS = \|Y - X \beta \|^2,
\]
and the RSS under the null hypothesis $H_0$ as:
\[
RSS_{H_0} = \|Y - X \beta_{H_0}\|^2,
\]
where $\beta_{H_0}$ is the Ordinary Least Squares (OLS) estimator constrained by $K \beta = d$. The constrained estimator $\beta_{H_0}$ is given by:
\[
\beta_{H_0} = \hat{\beta} - (X^\top X)^{-1} K^\top \left(K (X^\top X)^{-1} K^\top \right)^{-1} (K \hat{\beta} - d),
\]
where $\hat{\beta} = (X^\top X)^{-1} X^\top Y$ is the unconstrained OLS estimator.

\subsection{The F-Test Theorem}

\textbf{Theorem:} Assume a regular linear model with $\varepsilon \sim N(0, \sigma^2 I_n)$. Consider the hypotheses:
\[
H_0: K \beta = d \quad \text{vs.} \quad H_1: K \beta \neq d.
\]
Define:
\[
F = \frac{1}{p} \frac{RSS_{H_0} - RSS}{RSS/n}.
\]
Under $H_0$, $F$ follows an $F$-distribution with $m$ and $n - p$ degrees of freedom:
\[
F \sim F_{m, n-p}.
\]

\textbf{Proof:}

\begin{enumerate}
    \item \textbf{Relation between $RSS_{H_0}$ and $RSS$} \\
    We first show that:
    \[
    RSS_{H_0} - RSS = \|X (\beta_{H_0} - \hat{\beta})\|^2 = (K \hat{\beta} - d)^\top \left(K (X^\top X)^{-1} K^\top \right)^{-1} (K \hat{\beta} - d).
    \]
    
    Since $\beta_{H_0}$ minimizes $RSS$ subject to $K \beta = d$, it follows that:
    \[
    K \beta_{H_0} = d.
    \]
    Substituting the expression for $\beta_{H_0}$:
    \[
    K \beta_{H_0} = K \hat{\beta} - K (X^\top X)^{-1} K^\top \left(K (X^\top X)^{-1} K^\top \right)^{-1} (K \hat{\beta} - d) = d.
    \]
    
    \item \textbf{Distribution under $H_0$} \\
    Under $H_0$, we have:
    \[
    RSS_{H_0} \sim \sigma^2 \chi^2(n - m).
    \]
    Additionally, the difference $RSS_{H_0} - RSS$ is:
    \[
    RSS_{H_0} - RSS = \|X (\beta_{H_0} - \hat{\beta})\|^2.
    \]
    Let $Z = K \hat{\beta} - d$. Then:
    \[
    \mathbb{E}[Z] = K \mathbb{E}[\hat{\beta}] - d = K \beta - d = 0 \quad \text{under } H_0.
    \]
    The variance of $Z$ is:
    \[
    \text{Var}(Z) = \sigma^2 K (X^\top X)^{-1} K^\top.
    \]
    Therefore:
    \[
    Z \sim \mathcal{N}(0, \sigma^2 K (X^\top X)^{-1} K^\top).
    \]
    Consequently:
    \[
    \frac{RSS_{H_0} - RSS}{\sigma^2} = \frac{Z^\top \left(K (X^\top X)^{-1} K^\top \right)^{-1} Z}{\sigma^2} \sim \chi^2(m).
    \]
    
    Also, under $H_0$:
    \[
    RSS \sim \sigma^2 \chi^2(n - p).
    \]
    
    \item \textbf{Construction of the F-statistic} \\
    Combining the results, we define the F-statistic as:
    \[
    F = \frac{\frac{RSS_{H_0} - RSS}{m}}{\frac{RSS}{n - p}}.
    \]
    Under $H_0$, since $RSS_{H_0} - RSS$ and $RSS$ are independent,
    \[
    F \sim F_{m, n - p}.
    \]
\end{enumerate}

This concludes the proof of the F-test theorem.

\section{Analysis of Variance (ANOVA)}

ANOVA is a statistical method used to compare means across multiple groups to determine if at least one group mean is different from the others. It decomposes the total variability in the data into components attributable to different sources.

\subsection{Motivation}

Suppose we have data from $k$ different groups. We aim to test whether the means of these groups are equal.

\subsection{Model Specification}

We are given data:
\[
Y_{ij} = \mu_i + \varepsilon_{ij}, \quad i = 1, \ldots, k, \quad j = 1, \ldots, n_i,
\]
where:
\begin{itemize}
    \item $Y_{ij}$ is the observation in the $i$-th group and $j$-th replicate,
    \item $\mu_i$ is the mean of the $i$-th group,
    \item $\varepsilon_{ij} \sim \mathcal{N}(0, \sigma^2)$ are independent error terms.
\end{itemize}

\subsection{Factor Model}

ANOVA can be viewed as a special case of a linear model known as the \textit{factor model} with one categorical predictor (factor). The index $i = 1, \ldots, k$ represents different levels (groups) of the factor.

\begin{itemize}
    \item $n = \sum_{i=1}^k n_i$ is the total sample size.
    \item The design is \textbf{balanced} if all groups have the same number of observations, i.e., $n_1 = n_2 = \ldots = n_k$.
\end{itemize}

\textbf{Remark:} ANOVA can be expressed in matrix form as:
\[
\begin{pmatrix}
Y_{1,1} \\
Y_{1,2} \\
\vdots \\
Y_{k,n_k}
\end{pmatrix}
=
\begin{pmatrix}
\mathbf{1}_{n_1} & 0 & \cdots & 0 \\
0 & \mathbf{1}_{n_2} & \cdots & 0 \\
\vdots & \vdots & \ddots & \vdots \\
0 & 0 & \cdots & \mathbf{1}_{n_k}
\end{pmatrix}
\begin{pmatrix}
\mu_1 \\
\mu_2 \\
\vdots \\
\mu_k
\end{pmatrix}
+
\varepsilon,
\]
where $\mathbf{1}_{n_i}$ is a column vector of ones of length $n_i$.

\subsection{Hypothesis Testing in ANOVA}

We aim to test:
\[
H_0: \mu_1 = \mu_2 = \ldots = \mu_k \quad \text{vs.} \quad H_a: \exists i, j \text{ with } \mu_i \neq \mu_j.
\]

\subsection{Decomposition of RSS}

To perform ANOVA, we decompose the Total Sum of Squares (SST) into the Sum of Squares Between Groups (SSB) and the Sum of Squares Within Groups (SSW).

\textbf{Definitions:}
\begin{align*}
\bar{Y}_{i.} &= \frac{1}{n_i} \sum_{j=1}^{n_i} Y_{ij}, \quad i = 1, \ldots, k \quad \text{(Group Means)}, \\
\bar{Y}_{..} &= \frac{1}{n} \sum_{i=1}^k \sum_{j=1}^{n_i} Y_{ij} \quad \text{(Overall Mean)}.
\end{align*}

\textbf{Sum of Squares:}
\begin{align*}
SSB &= \sum_{i=1}^k n_i (\bar{Y}_{i.} - \bar{Y}_{..})^2 \quad \text{(Sum of Squares Between Groups)}, \\
SSW &= \sum_{i=1}^k \sum_{j=1}^{n_i} (Y_{ij} - \bar{Y}_{i.})^2 \quad \text{(Sum of Squares Within Groups)}.
\end{align*}

\textbf{Total Sum of Squares:}
\[
SST = \sum_{i=1}^k \sum_{j=1}^{n_i} (Y_{ij} - \bar{Y}_{..})^2.
\]

\textbf{Theorem (Decomposition of RSS):} 
\[
SST = SSB + SSW.
\]

\textbf{Proof:}

\begin{align*}
SST &= \sum_{i=1}^k \sum_{j=1}^{n_i} (Y_{ij} - \bar{Y}_{..})^2 \\
&= \sum_{i=1}^k \sum_{j=1}^{n_i} \left[(Y_{ij} - \bar{Y}_{i.}) + (\bar{Y}_{i.} - \bar{Y}_{..})\right]^2 \\
&= \sum_{i=1}^k \sum_{j=1}^{n_i} \left[(Y_{ij} - \bar{Y}_{i.})^2 + (\bar{Y}_{i.} - \bar{Y}_{..})^2 + 2(Y_{ij} - \bar{Y}_{i.})(\bar{Y}_{i.} - \bar{Y}_{..})\right] \\
&= \sum_{i=1}^k \sum_{j=1}^{n_i} (Y_{ij} - \bar{Y}_{i.})^2 + \sum_{i=1}^k \sum_{j=1}^{n_i} (\bar{Y}_{i.} - \bar{Y}_{..})^2 + 2 \sum_{i=1}^k \sum_{j=1}^{n_i} (Y_{ij} - \bar{Y}_{i.})(\bar{Y}_{i.} - \bar{Y}_{..}) \\
&= SSW + SSB + C,
\end{align*}
where:
\[
C = 2 \sum_{i=1}^k (\bar{Y}_{i.} - \bar{Y}_{..}) \sum_{j=1}^{n_i} (Y_{ij} - \bar{Y}_{i.}).
\]
Since:
\[
\sum_{j=1}^{n_i} (Y_{ij} - \bar{Y}_{i.}) = 0 \quad \text{for each } i,
\]
it follows that:
\[
C = 0.
\]
Thus:
\[
SST = SSB + SSW.
\]

\subsection{ANOVA F-Test}

\textbf{Theorem:}
\begin{enumerate}
    \item The least squares estimator for $\mu = (\mu_1, \ldots, \mu_k)^\top \in \mathbb{R}^k$ is:
    \[
    \hat{\mu} = (\bar{Y}_{1.}, \ldots, \bar{Y}_{k.})^\top.
    \]
    
    \item Under $H_0$:
    \[
    \frac{SSW}{\sigma^2} \sim \chi^2(n - k).
    \]
    
    \item Under $H_0$:
    \[
    \frac{SSB}{\sigma^2} \sim \chi^2(k - 1).
    \]
    
    \item $SSW$ and $SSB$ are independent under $H_0$, and the F-statistic defined by:
    \[
    F = \frac{\frac{SSB}{k - 1}}{\frac{SSW}{n - k}} \sim F(k - 1, n - k) \quad \text{under } H_0.
    \]
\end{enumerate}

\textbf{Proof:}

\begin{enumerate}
    \item \textbf{Estimator for $\mu$} \\
    The least squares estimator minimizes $SSW$. Taking derivatives with respect to $\mu_i$ and setting them to zero yields:
    \[
    \hat{\mu}_i = \bar{Y}_{i.} \quad \text{for each } i = 1, \ldots, k.
    \]
    
    \item \textbf{Distribution of SSW under $H_0$} \\
    Under $H_0$, all group means are equal to the overall mean $\bar{Y}_{..}$. Therefore, $SSW$ represents the variability within groups:
    \[
    SSW = \sum_{i=1}^k \sum_{j=1}^{n_i} (Y_{ij} - \bar{Y}_{i.})^2.
    \]
    Each term $(Y_{ij} - \bar{Y}_{i.})^2$ is chi-squared distributed with 1 degree of freedom. Summing over all groups and observations, we have:
    \[
    \frac{SSW}{\sigma^2} \sim \chi^2(n - k).
    \]
    
    \item \textbf{Distribution of SSB under $H_0$} \\
    Under $H_0$, the between-group variability $SSB$ is based on deviations of group means from the overall mean:
    \[
    SSB = \sum_{i=1}^k n_i (\bar{Y}_{i.} - \bar{Y}_{..})^2.
    \]
    Each term $n_i (\bar{Y}_{i.} - \bar{Y}_{..})^2 / \sigma^2$ follows a chi-squared distribution with 1 degree of freedom. Summing over $k$ groups, we obtain:
    \[
    \frac{SSB}{\sigma^2} \sim \chi^2(k - 1).
    \]
    
    \item \textbf{Independence of SSB and SSW and F-Statistic Distribution} \\
    Under $H_0$, $SSB$ and $SSW$ are independent because $SSB$ depends only on the group means, while $SSW$ depends on the deviations within groups, which are orthogonal to the group means.
    
    The F-statistic is defined as:
    \[
    F = \frac{\frac{SSB}{k - 1}}{\frac{SSW}{n - k}}.
    \]
    Since $\frac{SSB}{\sigma^2} \sim \chi^2(k - 1)$ and $\frac{SSW}{\sigma^2} \sim \chi^2(n - k)$, and they are independent, it follows that:
    \[
    F \sim F(k - 1, n - k) \quad \text{under } H_0.
    \]
\end{enumerate}

\section{Conclusion}

In this lecture, we established the foundational theorems for hypothesis testing in linear models and ANOVA. We derived the distribution of the F-statistic under the null hypothesis, which allows us to test the equality of multiple group means. Understanding these concepts is crucial for analyzing variance and making informed statistical inferences in various applied contexts.

\chapter{Lecture 12}
\section{Analysis of Variance (ANOVA)}

ANOVA is a statistical method used to compare the means of three or more groups to determine if at least one group mean is different from the others. It is particularly useful when dealing with experimental data where multiple treatments or conditions are being tested.

\subsection{Linear Model for ANOVA}

Consider a linear model where we have $k$ groups (or factors/categories), each with $n_i$ observations. The model can be expressed as:
\[
Y_{i,j} = \mu_i + \varepsilon_{ij} \quad \text{for } i = 1, \dots, k \text{ and } j = 1, \dots, n_i
\]
where:
\begin{itemize}
    \item $Y_{i,j}$ is the $j$-th observation in the $i$-th group.
    \item $\mu_i$ is the mean of the $i$-th group.
    \item $\varepsilon_{ij}$ are the random errors, assumed to be normally distributed with mean $0$ and variance $\sigma^2$, i.e., $\varepsilon_{ij} \sim N(0, \sigma^2)$.
\end{itemize}

\subsection{Ordinary Least Squares (OLS) Estimation}

In the context of ANOVA, the OLS estimates of the group means $\mu_i$ can be derived. Let $X$ be the design matrix for the linear model. A key note is that:
\[
X^T X = \|X\|^2_{\mathbb{R}^n}
\]
This relationship is fundamental in deriving the OLS estimates.

\subsection{Theorem: Properties of the ANOVA Model}

\begin{theorem}
In the ANOVA model with $\varepsilon_{ij} \sim N(0, \sigma^2)$, the following properties hold:
\begin{enumerate}
    \item The OLS estimate of the group means is given by:
    \[
    \hat{\mu} = \left(\bar{y}_{1,\cdot}, \dots, \bar{y}_{k,\cdot}\right) \quad \text{where} \quad \bar{y}_{i,\cdot} = \frac{1}{n_i} \sum_{j=1}^{n_i} y_{ij}
    \]
    
    \item The Within-Group Sum of Squares (SSW) scaled by $\sigma^2$ follows a chi-squared distribution:
    \[
    \frac{SSW}{\sigma^2} = \frac{1}{\sigma^2} \sum_{i=1}^k \sum_{j=1}^{n_i} \left(y_{ij} - \bar{y}_{i,\cdot}\right)^2 \sim \chi^2(n - k)
    \]
    where $n = \sum_{i=1}^k n_i$ is the total number of observations.
    
    \item Under the null hypothesis $H_0: \mu_1 = \mu_2 = \dots = \mu_k$, the Between-Group Sum of Squares (SSB) scaled by $\sigma^2$ follows a chi-squared distribution:
    \[
    \frac{SSB}{\sigma^2} = \frac{1}{\sigma^2} \sum_{i=1}^k n_i \left(\bar{y}_{i,\cdot} - \bar{y}\right)^2 \sim \chi^2(k - 1)
    \]
    where $\bar{y} = \frac{1}{n} \sum_{i=1}^k \sum_{j=1}^{n_i} y_{ij}$ is the overall mean.
    
    \item SSW and SSB are independent. Under $H_0$, the ratio
    \[
    \frac{(n - k)SSB}{(k - 1)SSW} \sim F(k - 1, n - k)
    \]
    follows an $F$-distribution with $(k - 1, n - k)$ degrees of freedom.
\end{enumerate}
\end{theorem}

\begin{proof}
We will prove each part of the theorem sequentially.

\begin{enumerate}
    \item \textbf{OLS Estimate of $\mu$}: The OLS estimator is given by
    \[
    \hat{\mu} = (X^T X)^{-1} X^T Y
    \]
    Given the structure of $X^T X$, we have:
    \[
    (X^T X)^{-1} = \begin{pmatrix}
        \frac{1}{n_1} & \dots & 0 \\
        \vdots & \ddots & \vdots \\
        0 & \dots & \frac{1}{n_k}
    \end{pmatrix}
    \]
    Multiplying this with $X^T Y$, we get:
    \[
    \hat{\mu} = \begin{pmatrix}
        \frac{1}{n_1} & \dots & 0 \\
        \vdots & \ddots & \vdots \\
        0 & \dots & \frac{1}{n_k}
    \end{pmatrix}
    \begin{pmatrix}
        \sum_{j=1}^{n_1} Y_{1,j} \\
        \vdots \\
        \sum_{j=1}^{n_k} Y_{k,j}
    \end{pmatrix}
    = \begin{pmatrix}
        \bar{Y}_{1,\cdot} \\
        \vdots \\
        \bar{Y}_{k,\cdot}
    \end{pmatrix}
    \]
    where $\bar{Y}_{i,\cdot} = \frac{1}{n_i} \sum_{j=1}^{n_i} Y_{i,j}$.
    
    \item \textbf{Distribution of SSW}: The Within-Group Sum of Squares is defined as
    \[
    SSW = \|Y - X\hat{\mu}\|^2_{\mathbb{R}^n} = \sum_{i=1}^k \sum_{j=1}^{n_i} \left(Y_{ij} - \bar{Y}_{i,\cdot}\right)^2
    \]
    Since $\varepsilon_{ij} \sim N(0, \sigma^2)$, each squared deviation $\left(Y_{ij} - \bar{Y}_{i,\cdot}\right)^2$ follows a scaled chi-squared distribution. Aggregating over all groups and observations, we have:
    \[
    \frac{SSW}{\sigma^2} \sim \chi^2(n - k)
    \]
    where $n - k$ degrees of freedom account for the estimation of $k$ group means.
    
    \item \textbf{Distribution of SSB under $H_0$}: Under the null hypothesis $H_0: \mu_1 = \mu_2 = \dots = \mu_k$, the Between-Group Sum of Squares is given by
    \[
    SSB = \sum_{i=1}^k n_i \left(\bar{Y}_{i,\cdot} - \bar{Y}\right)^2
    \]
    Under $H_0$, all group means are equal to the overall mean $\bar{Y}$. Therefore, $SSB$ measures the deviation of each group mean from the overall mean. The distribution of $SSB$ scaled by $\sigma^2$ is:
    \[
    \frac{SSB}{\sigma^2} \sim \chi^2(k - 1)
    \]
    This follows from the properties of chi-squared distributions and the fact that there are $k - 1$ independent comparisons among the $k$ group means.
    
    \item \textbf{F-Statistic and Independence of SSW and SSB}: The total sum of squares (SST) can be decomposed as:
    \[
    SST = SSW + SSB
    \]
    Under $H_0$, SSW and SSB are independent because the residuals within groups do not provide information about the between-group variability. The ratio of the scaled sums of squares follows an $F$-distribution:
    \[
    \frac{(n - k)SSB}{(k - 1)SSW} \sim F(k - 1, n - k)
    \]
    This result is derived from the general linear hypothesis testing framework, specifically Theorem 2.2.30 in \textit{Methoden der Statistik}.
\end{enumerate}
\end{proof}

\section{Exponential Families}

Exponential families encompass a wide range of probability distributions and include many of the common distributions used in statistics. They also generalize linear models, making them a fundamental concept in statistical theory.

\subsection{General Model and Hierarchy}

\[
\text{General Model} \supseteq \text{Exponential Families} \supseteq \text{Linear Models}
\]
This hierarchy indicates that exponential families are a subset of general statistical models and that linear models are a subset of exponential families.

\subsection{Regularity Assumptions}

For the theory to hold, certain regularity conditions must be satisfied by the statistical model $(P_\theta : \theta \in \Theta)$:

\begin{enumerate}
    \item \textbf{Dominated Model}: There exists a measure $\mu$ such that every $P_\theta$ is absolutely continuous with respect to $\mu$, denoted as $P_\theta \ll \mu$ for all $\theta \in \Theta$.
    
    \item \textbf{Parameter Space}: The parameter space $\Theta \subseteq \mathbb{R}^p$ is an open set with $p \geq 1$.
    
    \item \textbf{Likelihood Function}: The likelihood function $p_\theta(x) > 0$ for all $\theta \in \Theta$ and $x \in X$, ensuring that $\log p_\theta(x)$ is well-defined.
\end{enumerate}

\subsection{Score Function}

\begin{definition}[Score]
The score vector is defined as the gradient of the log-likelihood with respect to the parameter $\theta$:
\[
U_{\theta}(x) = \nabla_\theta \log p_\theta(x) = \begin{pmatrix}
    \frac{\partial}{\partial \theta_1} \log p_\theta(x) \\
    \vdots \\
    \frac{\partial}{\partial \theta_p} \log p_\theta(x)
\end{pmatrix}
\]
whenever it exists.
\end{definition}

\subsection{Fisher Information}

\begin{definition}[Fisher Information]
For $\theta \in \Theta$, the Fisher Information (FI) matrix is defined as:
\[
I(\theta) = \mathbb{E}\left[ U_\theta(x) U_\theta(x)^T \right] \in \mathbb{R}^{p \times p}
\]
whenever the expectation exists.
\end{definition}

\subsection{Additional Regularity Assumptions}

To further develop the theory, we impose additional regularity conditions:

\begin{enumerate}
    \item \textbf{Twice Differentiable Likelihood}: The likelihood function $p_\theta(x)$ is twice differentiable with respect to $\theta$, ensuring that the score function $U_\theta(x)$ is well-defined.
    
    \item \textbf{Finite Fisher Information}: For all $\theta \in \Theta$,
    \[
    \mathbb{E}_\theta \left[ \| U_\theta(x) \|^2_{\mathbb{R}^p} \right] < \infty
    \]
    ensuring that the Fisher Information matrix $I(\theta)$ is well-defined.
    
    \item \textbf{Interchange of Integration and Differentiation}: For relevant functions $h(x)$,
    \[
    \int h(x) \nabla_\theta p_\theta(x) \, \mu(dx) = \nabla_\theta \int h(x) p_\theta(x) \, \mu(dx)
    \]
    allowing the differentiation under the integral sign.
\end{enumerate}

\subsection{Properties of the Score and Fisher Information}

\begin{lemma}
Let $(P_\theta : \theta \in \Theta)$ be a regular statistical model as defined above. Then:
\begin{enumerate}
    \item The expected value of the score function is zero:
    \[
    \mathbb{E}_\theta \left[ U_\theta(x) \right] = 0
    \]
    
    \item The Fisher Information matrix is equal to the covariance matrix of the score function:
    \[
    I(\theta) = \text{Cov}\left( U_\theta(x) \right)
    \]
\end{enumerate}
\end{lemma}

\begin{proof}
\begin{enumerate}
    \item \textbf{Expected Score is Zero}: 
    \[
    \mathbb{E}_\theta \left[ U_\theta(x) \right] = \int_X \nabla_\theta \log p_\theta(x) \, p_\theta(x) \, \mu(dx) = \int_X \frac{\nabla_\theta p_\theta(x)}{p_\theta(x)} p_\theta(x) \, \mu(dx) = \int_X \nabla_\theta p_\theta(x) \, \mu(dx)
    \]
    Since $\int_X p_\theta(x) \, \mu(dx) = 1$ for all $\theta$, differentiating both sides with respect to $\theta$ gives:
    \[
    \nabla_\theta \int_X p_\theta(x) \, \mu(dx) = \int_X \nabla_\theta p_\theta(x) \, \mu(dx) = 0
    \]
    Therefore, $\mathbb{E}_\theta \left[ U_\theta(x) \right] = 0$.
    
    \item \textbf{Fisher Information as Covariance}: 
    By definition,
    \[
    I(\theta) = \mathbb{E}\left[ U_\theta(x) U_\theta(x)^T \right]
    \]
    Since $\mathbb{E}_\theta \left[ U_\theta(x) \right] = 0$, the covariance matrix is:
    \[
    \text{Cov}\left( U_\theta(x) \right) = \mathbb{E}\left[ U_\theta(x) U_\theta(x)^T \right] - \mathbb{E}\left[ U_\theta(x) \right] \mathbb{E}\left[ U_\theta(x)^T \right] = I(\theta)
    \]
\end{enumerate}
\end{proof}

\section{Uniformly Minimum Variance Unbiased Estimators (UMVUE)}

In statistical estimation, an unbiased estimator is one whose expected value equals the parameter it estimates. Among all unbiased estimators, the UMVUE is the one with the smallest variance, making it the most efficient unbiased estimator.

\subsection{Definition}

\begin{definition}[UMVUE]
Let $p(\theta) \in \mathbb{R}$ be a parameter of interest. A statistic $T: X \to \mathbb{R}$ is called a Uniformly Minimum Variance Unbiased Estimator (UMVUE) of $p(\theta)$ if:
\begin{itemize}
    \item $T$ is unbiased for $p(\theta)$, i.e., $\mathbb{E}_\theta[T(x)] = p(\theta)$ for all $\theta \in \Theta$.
    \item For any other unbiased estimator $S: X \to \mathbb{R}$ of $p(\theta)$, the variance of $T$ is less than or equal to the variance of $S$ for all $\theta \in \Theta$, i.e.,
    \[
    \text{Var}_\theta(T) \leq \text{Var}_\theta(S) \quad \forall \theta \in \Theta
    \]
\end{itemize}
\end{definition}

\subsection{Remarks}

\begin{enumerate}
    \item UMVUEs are the best possible unbiased estimators in terms of variance.
    
    \item Analogous to the Gauss-Markov theorem where the OLS estimator $\hat{\beta}_{OLS}$ is the best linear unbiased estimator (BLUE), the UMVUE is the best unbiased estimator without the linearity constraint.
    
    \item The Mean Squared Error (MSE) of an estimator $T$ can be decomposed as:
    \[
    \mathbb{E}_\theta \left[ \| T(x) - \rho(\theta) \|^2 \right] = \text{Bias}^2(T) + \text{Var}(T)
    \]
    For unbiased estimators, $\text{Bias}(T) = 0$, hence:
    \[
    \mathbb{E}_\theta \left[ \| T(x) - \rho(\theta) \|^2 \right] = \text{Var}(T) \leq \mathbb{E}_\theta \left[ \| S(x) - \rho(\theta) \|^2 \right] = \text{Var}(S)
    \]
\end{enumerate}

\subsection{Cramér-Rao Lower Bound}

The Cramér-Rao Lower Bound provides a lower bound on the variance of unbiased estimators. It is a fundamental result in estimation theory that helps identify the efficiency of estimators.

\begin{theorem}[Cramér-Rao Lower Bound]
Let $(P_\theta : \theta \in \Theta)$ be a regular statistical model. Let $\rho: \Theta \to \mathbb{R}$ be a continuously differentiable function of the parameter $\theta$. For any unbiased estimator $T$ of $\rho(\theta)$, i.e., $\mathbb{E}_\theta[T] = \rho(\theta)$, the variance of $T$ satisfies:
\[
\text{Var}_\theta(T) \geq \nabla_\theta \rho(\theta)^T I(\theta)^{-1} \nabla_\theta \rho(\theta)
\]
\end{theorem}

\begin{remark}
If the parameter space is one-dimensional ($\Theta \subseteq \mathbb{R}$) and $\rho(\theta) = \theta$, then the Cramér-Rao bound simplifies to:
\[
\text{Var}_\theta(T) \geq \frac{1}{I(\theta)}
\]
where $I(\theta)$ is the Fisher Information.
\end{remark}

\begin{proof}
Assume that $\Theta \subseteq \mathbb{R}$. Let $T$ be an unbiased estimator of $\rho(\theta)$, so $\mathbb{E}_\theta[T] = \rho(\theta)$. 

Consider the covariance between the score function $U_\theta(x)$ and the estimator $T(x)$:
\[
\text{Cov}_\theta(U_\theta, T) = \mathbb{E}_\theta[U_\theta T] - \mathbb{E}_\theta[U_\theta] \mathbb{E}_\theta[T] = \mathbb{E}_\theta[U_\theta T]
\]
since $\mathbb{E}_\theta[U_\theta] = 0$.

By the Cauchy-Schwarz inequality:
\[
|\text{Cov}_\theta(U_\theta, T)|^2 \leq \text{Var}_\theta(U_\theta) \cdot \text{Var}_\theta(T)
\]
which implies:
\[
\text{Var}_\theta(T) \geq \frac{|\text{Cov}_\theta(U_\theta, T)|^2}{\text{Var}_\theta(U_\theta)}
\]

Next, compute the covariance:
\[
\text{Cov}_\theta(U_\theta, T) = \mathbb{E}_\theta[U_\theta T] = \int_X U_\theta(x) T(x) p_\theta(x) \, \mu(dx) = \int_X \nabla_\theta \log p_\theta(x) T(x) p_\theta(x) \, \mu(dx)
\]
\[
= \int_X T(x) \nabla_\theta p_\theta(x) \, \mu(dx) = \nabla_\theta \int_X T(x) p_\theta(x) \, \mu(dx) = \nabla_\theta \mathbb{E}_\theta[T(x)] = \nabla_\theta \rho(\theta)
\]
since $T$ is unbiased.

Therefore:
\[
\text{Cov}_\theta(U_\theta, T) = \rho'(\theta)
\]
and:
\[
\text{Var}_\theta(U_\theta) = I(\theta)
\]
Substituting these into the inequality:
\[
\text{Var}_\theta(T) \geq \frac{(\rho'(\theta))^2}{I(\theta)}
\]
In higher dimensions, this generalizes to:
\[
\text{Var}_\theta(T) \geq \nabla_\theta \rho(\theta)^T I(\theta)^{-1} \nabla_\theta \rho(\theta)
\]
\end{proof}

\subsection{Invertibility of the Fisher Information Matrix}

An important regularity condition for the Cramér-Rao Lower Bound and related results is that the Fisher Information matrix $I(\theta)$ is invertible. This ensures that the lower bound on the variance of unbiased estimators is well-defined and finite.

\section{Conclusion}

In this lecture, we delved into the foundations of ANOVA, exploring the linear model framework, OLS estimation, and the distributional properties of the sum of squares under various hypotheses. We also introduced exponential families, discussing their place within the broader context of statistical models and the regularity conditions required for their analysis. Furthermore, we examined the concept of UMVUEs and established the Cramér-Rao Lower Bound, providing essential tools for assessing the efficiency of estimators.

Understanding these concepts is crucial for conducting rigorous statistical analysis and for developing efficient estimation procedures in various applications.
\chapter{Lecture 13}


\section*{Regular Statistical Models}

In this lecture, we delve into the framework of regular statistical models, explore the Cramér-Rao (CR) Inequality, and examine the structure of exponential families. We will begin by defining regular statistical models and proceed to establish foundational results, including the Fisher Information Matrix and the CR Inequality. Finally, we will explore the properties and examples of exponential families.

\subsection*{Definition of a Regular Statistical Model}

A \textbf{regular statistical model} is characterized by a parameter space and a family of probability density functions that satisfy certain smoothness and positivity conditions. Formally, a statistical model is regular if it satisfies the following:

\begin{itemize}
    \item \(\Theta \subset \mathbb{R}^p\) is an open set, where \(p\) denotes the number of parameters.
    \item For every \(\vartheta \in \Theta\) and \(x \in \mathcal{X}\), the density \(p_\vartheta(x)\) is strictly positive, i.e., \(p_\vartheta(x) > 0\).
    \item The density \(p_\vartheta(x)\) is continuously differentiable with respect to the parameter \(\vartheta\).
\end{itemize}

These conditions ensure that the model is smooth enough to allow for the interchange of differentiation and integration, which is crucial for deriving key statistical properties.

\subsection*{Fisher Information Matrix}

Given a regular statistical model, the \textbf{Fisher Information Matrix} quantifies the amount of information that an observable random variable \(X\) carries about the unknown parameter \(\vartheta\). It is defined as:

\[
I(\vartheta) = \mathbb{E}_\vartheta \left[ \nabla_\vartheta \log p_\vartheta(X) \cdot \nabla_\vartheta \log p_\vartheta(X)^T \right]
\]

Here, \(\nabla_\vartheta\) denotes the gradient with respect to the parameter vector \(\vartheta\), and the expectation is taken with respect to the distribution \(p_\vartheta\).

\begin{itemize}
    \item The Fisher Information Matrix \(I(\vartheta)\) exists for all \(\vartheta \in \Theta\).
    \item \(I(\vartheta)\) is positive definite, which implies that its inverse \(I(\vartheta)^{-1}\) exists.
\end{itemize}

The positivity and invertibility of \(I(\vartheta)\) are critical for deriving lower bounds on the variance of unbiased estimators, as we will see next.

\subsection*{Cramér-Rao / Information Inequality}

The Cramér-Rao Inequality provides a lower bound on the variance of unbiased estimators of a parameter. This fundamental result connects the variance of an estimator to the Fisher Information, establishing a benchmark for the efficiency of estimators.

\begin{theorem}[Cramér-Rao Inequality]
    Let \((p_\vartheta, \vartheta \in \Theta)\) be a regular statistical model. Suppose \(g: \Theta \to \mathbb{R}\) is a continuously differentiable function. Let \(T: \mathcal{X} \to \mathbb{R}\) be an unbiased estimator of \(g(\vartheta)\), meaning that \(\mathbb{E}_\vartheta[T] = g(\vartheta)\) for all \(\vartheta \in \Theta\).

    Then, for all \(\vartheta \in \Theta\),
    \[
    \text{Var}_\vartheta(T) \geq \left( g'(\vartheta) \right)^T I(\vartheta)^{-1} g'(\vartheta).
    \]
    This inequality is known as the \textbf{Cramér-Rao / Information Inequality}.
\end{theorem}

\subsection*{Score Vector}

The \textbf{Score Vector} plays a pivotal role in the analysis of statistical models. It is defined as the gradient of the log-likelihood function with respect to the parameter vector.

\[
U_\vartheta(x) = \nabla_\vartheta \log p_\vartheta(x)
\]

The Score Vector captures the sensitivity of the log-likelihood to changes in the parameter \(\vartheta\).

\subsection*{Remarks}

\begin{itemize}
    \item If the Fisher Information Matrix \(I(\vartheta)\) is large, it suggests that more information is contained in the data about the parameter \(\vartheta\), thereby enabling better estimation.
    \item The Fisher Information Matrix provides another interpretation related to the curvature of the log-likelihood function, linking it to the precision of parameter estimates.
\end{itemize}

\subsection*{Derivation of the Fisher Information}

Consider the case where the parameter space \(\Theta\) is one-dimensional, i.e., \(\Theta \subseteq \mathbb{R}\). Suppose that the density \(p_\vartheta(x)\) is twice differentiable with respect to \(\vartheta\). We derive expressions for the first and second derivatives of the log-likelihood function.

\[
\left( \log p_\vartheta(x) \right)' = \frac{p_\vartheta'(x)}{p_\vartheta(x)}
\]

\[
\left( \log p_\vartheta(x) \right)'' = \frac{p_\vartheta''(x)p_\vartheta(x) - \left( p_\vartheta'(x) \right)^2}{p_\vartheta(x)^2}
\]

Taking the expectation of the first derivative:

\[
\mathbb{E}_\vartheta \left[ \left( \log p_\vartheta(x) \right)' \right] = \int_x \frac{p_\vartheta'(x)}{p_\vartheta(x)} p_\vartheta(x) dx = \int_x p_\vartheta'(x) dx = \frac{d}{d\vartheta} \int_x p_\vartheta(x) dx = 0.
\]

This result follows from the fact that the integral of a probability density function over its entire support is always 1, and hence its derivative with respect to \(\vartheta\) is zero.

Next, we compute the expectation of the square of the first derivative:

\[
\mathbb{E}_\vartheta \left[ \left( \log p_\vartheta(x) \right)^2 \right] = - \mathbb{E}_\vartheta \left[ \left( \log p_\vartheta(x) \right)'' \right] = - \mathbb{E}_\vartheta \left[ U_\vartheta(x)^2 \right] = - I(\vartheta).
\]

This establishes the relationship between the second derivative of the log-likelihood and the Fisher Information.

\subsection*{Attaining the Cramér-Rao Bound}

The Cramér-Rao Inequality provides a lower bound on the variance of unbiased estimators. However, it is often of interest to determine when this bound is attained. The following theorem characterizes the conditions under which an estimator achieves the Cramér-Rao bound.

\begin{theorem}
    Let \((P_\theta, \theta \in \Theta)\) be a regular model with \(\Theta \subseteq \mathbb{R}\). Let \(\rho: \Theta \to \mathbb{R}\) be a continuously differentiable function, and let \(T\) be an unbiased estimator of \(\rho(\theta)\), i.e., \(\mathbb{E}_\theta[T] = \rho(\theta)\) for all \(\theta \in \Theta\).

    Then, \(T\) attains equality in the Cramér-Rao bound if and only if
    \[
    T(x) = \rho(\theta) + \rho'(\theta) I(\theta)^{-1} U_\theta(x)
    \]
    almost surely for all \(\theta \in \Theta\).
\end{theorem}

\begin{proof}
    Define \(v(\theta) = \rho'(\theta) I(\theta)^{-1}\). Consider the estimator \(T\) as given. We analyze the variance:

    \[
    0 \leq \text{Var}(T - v(\theta) U_\theta) = \text{Var}(T) + v(\theta)^2 \mathbb{E}_\theta[U_\theta^2] - 2 v(\theta) \text{Cov}_\theta(T, U_\theta)
    \]

    Substituting the definitions, we have:

    \[
    \text{Var}(T) - \rho'(\theta)^2 I(\theta)^{-1} \geq 0
    \]

    Equality holds if and only if:

    \[
    T - v(\theta) U_\theta = \text{Constant}
    \]

    Since \(T\) is unbiased, we have \(\mathbb{E}_\theta[T] = \rho(\theta)\), which implies that the constant must be \(\rho(\theta)\). Therefore:

    \[
    T = \rho(\theta) + v(\theta) U_\theta
    \]

    This establishes the necessary and sufficient condition for \(T\) to attain the Cramér-Rao bound.
\end{proof}

\begin{remark}
    \begin{enumerate}
        \item The estimator \(T(x)\) in the above equation is not always a measurable function of \(x\).
        \item If \(T\) attains the Cramér-Rao bound, we say that \(T\) is a \textbf{Cramér-Rao efficient} estimator.
    \end{enumerate}
\end{remark}

\subsection*{Corollary: Exponential Family Representation}

Under certain scaling conditions, the likelihood can be expressed in the form of an exponential family. Specifically, we have the following corollary.

\begin{corollary}
    Assume the previous scaling conditions and that \(\rho(\theta) \neq 0\) for all \(\theta \in \Theta\). Then, the likelihood can be written as:
    \[
    p_\theta(x) = c(x) \exp\left( n(\theta) T(x) - \Psi(\theta) \right)
    \]
    where:
    \begin{itemize}
        \item \(n: \Theta \to \mathbb{R}\) satisfies \(n'(\theta) = \frac{I(\theta)}{\rho'(\theta)}\).
        \item \(c(x)\) and \(\Psi(\theta)\) are functions that ensure the equality holds.
    \end{itemize}
\end{corollary}

\begin{proof}
    From the previous theorem, we have:
    \[
    T(x) = \rho(\theta) + \rho'(\theta) I^{-1}(\theta) U_\theta(x)
    \]
    Substituting \(U_\theta(x) = \frac{d}{d\theta} \log p_\theta(x)\), we obtain:
    \[
    T(x) = \rho(\theta) + \rho'(\theta) I^{-1}(\theta) \frac{d}{d\theta} \log p_\theta(x)
    \]
    Rearranging terms, we get:
    \[
    \left( T(x) - \rho(\theta) \right) \frac{I(\theta)}{\rho'(\theta)} = \frac{d}{d\theta} \log p_\theta(x)
    \]
    Integrating both sides with respect to \(\theta\) yields:
    \[
    \log p_\theta(x) = n(\theta) T(x) - \Psi(\theta) + \text{constant}(x)
    \]
    Exponentiating both sides gives:
    \[
    p_\theta(x) = c(x) \exp\left( n(\theta) T(x) - \Psi(\theta) \right)
    \]
    where \(c(x) = \exp(\text{constant}(x))\) and \(\Psi(\theta)\) absorbs any remaining terms.
\end{proof}

\subsection*{Exponential Families}

Exponential families are a broad class of probability distributions that have a particular structure, allowing for elegant mathematical properties and ease of analysis. They are central to many areas of statistics, including generalized linear models and Bayesian statistics.

\begin{definition}[Exponential Families]
    A regular statistical model \((P_\theta : \theta \in \Theta)\) is called a \textbf{$k$-parameter Exponential Family} (\(k \geq 1\)) if there exist measurable functions:
    \begin{enumerate}
        \item \(n: \Theta \to \mathbb{R}^k\)
        \item \(T: \mathcal{X} \to \mathbb{R}^k\)
        \item \(c: \mathcal{X} \to [0, \infty)\)
    \end{enumerate}
    such that the density can be expressed as:
    \[
    p_\theta(x) = c(x) \exp\left( \langle n(\theta), T(x) \rangle_{\mathbb{R}^k} - \Psi(\theta) \right)
    \]
    for all \(\theta \in \Theta\) and \(x \in \mathcal{X}\), where:
    \[
    \Psi(\theta) = \log \left( \int_{\mathcal{X}} c(x) \exp\left( \langle n(\theta), T(x) \rangle_{\mathbb{R}^k} \right) d\mu(x) \right)
    \]
\end{definition}

\begin{remark}
    \begin{enumerate}
        \item A key feature of exponential families is the factorization of the term \(\langle n(\theta), T(x) \rangle_{\mathbb{R}^k}\), which allows for the derivation of sufficient statistics and facilitates the computation of maximum likelihood estimates.
        \item Exponential forms are motivated by the quest to find general models where Cramér-Rao efficient procedures exist, leveraging the structure provided by the exponential family.
    \end{enumerate}
\end{remark}

\subsection*{Examples of Exponential Families}

\subsubsection*{Example 1: Binomial Distribution}

Consider the Binomial distribution with parameters \(n\) (number of trials) and \(\theta\) (probability of success). The probability mass function is:

\[
p_\theta(k) = \binom{n}{k} \theta^k (1 - \theta)^{n - k}
\]

This can be rewritten in exponential family form as:

\begin{align*}
p_\theta(k) &= \binom{n}{k} \exp\left( k \log \theta + (n - k) \log(1 - \theta) \right) \\
&= \underbrace{\binom{n}{k}}_{c(k)} \exp\left( \underbrace{k}_{T(k)} \underbrace{\log \left( \frac{\theta}{1 - \theta} \right)}_{n(\theta)} + \underbrace{n \log(1 - \theta)}_{\Psi(\theta)} \right)
\end{align*}

Here, we identify:
\begin{itemize}
    \item \(c(k) = \binom{n}{k}\)
    \item \(T(k) = k\)
    \item \(n(\theta) = \log \left( \frac{\theta}{1 - \theta} \right)\)
    \item \(\Psi(\theta) = n \log(1 - \theta)\)
\end{itemize}

\subsubsection*{Example 2: Normal Distribution}

Consider the Normal distribution with parameters \(\mu\) (mean) and \(\sigma^2\) (variance), denoted as \(N(\mu, \sigma^2)\). The probability density function is:

\[
p_{\mu, \sigma^2}(x) = \frac{1}{\sqrt{2\pi \sigma^2}} \exp\left( -\frac{(x - \mu)^2}{2\sigma^2} \right)
\]

This can be expressed in exponential family form by expanding the exponent:

\[
p_{\mu, \sigma^2}(x) = \frac{1}{\sqrt{2\pi \sigma^2}} \exp\left( \frac{x^2 - 2x\mu + \mu^2}{2\sigma^2} \right)
\]

By identifying appropriate functions, we have:

\[
p_{\mu, \sigma^2}(x) = \underbrace{\frac{1}{\sqrt{2\pi \sigma^2}}}_{c(x)} \exp\left( \langle n(\theta), T(x) \rangle - \Psi(\theta) \right)
\]

where:
\begin{itemize}
    \item \(T(x) = \begin{pmatrix} x^2 \\ x \end{pmatrix}\)
    \item \(n(\theta) = \begin{pmatrix} -\frac{1}{2\sigma^2} \\ \frac{\mu}{\sigma^2} \end{pmatrix}\)
    \item \(\Psi(\theta) = \frac{\mu^2}{2\sigma^2} + \frac{1}{2} \log(2\pi \sigma^2)\)
\end{itemize}

\subsubsection*{Example 3: Poisson Distribution}

Consider the Poisson distribution with parameter \(\theta > 0\). The probability mass function is:

\[
p_\theta(k) = e^{-\theta} \frac{\theta^k}{k!}
\]

This can be written in exponential family form as:

\[
p_\theta(k) = \underbrace{\frac{1}{k!}}_{c(k)} \exp\left( k \log \theta - \theta \right)
\]

Here, we identify:
\begin{itemize}
    \item \(c(k) = \frac{1}{k!}\)
    \item \(T(k) = k\)
    \item \(n(\theta) = \log \theta\)
    \item \(\Psi(\theta) = \theta\)
\end{itemize}

\subsection*{Natural Exponential Families}

Natural exponential families are a subclass of exponential families where the natural parameter space coincides with the interior of the set where the log-partition function is finite.

\begin{definition}[Natural Exponential Family]
    Define:
    \[
    \Xi = \left\{ n \in \mathbb{R}^k \ \bigg| \ \int_{\mathcal{X}} c(x) \exp\left( \langle n, T(x) \rangle_{\mathbb{R}^k} \right) d\mu(x) < \infty \right\}
    \]
    A model \((P_n \mid n \in \Xi)\) is called a \textbf{Natural Exponential Family} if:
    \begin{itemize}
        \item \(\Theta = \Xi\)
        \item The density can be written as:
        \[
        \frac{dP_n}{d\mu}(x) = c(x) \exp\left( \langle n, T(x) \rangle_{\mathbb{R}^k} - \Psi(n) \right)
        \]
        for all \(x \in \mathcal{X}\) and \(n \in \Xi\).
    \end{itemize}
\end{definition}

\begin{remark}
    A natural exponential family is fully specified by the functions \(c(x)\), \(T(x)\), and the natural parameter space \(\Xi\). This structure facilitates the derivation of properties such as sufficient statistics and conjugate priors in Bayesian analysis.
\end{remark}

\subsection*{Properties of Natural Exponential Families}

We explore some key properties of natural exponential families, particularly focusing on the relationship between the log-partition function and moments of the sufficient statistic.

\begin{lemma}
    Let \((P_n \mid n \in \Xi)\) be a 1-parameter natural exponential family. For all \(n \in \text{int}(\Xi)\),
    \begin{enumerate}
        \item \(\Psi'(n) = \mathbb{E}_n[T]\)
        \item \(\Psi''(n) = \text{Var}_n[T]\)
    \end{enumerate}
\end{lemma}

\begin{proof}
    Let \(n \in \text{int}(\Xi)\) and define:
    \[
    \gamma(n) = e^{\Psi(n)} = \int_{\mathcal{X}} c(x) \exp(n T(x)) d\mu(x)
    \]
    We first show that \(\gamma\) is infinitely differentiable at \(n\). Observe that:
    \[
    \frac{d}{dn} \left( c(x) \exp(n T(x)) \right) = c(x) T(x) \exp(n T(x))
    \]
    To apply the Dominated Convergence Theorem (DCT), we require that for some \(\varepsilon > 0\), the function \(c(x) T(x) \exp(n T(x))\) is dominated by an integrable function. Specifically, for sufficiently small \(\varepsilon\), we have:
    \[
    T(x) \exp(n T(x)) \leq C \exp((n + \varepsilon) T(x))
    \]
    for some constant \(C > 0\), ensuring integrability.

    Applying DCT, we differentiate \(\gamma(n)\):
    \[
    \gamma'(n) = \int_{\mathcal{X}} c(x) T(x) \exp(n T(x)) d\mu(x) = \mathbb{E}_n[T]
    \]
    Therefore, the derivative of \(\Psi(n)\) is:
    \[
    \Psi'(n) = \frac{d}{dn} \log \gamma(n) = \frac{\gamma'(n)}{\gamma(n)} = \mathbb{E}_n[T]
    \]

    Differentiating again, we obtain:
    \[
    \Psi''(n) = \frac{d}{dn} \mathbb{E}_n[T] = \text{Var}_n[T]
    \]
    This follows from the properties of exponential families, where the second derivative of the log-partition function corresponds to the variance of the sufficient statistic.
\end{proof}

\begin{remark}
    The lemma highlights the intimate connection between the derivatives of the log-partition function \(\Psi(n)\) and the moments of the sufficient statistic \(T\). Specifically, the first derivative gives the mean, and the second derivative gives the variance, providing a direct link between the parameters of the exponential family and the moments of the data.
\end{remark}



















\iffalse


\appendix

\numberwithin{definition}{section}
\numberwithin{theorem}{section}
\numberwithin{proposition}{section}
\numberwithin{lemma}{section}
\numberwithin{corollary}{section}
\numberwithin{remark}{section}
\numberwithin{example}{section}
\numberwithin{supplement}{section}

%%%%%%%%%%%%%%%%%%%%%%%%%%%%%%%%%%%%%%%%%%%%%%%%%%%%%%%%%%%%%%%%%%%%%%%%%%%%%%%
%%%%%%%%%%%%%%%%%%%%%%%%%%%%%%%%%%%%%%%%%%%%%%%%%%%%%%%%%%%%%%%%%%%%%%%%%%%%%%%
%%%%%%%%%%%%%%%%%%%%%%%%%%%%%%%%%%%%%%%%%%%%%%%%%%%%%%%%%%%%%%%%%%%%%%%%%%%%%%%
\chapter{Appendix to the chapters} \label{ch:appendix}
%%%%%%%%%%%%%%%%%%%%%%%%%%%%%%%%%%%%%%%%%%%%%%%%%%%%%%%%%%%%%%%%%%%%%%%%%%%%%%%
%%%%%%%%%%%%%%%%%%%%%%%%%%%%%%%%%%%%%%%%%%%%%%%%%%%%%%%%%%%%%%%%%%%%%%%%%%%%%%%
%%%%%%%%%%%%%%%%%%%%%%%%%%%%%%%%%%%%%%%%%%%%%%%%%%%%%%%%%%%%%%%%%%%%%%%%%%%%%%%
%%%%%%%%%%%%%%%%%%%%%%%%%%%%%%%%%%%%%%%%%%%%%%%%%%%%%%%%%%%%%%%%%%%%%%%%%%%%%%%
\section{Introducing the problem}\label{sec: appendix to chap 1}
%%%%%%%%%%%%%%%%%%%%%%%%%%%%%%%%%%%%%%%%%%%%%%%%%%%%%%%%%%%%%%%%%%%%%%%%%%%%%%%

.....

\fi 

\medskip
\end{document}